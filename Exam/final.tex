\documentclass[]{exam}  % use the 'answer' option to compile solutions
\usepackage{subfiles}
\usepackage{marktext} 
\usepackage{svmacro}
\usepackage[utf8]{inputenc}
\usepackage{lastpage}
\usepackage[american,siunitx]{circuitikz}
\sisetup{detect-all}
\usepackage{amsmath}

\newtheorem{problem}{Problem}

\newcommand{\mb}[1]{\mathbf{#1}} %mathbf
\begin{document}
	\title{MATH 101 Calculus\\ Final Exam}
	\date{December 12th, 2023}
	
	\begin{center}
		\Large
		{\bfseries\makeatletter\@title\makeatother}
		
		\bigskip
		
		Fulbright University Vietnam
		
		\makeatletter\@date\makeatother
		
%		\ifanswer
		\vspace{1pc}
		
%		{\sffamily\bfseries SOLUTIONS}
		
		\vspace{1pc}
%		\else
		\vfill
		
		\begin{tabular}{rp{10cm}}
			Student Full Name: & \hrulefill \\[1pc]
			Student ID: & \hrulefill \\[1pc]
%			Number of extra pages: & \hrulefill 
		\end{tabular}
		
		\vfill
%		\fi
		
%		\bigsize
		\begin{tabular}{| >{\bfseries}c | c | c | }
			\hline
			Question & \bfseries Maximal Points &\bfseries Your Points\\
			\hline
			\hline
			% You need to fill this table in yourself, sadly.
			1 &  25 &\\ \hline
			2 &  25 &\\ \hline
			3 &  25 &\\ \hline
			4 &  25 &\\ \hline
			\hline
			Total & 100 & \\
			\hline
		\end{tabular}
		
	\end{center}
	
%	\ifanswer\else
	
	\vfill
	
	\textbf{Instructions:}
	
	\begin{enumerate}
		\item You have \textbf{eighty (80) minutes} to complete the exam.  
		\item This exam contains 4 problems. Some pages are intentionally left blank for you to write your solutions. If you run out of space you can use the backs of the pages, but please indicate clearly on the front pages that you have done so.
		
%		\item There are 4 problems
		
		\item Phones are not allowed. Please turn off your phones during the exam.
            If you need to use the restroom, please turn in your phones before leaving the room.
		
		\item \textbf{Show and justify your work unless indicated otherwise.}

		\item Please sign the Honor Code statement below.
	\end{enumerate}
	
	\fbox{\parbox{\textwidth}{
			In recognition of and in the spirit of the Fulbright University Vietnam Honor Code, I certify that I will neither give nor receive unpermitted aid on this examination.
			
			\vspace{1cm}
			
			\hspace*{\fill}
			Signature: \quad \rule{10cm}{0.5pt}
			\hspace*{\fill}
			
			\bigskip
	}}


    \newpage
\begin{problem}
    Suppose that $A = \{1,2,3\}$, $B = \{4,5\}$, $C = \{6,7,8\}$, $R = \{(1,7), (3,6), (3,7)\}$, and $S = \{(4,7), (4,8), (5,6)\}$. Note that $R$ is a relation from $A$ to $C$, and $S$ is a relation from $B$ to $C$. 

    \begin{enumerate}
        \item Represent the relations via sets and arrows.
        \item Write down in the set notation the following relation: 
         $S^{-1} \circ R$.
    \end{enumerate}
    \newpage

\end{problem}

    \begin{problem}
        True or False? Give some reason for your answers.
        \begin{enumerate}
            \item $\emptyset \in \set{\set{ \emptyset}}$
            \item $\set{\emptyset} \in \set{\set{\emptyset}}$
            \item $\cP(\emptyset) \in \cP(\cP(\emptyset))$
            \item $\cP(\cP (\emptyset )) \in \set{\emptyset, \set{\emptyset, \set{\emptyset}}}$
        \end{enumerate}
    \end{problem}

    \newpage
    \begin{problem}
        Prove by induction that $\forall n \in \N$, we have
        \begin{equation*}
            1^2 + 2^2 + \dots + n^2 = \frac{1}{6} n (n + 1)(2n + 1) \,.
        \end{equation*}
    \end{problem}


    \newpage

    \begin{problem}
        \begin{enumerate}
            \item Let $f:A \to B$ and $g:B\to C$ be functions.
        Prove that if $f$ and $g$ are both injective, then so is $g\circ f$.
            \item let $f:(0,\infty) \to (0,\infty)$ and $g:(0,\infty) \to (0,\infty)$
                be defined as
                \begin{equation*}
                    f(x) = x^3  \,,
                \end{equation*}
                and
                \begin{equation*}
                    g(x) = \frac{1}{x} \,.
                \end{equation*}
                \begin{enumerate}
                    \item Find a formula for $g \circ f$.
                    \item Is $g \circ f$ injective and why?
                \end{enumerate}
        \end{enumerate}
    \end{problem}



\end{document}
