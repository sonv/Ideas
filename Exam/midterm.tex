\documentclass[12pt]{amsart}
\usepackage{amsaddr}
\usepackage{marktext} 
%% Remove draft for real article, put twocolumn for two columns
\usepackage{svmacro}
\usepackage[utf8]{inputenc}
\usepackage{enumitem}
\usepackage[style=alphabetic, backend=biber]{biblatex}
\addbibresource{bibliography.bib}

%% commentary bubble
\newcommand{\SV}[2][]{\sidenote[colback=green!10]{\textbf{SV\xspace #1:} #2}}

%% Title 
\title{ MATH 102: MIDTERM }
\author{Good Luck!}

%\author{Co-author}
%\address{  }
%\email {  }
%
\date{\today}

\begin{document}

\maketitle


There are five questions. Make sure you justify all your work for complete credit.

\section*{Rules}

\begin{itemize}[leftmargin=*]
    \item You have 90  minutes to complete your work..
    \item Closed books.
    \item No use of internet, textbooks, computer algebra systems, calculators. 
    \item No collaboration.
    \item 1 person per bathroom break. When you go to the bathroom, turn in your cellphone and exam until return.
\end{itemize}

\newpage

\section*{Questions}

\begin{enumerate}[label=\arabic*.,itemsep=10pt, leftmargin=*]
    \item
        \begin{enumerate}
            \item \textit{[10 points.]} Use the truth table to show that
        \begin{equation*}
            P\implies Q \equiv (\neg P) \vee  Q \,.
        \end{equation*}
    \item \textit{[10 points.]} Find an equivalent expression of 
        \begin{equation*}
            \neg( P \implies Q) 
        \end{equation*}
        and use the truth table to check it.
        \end{enumerate}
\newpage
    \item 
        The symmetric difference of two sets $A$ and $B$ is defined as follows
        \begin{equation*}
            A \triangle B = (A\setminus B) \cup (B\setminus A) \,.
        \end{equation*}
        \begin{enumerate}
            \item \textit{[10 points.]}
                Use the Venn diagram to represent $A \triangle B$.
            \item \textit{[10 points.]}
                Let $A = \set{1,2,3,4,5,6,7}$ and $B = \set{2,4,6}$. What is $A\triangle B$?
        \end{enumerate}
        \newpage
        \item \textit{[20 points.]} For any sets $A$ and $B$, prove that
                \begin{equation*}
                    A \triangle B \subseteq A \cup B \,.
                \end{equation*}
                Hint: Start with ``Let $x$...'' and use definitions of the operations.

                \newpage
                \item\textit{[20 points.]}
                Analyze the logical forms of the following statements. The universe of discourse is $\mathbb{R}.$ What are the free variables in each statement?
\begin{enumerate}
    \item Every number that is larger than $x$ is larger than $y$. \\
    \item For every number $a$, the equation $ax^2+4x-2=0$ has at least one solution if $a\ge -2.$\\
    \item All solutions of the inequality $x^3-3x<3$ are smaller than $10.$\\
    \item If there is a number $x$ such that $x^2+5x=w$ and there is a number $y$ such that $4-y^2=w$, then $w$ is between $-10$ and $10.$\\
\end{enumerate}

\newpage
    \item 
        \begin{enumerate}
            \item 
        \textit{[10 points.]}
            Translate the following statement into a logical formula with predicates and quantifiers.
                
                ``If all humans are heroes all the time then no humans are heroes anytime.''
                
                Hint: Define $X$ to be set of humans and $T$ set of time.
            Suggested variables: $x$ for humans, 
            $t$ for time.
            Suggested predicate: 
            $P(x,t)=$``$x$ is a hero in time $t$''.
            

            \item 
        \textit{[5 points.]}
        \textit{
        A logical formula is said to be simplest if the negation signs (if there is any) 
        are right in front of the predicate and NOT in front of quantifiers. For example,
        ``$ \neg (\forall x \in X, P(x))$'' is NOT simplest, 
    but ``$\exists x \in X, \neg P(x)$'' is simplest.}

            Find the simplest logical formulae 
                to express the negation of the statement in (a).  
                
                Hint: Use Question 1 recall  $\neg (\forall x\in X, P(x)) \equiv \exists x\in X, \neg P(x)$
                and
                $\neg (\exists x\in X, P(x)) \equiv \forall x\in X, \neg P(x)$.
            \item
        \textit{[5 points.]}
            Translate the resulting formula in (b) back into English. 

        \end{enumerate}
\end{enumerate}



\printbibliography 
%\bibliography{refs}
%\bibliographystyle{halpha-abbrv}


\end{document}
