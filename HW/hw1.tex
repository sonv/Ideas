\documentclass[12pt]{amsart}
\usepackage{marktext} 
%% Remove draft for real article, put twocolumn for two columns
\usepackage{svmacro}
\usepackage[utf8]{inputenc}
\usepackage{lineno}
\usepackage[style=alphabetic, backend=biber]{biblatex}
\addbibresource{bibliography.bib}

%% commentary bubble
\newcommand{\SV}[2][]{\sidenote[colback=green!10]{\textbf{SV\xspace #1:} #2}}

%% Title 
\title{ MATH 102: Homework 1 (Solution)}
\author{Graded work turned in by Thursday, September 5}

%\author{Co-author}
%\address{  }
%\email {  }
%
\date{\today}

\begin{document}

\maketitle

\begin{problem}[Exercise 1.2]
The error comes from the third line of the proof, where
the author divides both sides of the equality by $(x-y)$.
This is illegal because $x-y = 0$.
\end{problem}

\begin{problem}[Exercise 1.3]
\begin{enumerate}
	\item No. The number of squares on the board would be $m\times n$,
	      which would be odd. However, to perfectly cover
	      the board, one needs an even number of square.
	\item It depends! Since the number of squares of $m\times n$ board
	      is odd, we suppose that there is one white square more than black squares.

		      {\bf Case 1.} If we remove a white square, then we can construct an example
	      that you can fill in the box (give one!). Be careful that you can remove any white square
	      on the board, not just from the corner. Your construction needs to take this into account.

		      {\bf Case 2.} If we remove a black square, then there are two white squares more than
	      black squares. This imbalance implies that we cannot perfectly cover the board.
\end{enumerate}
\end{problem}

\begin{problem}[Exercise 1.15]
(Quite a few people thought about this idea but couldn't pursue to the end or
there are something unclear in the proofs.)

First, we claim that any odd integer $m$ in $\set{1,2, \dots, 3n}$ can be written as $k 3^i$, for
some odd number $k$ in $\set{1,2, \dots, 3n}$ not divisible by $3$ and some number $i$ in $\set{0, 1, \dots}$.
To see this, we note that if $m$ is not divisible by $3$, then $m = m3^0 $.
On the other hand, if $m$ is divisible by $3$, then $ m = 3 s$. Keep dividing $s$ by $3$ until
you cannot do it anymore withouth resulting in non-integer.
At the end of this process, we have $m = 3^ik$ for $i\geq 1$ and $k$ odd not divisible by $3$.

For each odd number $k$ in $\set{1, 2, \dots, 3n}$ not divisible by $3$, we denote
\begin{align*}
	A_k = \set{k, 3k, 3^2 k, 3^3 k, \dots}\,.
\end{align*}

We claim that there are only at most $n$ such $k$ in $\set{1, 2, \dots, 3n}$ (odd and not divisible by $3$).
To see this, we first divide the list into $n$ sublists
\begin{align*}
	B_1 & = \set{ 1,2,3}         \\
	B_2 & = \set{4,5,6}          \\
	    & \vdots                 \\
	B_n & = \set{3n-2, 3n-1, 3n}
\end{align*}
Consider a set $B_i = \set{3i-2, 3i-1, 3i}$ among those sets.
Note that, the only number in $B_i$ that is divisible by $3$ is $3i$.
Remove $3i$ from it gives us $\set{3i-2, 3i-1}$, which contains only numbers
that don't divide $3$. For any two consecutive numbers, one must be even and the
other one must be odd. Remove the even number from $\set{3i-2 , 3i-1}$
gives you a set $C_i$ that contains only odd number that is not divisible by $3$.

Since each $B_i$ produces one $C_i$, there are $n$ $B_i$'s, there are $n$
odd numbers that is not divisible by $3$.
Hence, there are total $n$ sets $A_k$ defined above.
By pigeonhole principle, if we pick $n+1$ odd numbers,
at least two of them must belong to one of the $A_k$'s
together.
The larger number, say $M$, must be divisible by the smaller number, say $m$,
$M = 3^i m$, where $i\geq 1$, by construction.
\end{problem}

\begin{problem}[Exercise 1.20](This is taken from Khanh Linh's solution.)

There are $n$ people at the party, so each person can be acquainted with $1$ to $n - 1$
other people.
Hence, there are $n - 1$ possible number of acquaintances. View each of these $n - 1$
cases as a box.
Since there are $n = (n - 1) + 1$ people and $n - 1$ boxes,
there exists 1 box with
at least 2 people.
Thus, there are at least two people at this party who have the same number of
acquaintances.
\end{problem}



\printbibliography
%\bibliography{refs}
%\bibliographystyle{halpha-abbrv}


\end{document}
