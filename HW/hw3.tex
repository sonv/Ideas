\documentclass[12pt]{amsart}
\usepackage{marktext} 
%% Remove draft for real article, put twocolumn for two columns
\usepackage{svmacro}
\usepackage[utf8]{inputenc}
\usepackage{lineno}
\usepackage[style=alphabetic, backend=biber]{biblatex}
\addbibresource{bibliography.bib}

%% commentary bubble
\newcommand{\SV}[2][]{\sidenote[colback=green!10]{\textbf{SV\xspace #1:} #2}}

%% Title 
\title{ MATH 102: Homework 3}
\author{Solution}

%\author{Co-author}
%\address{  }
%\email {  }
%
\date{\today}

\begin{document}

\maketitle

Grade: 3.1, 3.8, 3.9, 3.12, 3.14, 3.15, 3.21, 3.22

\begin{problem}[Exercise 3.1]
\begin{enumerate}[a.]
	\item $\{ \{ \emptyset \} \}$
	\item $ \{ \{ \{ Porco \}  \}  \}$
	\item $ \{ Dragon, \{ \{ Porco \}\}\}$
	\item $ \{ Tofu, \{Dragon, \{ Porco \}\}\}$
	\item $ \{  \{ Dragon,\emptyset\}\}$
	\item $ \{ Porco,  \{ \emptyset\}\}$
\end{enumerate}
\end{problem}

\begin{problem}[Exercise 3.8]
(a)T, (b)F, (c)F, (d)T, (e)T, (f)T, (g)F, (h)T, (i)T, (j)F,
(k)T, (l)T, (m)T, (n)F, (o)T
\end{problem}

\begin{problem}[Exercise 3.9]
\begin{enumerate}
	\item $\{ \emptyset, \{1\}, \{2\}, \{3\}, \{1,2\}, \{1,3\}, \{2,3\}, \{1,2,3\}\}$
	\item $ \{ \emptyset, \{ \N \}, \{\{\Q, \R \}\}, \{ \N, \{\Q, \R \}\} \}$
	\item $ \{\emptyset, \{ \N \}, \{\Q\}, \{ \R \}, \{\N, \Q\}, \{ \N, \R\}, \{\Q, \R\}, \{\N, \Q, \R \} \}$
	\item $\{ \emptyset \}$
\end{enumerate}
\end{problem}


\begin{problem}[Exercise 3.12]
Prove
\begin{equation*}
	( A \cap B)^c = A^c \cup B^c \,.
\end{equation*}
\end{problem}
\begin{proof}
	We need to prove two ways.
	\begin{enumerate}[1.]
		\item First we need to show $ ( A \cap B)^c \subseteq A^c \cup B^c $.

		      Let $x \in (A \cap B)^c$.
		      This means $ x \not\in A \cap B$, i.e,
		      $x \not \in A$ and $x \not \in B$.
		      Therefore, $x\in A^c$ and $x \in B^c$.
		      So, $x \in A^c$ and therefore, $x \in A^c \cup B^c$.
		      Therefore, $ ( A \cap B)^c \subseteq A^c \cup B^c $.
		      \\
		\item Then we need to show $ ( A \cap B)^c \supseteq A^c \cup B^c $.

		      Let $x \in A^c \cup B^c$.
		      Therefore, $x \in A^c$ or $x \in B^c$. There are two cases.

		      Case 1. If $x \in A^c$, then $x \not\in A$. Therefore,
		      $x \not\in A \cap B$. Thus, $x\in (A \cap B)^c$.

		      Case 2. If $x \in B^c$, then $x \not\in B$. Therefore,
		      $x \not\in A \cap B$. Thus, $x\in (A \cap B)^c$.

		      From both cases, we have
		      $x\in (A \cap B)^c$.
		      Therefore, $ ( A \cap B)^c \supseteq A^c \cup B^c $
	\end{enumerate}
\end{proof}


\begin{problem}[Exercise 3.14]
\begin{enumerate}
	\item $(A \backslash B) \cup (B \cap C)$
	\item $  (A \cap C ) \backslash B $
	\item $(A \cup B \cup C ) \backslash (A \cap B \cap C) $
\end{enumerate}
\end{problem}

\begin{problem}[Exercise 3.15]
Let $A$ and $B$ be sets. Prove
\begin{equation*}
	\cP(A) \cup \cP(B) \subseteq \cP(A \cup B) \,.
\end{equation*}
For the example, come up with your own and check.
\end{problem}

\begin{proof}
	Let $X \in \mathcal{P}(A) \cup \mathcal{P}(B)$.
	This means $X \in \mathcal{P}(A)$ or $X \in \mathcal{P}(B)$.
	There are two cases.

	Case 1. $X \in \mathcal{P}(A)$. Then, $X \subset A$.
	Therefore, $X \subset A \cup B$.
	By definition, $X \in \mathcal(A \cup B)$.

	Case 2. $X \in \mathcal{P}(B)$. Then, $X \subset B$.
	Therefore, $X \subset A \cup B$.
	By definition, $X \in \mathcal(A \cup B)$.

	In both cases, we have $X \in \mathcal(A \cup B)$.
	Therefore,
	$ \cP(A) \cup \cP(B) \subseteq \cP(A \cup B)$.
\end{proof}

\begin{problem}[Exercise 3.21]
Prove
\begin{equation*}
	\{ n\in \Z : 2 \mid n \} \cap \{ n \in \Z: 9 \mid n\}
	\subseteq
	\{ n \in \Z: 6 \mid n \} \,.
\end{equation*}
\end{problem}

\begin{proof}
	Let $x \in  \{ n\in \Z : 2 \mid n \} \cap \{ n \in \Z: 9 \mid n\} $.
	Then, it must be true that $2 \mid x$ and $9 \mid x$.
	By definition, $x = 2n$ for some $n\in \Z$ and $x = 9m$ for
	some $m \in \Z$.
	So, $2n = 9m$ and $2 \mid 9m$.
	By Question 4 in Worksheet 4, either $2 \mid 9$ or $2 \mid m$.
	Since it is for sure that $2 \nmid 9$, it must be the case that
	$2\mid m$.

	Therefore, $m = 2k$.
	This means that
	\begin{equation*}
		x = 3m = 3\cdot 2k = 6k \,.
	\end{equation*}
	So, $6 \mid x$ and $x \in  \{ n \in \Z: 6 \mid n \}$.
\end{proof}

\begin{problem}[Exercise 3.22]
Prove
\begin{equation*}
	\{ (m,n) \in \Z \times \Z: m \equiv n \, (\mathrm{mod} 6) \}
	\subseteq
	\{ (m,n) \in \Z \times \Z: m \equiv n \, (\mathrm{mod} 2) \} \,.
\end{equation*}
\end{problem}

\begin{proof}
	Let $(x,y) \in \{ (m,n) \in \Z \times \Z: m \equiv n \, (\mathrm{mod} 6) \}$.
	Then $x\equiv y \, (\mathrm{mod} 6)$.
	By definition,
	$6 \mid (x - y)$ and so $x - y = 6k$ for some $k \in \Z$.
	But then,
	\begin{equation*}
		x - y = 2\cdot (3k) \,.
	\end{equation*}
	Therefore, $2 \mid (x-y)$ and $x \equiv y \, (\mathrm{mod} 2)$.
	Thus, $(x,y) \in
		\{ (m,n) \in \Z \times \Z: m \equiv n \, (\mathrm{mod} 2) \}$. So,
	\begin{equation*}
		\{ (m,n) \in \Z \times \Z: m \equiv n \, (\mathrm{mod} 6) \}
		\subseteq
		\{ (m,n) \in \Z \times \Z: m \equiv n \, (\mathrm{mod} 2) \} \,.
	\end{equation*}
\end{proof}

%\bibliography{refs}
%\bibliographystyle{halpha-abbrv}


\end{document}
