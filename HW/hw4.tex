\documentclass[12pt]{amsart}
\usepackage{marktext} 
%% Remove draft for real article, put twocolumn for two columns
\usepackage{svmacro}
\usepackage[utf8]{inputenc}
\usepackage{lineno}
\usepackage[style=alphabetic, backend=biber]{biblatex}
\addbibresource{bibliography.bib}

%% commentary bubble
\newcommand{\SV}[2][]{\sidenote[colback=green!10]{\textbf{SV\xspace #1:} #2}}

%% Title 
\title{ MATH 102: Homework 4}
\author{Due date: Tuesday, Oct 3}

%\author{Co-author}
%\address{  }
%\email {  }
%
\date{\today}

\begin{document}

\maketitle

Note that you need to turn in \LaTeX version of this homework.

We briefly mentioned truth sets in class.
Here's the definition from the textbook.
\begin{definition}
    The truth set of a statement $P(x)$ is the set of all values of $x$
that make the statement $P(x)$ true. In other words, it is the set defined by using
the statement $P(x)$ as an elementhood test:
Truth set of $P(x) = {x | P(x)}$.
\end{definition}

In Velleman:

Section 1.3: 4, 5, 6, 8.

Section 2.1: 3, 4, 6.

(For the problems in Section 2.1 ``Universe of discourse'' is the same with ``Domain of disclosure'' mentioned in class)







\printbibliography 
%\bibliography{refs}
%\bibliographystyle{halpha-abbrv}


\end{document}
