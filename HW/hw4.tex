\documentclass[12pt]{amsart}
\usepackage{marktext} 
%% Remove draft for real article, put twocolumn for two columns
\usepackage{svmacro}
\usepackage[utf8]{inputenc}
\usepackage{lineno}
\usepackage[style=alphabetic, backend=biber]{biblatex}
\addbibresource{bibliography.bib}

%% commentary bubble
\newcommand{\SV}[2][]{\sidenote[colback=green!10]{\textbf{SV\xspace #1:} #2}}

%% Title 
\title{ MATH 102: Homework 4}
\author{Due date: Thursday, Oct 24}

%\author{Co-author}
%\address{  }
%\email {  }
%
\date{\today}

\begin{document}

\maketitle

Note that you need to turn in \LaTeX version of this homework.

Do Problem 1 below AND the problems from the book
\begin{problem}
In class on Oct 8, we discussed about inflation and present value.
In short, if someone gives you $F$ amount of dollars every year
and the inflation rate is $r$, then the present value of $N$ year
of this fixed income would be
\begin{equation*}
	PV = F + \frac{F}{1 + r} + \dots + \frac{F}{(1 + r)^N} \,.
\end{equation*}

\begin{enumerate}
	\item Use induction to prove that after $N$ year,
	      \begin{equation*}
		      PV = F \left(\frac{ 1 + r - \frac{1}{(1+r)^{N}} }{r} \right)
	      \end{equation*}
	\item Apply this to the situation when $F = \$1$ and $r = 5\%$ (I made a mistake in
	      class about this)
	\item What happen if you were to live forever? How much would you buy a stock if
	      it pays you $\$1$ per year from now to eternity?
\end{enumerate}

\begin{proof}
	We shall proceed by induction.
	Step 1: Base case. For $N =1$, we have
	$$ F = F \frac{ 1 + r - 1}{r}.$$

	Step 2: Induction hypothesis. Suppose that the statement is true for $N = k$, i.e.,
	$$ F + \dots + \frac{F}{(1+r)^k} = F\left( \frac{ 1 + r - \frac{1}{(1+r)^k}}{r} \right)$$

	Step 3: Inductive step. We need to show that the statement is true for $N = k+1$.
	To do this, we consider
	\begin{align*}
		 & F + \dots + \frac{F}{(1+r)^k} + \frac{F}{(1+r)^{k+1}}                          \\
		 & = F\left( \frac{ 1 + r - \frac{1}{(1+r)^k}}{r} \right) + \frac{F}{(1+r)^{k+1}} \\
		 & = \frac{ F\left(1 + r - \frac{1}{(1+r)^k} \right) + \frac{rF}{(1+r)^{k+1}}}{r} \\
		 & = F\frac{ 1 + r - \frac{1}{(1+r)^k}  + \frac{r}{(1+r)^{k+1}}}{r}               \\
		 & = F\frac{ 1 + r + \frac{-1-r + r}{(1+r)^{k+1}} }{r}                            \\
		 & = F\frac{ 1 + r - \frac{1}{(1+r)^{k+1}} }{r}                                   \\
	\end{align*}
	as desired.

	Step 4: Conclusion. From all of the above steps, we have shown by induction that
	\begin{align*}
		 & F + \frac{F}{1 + r} + \dots + \frac{F}{(1 + r)^N}         \\
		 & = F \left(\frac{ 1 + r - \frac{1}{(1+r)^{N}} }{r} \right)
	\end{align*}
	for every $N \in \N$.
\end{proof}

\end{problem}

\section*{Problems from the file chapter4-hw.pdf}

Try: 4.4, 4.5, 4.7, 4.21, 4.23

Grade: 4.4.f, 4.5.a,b, 4.7, 4.21.b, 4.23.a

\begin{problem}[4.4.f]
We shall proceed by induction.

Step 1: Base Case.
For \( n = 1 \), we check whether the equation holds:
\[
	1 \cdot 1! = 1
\]

Step 2: Inductive Hypothesis.
Assume that the statement holds for some \( n = k \), i.e.,
\[
	1 \cdot 1! + 2 \cdot 2! + \cdots + k \cdot k! = (k+1)! - 1
\]

Step 3: Inductive Step.
We need to show that the statement also holds for \( n = k + 1 \). Starting from the inductive hypothesis, we need to prove:
\[
	1 \cdot 1! + 2 \cdot 2! + \cdots + (k+1) \cdot (k+1)! = (k+2)! - 1
\]

Using the inductive hypothesis, we can write the left-hand side as:
\[
	\left(1 \cdot 1! + 2 \cdot 2! + \cdots + k \cdot k!\right) + (k+1) \cdot (k+1)!
\]
Substitute the inductive hypothesis:
\[
	(k+1)! - 1 + (k+1) \cdot (k+1)!
\]

Factor out \( (k+1)! \):
\[
	(k+1)! \left(1 + (k+1)\right) - 1
\]
\[
	(k+1)! \cdot (k+2) - 1
\]
\[
	(k+2)! - 1
\]

Thus, the statement holds for \( n = k+1 \).

Step 4: Conclusion.
By induction, the formula
\[
	1 \cdot 1! + 2 \cdot 2! + \cdots + n \cdot n! = (n+1)! - 1
\]
is true for all \( n \in \N \).
\end{problem}

\begin{problem}[4.5.a]
We proceed by induction.

Step 1: Base Case.
For $n = 1$, we check if the inequality holds:
\[
	n + 2 = 1 + 2 = 3
\]
\[
	4n^2 = 4(1)^2 = 4
\]
Clearly, $3 < 4$, so the inequality holds for $n = 1$.

Step 2: Induction hypothesis.
Now, assume the inequality holds for some $n = k$, i.e.,
\[
	k + 2 < 4k^2
\]


Step 3: Inductive step.
We need to show that the inequality holds for $n = k + 1$. That is, we want to prove:
\[
	(k + 1) + 2 < 4(k + 1)^2 \,.
\]

We have
\begin{equation*}
	k + 3  = (k+2) +1 < 4k^2 + 1 < 4k^2 + 8k + 4 = 4(k+1)^2 \,.
\end{equation*}
Thus, the inequality holds for $n=k+1$.

Step 4: Conclusion.
By induction, the inequality $n + 2 < 4n^2$ holds for all $n \in \N$.

\end{problem}

\begin{problem}[4.5.b]
We proceed by induction.

Step 1: Base case.
For $n = 1$,
$$ \frac{1}{\sqrt 1} \leq 2 \sqrt 1 - 1 \,.$$

Step 2: Induction hypothesis.
Suppose the inequality holds for $n = k$, i.e.,
$$ \frac{1}{\sqrt{1}} + \dots + \frac{1}{\sqrt k} \leq 2 \sqrt k - 1 \,. $$


Step 3: Inductive step.
We need to show that the inequality holds for $n = k+1$.
To do this, consider
\begin{align*}
	 & \frac{1}{\sqrt{1}} + \dots + \frac{1}{\sqrt k} + \frac{1}{\sqrt{k+1}} \\
	 & \leq 2 \sqrt{k} - 1 + \frac{1}{\sqrt{k+1}}                            \\
	 & = \frac{ 2 \sqrt{k(k+1)} - \sqrt{k+1} +1}{\sqrt{k+1}}                 \\
	 & \leq \frac{ 2 (k+1) - \sqrt{k+1}}{\sqrt{k+1}}                         \\
	 & = 2 \sqrt{k+1} -1 \,.
\end{align*}

Step 4: Conclusion.
By induction,
it is true that
$$ \frac{1}{\sqrt{1}} + \dots + \frac{1}{\sqrt n} \leq 2 \sqrt n - 1 \,,$$
for every $n\in \N$.

\end{problem}


\begin{problem}[4.7]
We proceed by induction.

Step 1: Base case.  For $n= 0$,
$0^2 < 3^0$.
For $n = 1$, $1 < 3$.

Step 2: Induction hypothesis. Suppose the inequality is true for $n = k\geq 2$, i.e.,
\begin{align*}
	k^2 < 3^k \,.
\end{align*}

Step 3: Inductive step. We need to show the inequality is true for $n= k+1$.
To do this, we consider
\begin{align*}
	(k+1)^2 & = k^2 + 2k +1 < 3^k + 2k +1\,.
\end{align*}
Note that when $k\geq 2$, $2k < k^2$ and $1 < k^2$.
Therefore,
\begin{align*}
	(k+1)^2 & < 3^k + 2k +1 < 3*3^k = 3^{k+1}\,.
\end{align*}

Step 4: Conclusion. By induction,
\begin{equation*}
	n^2 < 3^n
\end{equation*}
for $n \geq 0$.
\end{problem}


%\bibliography{refs}
%\bibliographystyle{halpha-abbrv}


\end{document}
