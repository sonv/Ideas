\documentclass[12pt]{amsart}
\usepackage{marktext} 
%% Remove draft for real article, put twocolumn for two columns
\usepackage{svmacro}
\usepackage[utf8]{inputenc}
\usepackage{lineno}
\usepackage[style=alphabetic, backend=biber]{biblatex}
\addbibresource{bibliography.bib}

%% commentary bubble
\newcommand{\SV}[2][]{\sidenote[colback=green!10]{\textbf{SV\xspace #1:} #2}}

%% Title 
\title{ MATH 102: Homework 6}
\author{Due date: Friday, Dec 8, at 11:59pm }

%\author{Co-author}
%\address{  }
%\email {  }
%
\date{\today}

\begin{document}

\maketitle

\begin{problem}
    Do Problem 4 in Worksheet 11.
\end{problem}

\begin{problem}
    \begin{enumerate}
        \item 
    What's wrong with the following proof that for all \(n \in \mathbb{N}\), 
\[ 1 \cdot 3^0 + 3 \cdot 3^1 + 5 \cdot 3^2 + \ldots + (2n+1) \cdot 3^n = n \cdot 3^{n+1}? \]

\textbf{Proof.} We use mathematical induction. Let \(n\) be an arbitrary natural number, and suppose that 
\[ 1 \cdot 3^0 + 3 \cdot 3^1 + 5 \cdot 3^2 + \ldots + (2n + 1)3^n = n3^{n+1}. \]
Then
\begin{align*}
&1 \cdot 3^0 + 3 \cdot 3^1 + 5 \cdot 3^2 + \ldots + (2n+1)3^n + (2n+3)3^{n+1} \\
&= n3^{n+1} + (2n + 3)3^{n+1} \\
&= (3n + 3)3^{n+1} \\
&= (n + 1)3^{n+2}.
\end{align*}

\item Find the correct formula for  
\[ 1 \cdot 3^0 + 3 \cdot 3^1 + 5 \cdot 3^2 + \ldots + (2n+1) \cdot 3^n = n \cdot 3^{n+1} \]
    and prove it.
    \end{enumerate}
\end{problem}

\begin{problem}
    Prove by induction that for all $n\in \N$, 
    \begin{enumerate}
        \item 
    \begin{equation*}
        6 \vert (n^3 - n) \,.
    \end{equation*}
\item \begin{equation*}
        1+2^3 + \dots + n^3 = \frac{n^2(n+1)^2}{4}\,.
\end{equation*}
    \end{enumerate}
\end{problem}

\begin{problem}
    Recall factorial from high school:
    \begin{equation*}
        n! = 1\cdot 2\cdot \dots \cdot n \,.
    \end{equation*}
    We also define the special case $0! =1$.

    Next, we use this to define the concept of combination:
    \begin{equation*}
        {n \choose k }= \frac{n!}{k!(n-k)!} \,.
    \end{equation*}

    \begin{enumerate}
    \item Prove the following formula (without induction) 
            \begin{equation*}
                {n+ 1 \choose k+1 } = {n \choose k } + {n \choose k+1}
            \end{equation*}
        \item Prove the following formula using induction
            \begin{equation*}
                {n \choose 0} + {n \choose 1} + \dots + {n \choose n} = 2^n \,.
            \end{equation*}
            (Hint: use part (1))
        \item (Optional) Prove the general formula using induction
            \begin{equation*}
                (x+y)^n = \sum_{i=0}^n {n\choose i} x^i y^{n-i} \,.
            \end{equation*}
    Here, we define the sigma notation
    \begin{equation*}
        \sum_{i=0}^n f_i = f_0 + f_1 + \dots + f_n \,.
    \end{equation*}
    \end{enumerate}
\end{problem}




\printbibliography 
%\bibliography{refs}
%\bibliographystyle{halpha-abbrv}


\end{document}
