\documentclass[12pt]{amsart}
\usepackage{../marktext} 
%% Remove draft for real article, put twocolumn for two columns
\usepackage{../svmacro}
\usepackage[utf8]{inputenc}
\usepackage{lineno}
%\usepackage{authblk}
\usepackage[style=alphabetic, backend=biber]{biblatex}
\usepackage{enumitem}
\usepackage[margin=1in]{geometry}

%% commentary bubble
\newcommand{\SV}[2][]{\sidenote[colback=green!10]{\textbf{SV\xspace #1:} #2}}

%% Title 
\title{ MATH 170: Midterm 1 }
%\author[1]{Co-author}
\author{Good luck!}
%\affil[1]{Institute}
\date{}

\begin{document}


\maketitle

There are three questions. Make sure you justify all your work for complete credit.

\section*{Rules}

\begin{itemize}[leftmargin=*]
    \item You have 50 minutes to complete your work and 10 minutes to upload your work.
    \item Open notes (you can use your notes freely).
    \item No use of internet, textbooks, computer algebra systems, calculators. 
    \item No collaboration.
\end{itemize}

\section*{Questions}

\begin{enumerate}[label=\arabic*.,itemsep=10pt, leftmargin=*]
    \item 
        Let $P,Q,R$ be propositional variables
        that represent the following:
        \begin{itemize}
            \item $P=$ ``$24540$ is divisible by $2$''
            \item $Q =$ ``$3$ divides $24540$''
            \item $R=$ ``$5$ divides $11111$''
        \end{itemize}
            
        \begin{enumerate}
        \item
        \textit{[10 points.]}
        Use criteria of divisibility to check if $P$, $Q$, $R$ are correct.
        \item
        \textit{[5 points.]}
        Divide $11111$ by $3$ with remainder.
        \item
        \textit{[5 points.]}
        Find $\gcd(123,9)$.
        \end{enumerate}

    \item 
        There is a restaurant that exclusively delivers instant noodles.
        At the moment, they have 100 customers, all of them are UPenn students.
        The students are happy if they have 12 bowls of instant noodles per week.
        The survival goal of the company is to keep at least one customer happy per week.
        Being a small cafe, they have only three delivery workers who only work 40 hour per week each.
        Assume the productivity of the workers is the same (i.e., they deliver the same number of bowls per hour).
        Assume also that the restaurant doesn't know who they're delivering to.
        How many bowls of instant noodles does each delivery worker have to deliver per hour
        in order to keep the restaurant alive? 
        

        You may follow the provided steps to solve the problem. Any other correct approach will give points too.
        
        \begin{enumerate}
            \item
        \textit{[5 points.]}
            Write out the formulation of the generalized pigeonhole principle.
        
            \item 
        \textit{[5 points.]}
            Identify the pigeons.

            \item 
        \textit{[5 points.]}
            Identify the pigeonholes. 

            \item 
        \textit{[5 points.]}
            Apply the generalized pigeonhole principle.
        \end{enumerate}
    \item 
        \begin{enumerate}
            \item 
        \textit{[10 points.]}
            Translate the following statement into a logical formula.
                
                ``If all humans are heroes all the time then no humans are heroes anytime.''
                
            Suggested variables: $x$ for humans, 
            $t$ for time.
            Suggested predicates: 
            $P(x,t)=$``$x$ is a hero in time $t$'',
            $Q(x,t)=$``$x$ is a hero in time $t$''.
            

            \item 
        \textit{[5 points.]}
            Find the simplest logical formulae 
                to express the negation of the above statements. 
                
            \item
        \textit{[5 points.]}
            Translate the resulting formula in 3b back into English. 

        \end{enumerate}
\end{enumerate}




\end{document}
