\documentclass[12pt]{amsart}
\usepackage{../marktext} 
%% Remove draft for real article, put twocolumn for two columns
\usepackage{../svmacro}
\usepackage[utf8]{inputenc}
\usepackage{lineno}
%\usepackage{authblk}
\usepackage[style=alphabetic, backend=biber]{biblatex}
\usepackage{enumitem}

%% commentary bubble
\newcommand{\SV}[2][]{\sidenote[colback=green!10]{\textbf{SV\xspace #1:} #2}}

%% Title 
\title{ MATH 170: Midterm 1 }
%\author[1]{Co-author}
\author{Good luck!}
%\affil[1]{Institute}
\date{}

\begin{document}


\maketitle

There are three questions. Make sure you justify all your work for complete credit.

\section*{Rules}

\begin{itemize}[leftmargin=*]
    \item You have 50 minutes to complete your work and 10 minutes to upload your work.
    \item Open notes (you can use your notes freely).
    \item No use of internet, textbooks, computer algebra systems, calculators. 
    \item No collaboration.
\end{itemize}

\section*{Questions}

\begin{enumerate}[label=\arabic*.,itemsep=10pt, leftmargin=*]
    \item 
        Consider the following statements:
        \begin{itemize}
            \item $P=$ ``$2$ divides $1234$''
            \item $Q =$ ``$3$ divides $124$''
            \item $R=$ ``$5$ divides $1250$''
        \end{itemize}
        \begin{enumerate}
        \item
        \textit{[10 points.]}
        Use criteria of divisibility to check if $P$, $Q$, $R$ are correct.
        \item
        \textit{[5 points.]}
        Divide $124$ by $3$ with remainder.
        \item
        \textit{[5 points.]}
        Find $\gcd(124,12)$.
        \end{enumerate}
    \item 
        There is a cafe that exclusively delivers bubble tea.
        At the moment, they have 150 customers, all of them are UPenn students.
        The students are happy if they have at least 10 bubble teas a week.
        The survival goal of the company is to keep at least one customer happy per week.
        Being a small cafe, they have only three delivery workers who only work 40 hour per week each.
        Assume the productivity of the workers are the same (i.e., they deliver the same number of bubble teas per hour).
        Assume also that the restaurant doesn't know who they're delivering to.
        How many orders does each delivery worker have to deliver per hour to keep the restaurant alive? 
        
        You may follow the provided steps to solve the problem. Any other correct approach will give points too.
        
        \begin{enumerate}
            \item
        \textit{[5 points.]}
            Write out the formulation of the generalized pigeonhole principle.
        
            \item 
        \textit{[5 points.]}
            Identify the pigeons.

            \item 
        \textit{[5 points.]}
            Identify the pigeonholes. 

            \item 
        \textit{[5 points.]}
            Apply the generalized pigeonhole principle.
        \end{enumerate}


    \item 
        \begin{enumerate}
            \item 
        \textit{[10 points.]}
            Translate the following statement into a logical formula.
                
                ``If all pandas eat bamboo, then there is no panda that is carnivorous.''
                
            Suggested use of variables: $p\in P$ for pandas, $B(p)$ for ``$p$ eats bamboo'', $C(p)$ for  ``$p$ is carnivorous''.

            \item 
        \textit{[5 points.]}
            Find the simplest logical formula
                to express the negation of the above statement. 
            (Recall that negation means that you add the logical operator $\neg$ in front of the whole formula.)

            \item 
        \textit{[5 points.]}
            Translate the resulting formula in 3b back into English. 
        \end{enumerate}
\end{enumerate}




\end{document}
