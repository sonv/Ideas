\documentclass[12pt]{amsart}
\usepackage{../marktext} 
%% Remove draft for real article, put twocolumn for two columns
\usepackage{../svmacro}
\usepackage[utf8]{inputenc}
\usepackage{lineno}
%\usepackage{authblk}
\usepackage[style=alphabetic, backend=biber]{biblatex}
\usepackage{enumitem}

\usepackage[top=1in]{geometry}

%% commentary bubble
\newcommand{\SV}[2][]{\sidenote[colback=green!10]{\textbf{SV\xspace #1:} #2}}

%% Title 
\title{ MATH 170: Midterm 2 \\
\tiny{Solutions for Section 001}}
%\author[1]{Co-author}
\author{Good luck!}
%\affil[1]{Institute}
\date{}

\begin{document}


\maketitle

There are three questions. Make sure you justify all your work for complete credit.

\section*{Rules}

\begin{itemize}[leftmargin=*]
    \item You have 50 minutes to complete your work and 10 minutes to upload your work.
    \item Open notes (you can use your notes freely).
    \item No use of internet, textbooks, computer algebra systems, calculators. 
    \item No collaboration.
\end{itemize}

\section*{Questions}

\begin{enumerate}[label=\arabic*.,itemsep=10pt, leftmargin=*]
    \item 
    Let $A, B$ be sets.
    Recall that the symmetric difference of two sets $A$ and $B$ is 
    \begin{equation*}
        A\triangle B = \set{ x \st x \in A \text{ or } x \in B \text{ but not both}} \,.
    \end{equation*}
    Use the De Morgan's laws and Venn diagram to find out if the following is true:
    \begin{equation*}
        (A\triangle B) \setminus ( A \cup B ) = \varnothing.
    \end{equation*}

\textit{Solution:}
By De Morgan's laws, we see
$$  (A\triangle B) \setminus ( A \cup B ) = 
((A\triangle B) \setminus A) \cap 
((A\triangle B) \setminus B)
.$$
From Venn diagram (draw it yourself!), we see that $(A\triangle B) \setminus A = B \setminus A$ and $(A\triangle B) \setminus B = A \setminus B$ do not intersect, therefore 
$$  (A\triangle B) \setminus ( A \cup B ) =
((A\triangle B) \setminus A) \cap 
((A\triangle B) \setminus B) = 
\varnothing
,$$
and the answer is yes.

    \item 
    \begin{enumerate}
        \item Give an example of a function $f: X\to X$ such that $f\circ f = \mathrm{id}_X$, but $f \neq \mathrm{id}_X$. Hint: you may choose the set $X$ to be finite or some set of numbers.
        
\textit{Solution:}
One example is $f: \mathbb{R} \to \mathbb{R}$, $f(x) = -x$. We see that $f(1) = -1$, so $f \neq \mathrm{id}_{\mathbb{R}}$. Further, $f(f(x)) = f(-x) = -(-x) = x$ for all $x \in \mathbb{R}$, so $f \circ f = \mathrm{id}_X$.

Another example is when $X = \{1,2\}$, and we define $f$ for each value: $f(1) = 2, f(2) = 1$. We can see that this $f$ switches 1 and 2. It is not the identity function, but $f(f(1)) = f(2) = 1$ and $f(f(2)) = f(1) = 2$, therefore $f \circ f = \mathrm{id}_X$.

There may be other examples.

        \item For your example, compute $f \circ f \circ f$.

\textit{Solution:}
It will always be $f \circ f \circ f = f$, regardless of your example.
    \end{enumerate}
    


    \item Recall our definition of the Fibonacci series:
    \begin{itemize}
        \item $F_0 = 0$, $F_1 = 1$,
        \item $F_{n} = F_{n-1} + F_{n-2}$ for $n\geq 2$.
    \end{itemize}
   The  first several entries in the series are:
    \begin{equation*}
    \begin{split}
    &F_0 = 0,\, F_1 = 1,\,
    F_2 = 1+0 = 1,\, F_3 = 1+1 = 2,\,
    F_4 = 3,\,
    F_5 = 5, \\
    &F_6 = 8,\,
    F_7 = 13,\,
    F_8 = 21,\,
    F_9 = 34, \dots
    \end{split}
    \end{equation*}
    Follow these steps to prove by induction that $\forall n \in \mathbb{N}$, the Fibonacci number $F_{3n}$ is even.
    \begin{enumerate}
    \item Determine the statements $P(n)$.
    
\textit{Solution:}
$P(n)$ is the statement ``$2 | F_{3n}$''.
    \item Base step of induction: prove $P(0)$.
    
\textit{Solution:}
$P(0)$ states that 2 divides $F_0 = 0$, which is true.
    \item Express $F_{3n+3}$ as a sum of $F_{3n}$ and some even number.
    
\textit{Solution:}
$$F_{3n+3} = F_{3n+2} + F_{3n+1} =
    2F_{3n+1} + F_{3n}
.$$
    \item Assuming $P(n)$, use the previous part to prove $P(n+1)$.
    
\textit{Solution:}
If $F_{3n}$ is even, then by part (c), $F_{3n+3}$ is a sum of two even numbers, therefore it is also even.
    \item Use the Induction Principle to prove that $2$ divides $F_{3n}$ for all $n$.

\textit{Solution:}
We proved the base $P(0)$ and step of induction $P(n) \Rightarrow P(n+1)$, so by the Induction Principle, the statements $P(n)$ are true for all $n\in \mathbb{N}$, i.e. $F_{3n}$ is even for all natural $n$.
    \end{enumerate}

\end{enumerate}



\end{document}
