\documentclass[12pt]{amsart}
\usepackage{../marktext} 
%% Remove draft for real article, put twocolumn for two columns
\usepackage{../svmacro}
\usepackage[utf8]{inputenc}
\usepackage{lineno}
%\usepackage{authblk}
\usepackage[style=alphabetic, backend=biber]{biblatex}
\usepackage{enumitem}
\usepackage[margin=1in]{geometry}

%% commentary bubble
\newcommand{\SV}[2][]{\sidenote[colback=green!10]{\textbf{SV\xspace #1:} #2}}

%% Title 
\title{ MATH 170: Midterm 2 }
%\author[1]{Co-author}
\author{Good luck!}
%\affil[1]{Institute}
\date{}

\begin{document}


\maketitle

There are three questions. Make sure you justify all your work for complete credit.

\section*{Rules}

\begin{itemize}[leftmargin=*]
    \item You have 50 minutes to complete your work and 10 minutes to upload your work.
    \item Open notes (you can use your notes freely).
    \item No use of internet, textbooks, computer algebra systems, calculators. 
    \item No collaboration.
\end{itemize}

\section*{Questions}

\begin{enumerate}[label=\arabic*.,itemsep=10pt, leftmargin=*]
    \item 
    Let $A, B$ be sets.
    Recall that the symmetric difference of two sets $A$ and $B$ is 
    \begin{equation*}
        A\triangle B = \set{ x \st x \in A \text{ or } x \in B \text{ but not both}} \,.
    \end{equation*}
    Determine whether the following are true or false via the Venn diagram.
    \begin{equation*}
         A = A \cup ((A\cup B) \setminus (A \triangle B)) \,.
    \end{equation*}
    \begin{proof}[Solution]
        True.
    \end{proof}

    \item 
    \begin{enumerate}
        \item Give an example of a function $f$ from $X= \N$ to $Y=\N$ that is injective but not surjective.
        \item Let $Z = \set{ y \in Y \st f(x) = y}$.
        Now consider the function $f$ above but a new codomain $Z$,i.e., $f: \N \to Z$. 
        Is this new function a bijection?
        Explain your reasoning.
    \end{enumerate}
    \begin{proof}[Solution]
        a. 
        There are numerous functions. Two common ones are
        \begin{equation*}
            f(n) =  2n 
        \end{equation*}
        and 
        \begin{equation*}
            f(n) = n^2\,.
        \end{equation*}

        b. It is a bijection. In words, the set $Z$ is the set of natural numbers so that every element in $Z$
        gets mapped to by an element in $X$, i.e, pick  a $y\in Z$, then $y = f(x)$ for some $x\in X$.
        So this function $f: \N \to Z$ is surjective. It was injective by design. So it is bijective.
    \end{proof}

    \item Prove by induction that $\forall n \in \N$,
    \begin{equation*}
       1 + 2^2 + \dots + n^2 = \frac{n(n+1)(2n+1)}{6}\,.
    \end{equation*}
    \begin{proof}[Solution]
       1. Determine $P(n)$, which says
    \begin{equation*}
       1 + 2^2 + \dots + n^2 = \frac{n(n+1)(2n+1)}{6}\,.
    \end{equation*}

    2. Base case: $P(0)$ says 
        \begin{equation*}
            0 = 0 \,,
        \end{equation*}
        which is true.

    3. Suppose $P(0) \wedge \dots \wedge P(k)$ is true.
    In particular, it is true that 
    \begin{equation*}
       1 + 2^2 + \dots + k^2 = \frac{k(k+1)(2k+1)}{6}\,.
    \end{equation*}
    We want to show that $P(k+1)$ is true. In particular, we want to show the following is true
    \begin{equation*}
       1 + 2^2 + \dots + k^2 + (k+1)^2 = \frac{(k+1)(k+2)(2k+3)}{6}\,.
    \end{equation*}
    To do this, consider
    \begin{align*}
       1 + 2^2 + \dots + k^2 + (k+1)^2 & = 
            \frac{k(k+1)(2k+1)}{6} + (k+1)^2 \\
           & =  \frac{k(k+1)(2k+1) + 6 (k+1)^2}{6}  \\
           & =  \frac{(k+1)(k(2k+1) + 6k + 6) }{6}  \\
           & =  \frac{(k+1)(2k^2+ k+ 6k + 6) }{6}  \\
           & =  \frac{(k+1)(2k^2+ 3k+ 4k + 6) }{6}  \\
           &= \frac{(k+1)(k+2)(2k+3)}{6}\,.
    \end{align*}
This shows that $P(k+1)$ is true.

4. By induction theorem, $P(n)$ is true for all $n\in \N$.
    \end{proof}

\end{enumerate}

\end{document}
