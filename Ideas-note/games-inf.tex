%%%
%%%
\chapter{Playing games with infinity}
%%%
%%%



%%%
\section{Games}
%%%

Consider the following two-player game, which we will call Cantor's Game.

Player 1 begins by writing a sequence of $X$'s and $O$'s in the top row of the grid below. Player 2 then writes either an $X$ or an $O$ in the first box on the bottom. Player 1 then writes a sequence of $X$'s and $O$'s in the second row of the grid. Player 2 writes an $X$ or $O$ in the second bottom box. The players continue until all boxes are filled. Player 1 wins if the sequence on the bottom exactly matches any of the sequences Player 1 has written in the grid. Player 2 wins otherwise.

There is a grid below. Play the game a few times, then answer the following question:
 which player has a winning strategy, and why?

\vspace{.1in}
\Huge \noindent
\begin{tabular}{|c|c|c|c|c|}\hline
\ \ \ \ \ \ \ \ \ \ &\ \ \ \ \ \ \ \ \ \ &\ \ \ \ \ \ \ \ \ \ &\ \ \ \ \ \ \ \ \ \ &\ \ \ \ \ \ \ \ \ \ \\ \hline
\ \ \ \ \ \ \ \ \ \ &\ \ \ \ \ \ \ \ \ \ &\ \ \ \ \ \ \ \ \ \ &\ \ \ \ \ \ \ \ \ \ &\ \ \ \ \ \ \ \ \ \ \\ \hline
\ \ \ \ \ \ \ \ \ \ &\ \ \ \ \ \ \ \ \ \ &\ \ \ \ \ \ \ \ \ \ &\ \ \ \ \ \ \ \ \ \ &\ \ \ \ \ \ \ \ \ \ \\ \hline
\ \ \ \ \ \ \ \ \ \ &\ \ \ \ \ \ \ \ \ \ &\ \ \ \ \ \ \ \ \ \ &\ \ \ \ \ \ \ \ \ \ &\ \ \ \ \ \ \ \ \ \ \\ \hline
\ \ \ \ \ \ \ \ \ \ &\ \ \ \ \ \ \ \ \ \ &\ \ \ \ \ \ \ \ \ \ &\ \ \ \ \ \ \ \ \ \ &\ \ \ \ \ \ \ \ \ \ \\ \hline
\end{tabular}
\vspace{.25in}

\noindent
\begin{tabular}{|c|c|c|c|c|}\hline
\ \ \ \ \ \ \ \ \ \ &\ \ \ \ \ \ \ \ \ \ &\ \ \ \ \ \ \ \ \ \ &\ \ \ \ \ \ \ \ \ \ &\ \ \ \ \ \ \ \ \ \ \\ \hline
\end{tabular}
\normalsize



%%%
\section{Infinity}
%%%

\begin{definition}[Countable sets]
    A set $X$ is \emph{countably infinite} if there exists a bijection $f: \N \to X$.
    The bijection $f$ is called an \emph{enumeration} of $X$.
    We say $X$ is \emph{countable} if it is finite or countably infinite.
    If there is no such bijection and the set is not finite, then $X$ is said to be
    \emph{uncountably infinite}.
\end{definition}

\begin{example}
   The set of even natural numbers, denoted by $E$, is countably infinite. 
   To see this, we consider the function $f:\N \to E$.
   \begin{equation*}
       f(n) = 2n \,.
   \end{equation*}
\end{example}

A more interesting example is the following.

\begin{example}
    The set of non-negative rational numbers is countably infinite.
    We see this by the snake argument.
\end{example}

\begin{theorem}
   $\R$ is uncountably infinite. 
\end{theorem}


