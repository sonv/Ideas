\documentclass[12pt]{amsart}
\usepackage{../marktext} 
%% Remove draft for real article, put twocolumn for two columns
\usepackage{../svmacro}
\usepackage[utf8]{inputenc}
\usepackage{lineno}
\usepackage[style=alphabetic, backend=biber]{biblatex}
\addbibresource{bibliography.bib}
\usepackage{enumitem}

%% commentary bubble
\newcommand{\SV}[2][]{\sidenote[colback=green!10]{\textbf{SV\xspace #1:} #2}}

%% Title 
\title{ MATH 170: Homework 1 Solutions }
%\author[1]{Co-author}
\author{Teaching team}
%\affil[1]{Institute}
\date{\today}

\begin{document}

\maketitle

\begin{enumerate}[label=\arabic*.,itemsep=10pt, leftmargin=*]
    \item  $15$ children participate in an Easter Egg hunt. If they bring back a total of $100$ eggs, must two of them bring back the same number of eggs?

\begin{proof}[Solution]
   There are various ways to do this problem. 
   
   \emph{Solution 1.}
   Suppose all of the children bring back different
   number of eggs.
   Say, some bring back, 1, some bring back 10, etc.

   Let us be pessimistic about the children's abilities to find eggs so they will
   find the minimum number of eggs possible.
   So, the possible numbers of eggs would be
   \begin{equation*}
       0, 1, 2, \dots, 14\,.
   \end{equation*}
   Already, one can see that this is impossible because
   \begin{equation*}
       1+2+\dots+14 = 105\,,
   \end{equation*}
   which is bigger than $100$.
   So there must be two children who have the same number of eggs.

   \emph{Solution 2.} If you insist to use pigeonhole principle, here is a way.
   First, list the children according to the number of eggs they find, from 
   littlest to largest.
   Let $n_i$ be the number of eggs that the $i^{th}$ child finds.
   We have the following list
   \begin{equation*}
       \set {n_1, n_2, \dots, n_{15} } \,.
   \end{equation*}
   where $0 \leq n_1 \leq n_2 \leq \dots \leq n_{15}$.
   We note that it CANNOT be the case that $n_i \geq i$ for all $i=1, \dots, 15$
   just because $1+2+\dots+ 15 = 120 > 100$.
   So, there must be a number $i > 1$ where $n_i < i$.
   So, the number of children up to that point is $i$ and the number of maximum eggs
   that children $\set{1,2,...,i}$ must have is $n_i < i$.
   Now, by pigeonhole principle (pigeons are children, pigeonholes are numbers of eggs),
   there must be two children who have the same number of eggs.

   Note: this solution is a bit more satisfying as we don't have to be pessimistic about
   the abilities of the children to find eggs.
\end{proof}
    
    \item  Suppose we pick $n+1$ numbers from the set $\set{1, 2,\dots, 2n}$ where $n\geq 1$. 
Let $j \geq 1$ be a divisor of $n$.
Is it true or false that there are always two numbers have a difference of $j$ 
apart from each other?
Why?
\begin{proof}[Solution]
    True.
    
    Step 0. Simple case $j = 1$.
    Let the numbers be the pigeons. 
    
    Let the pigeonholes be as follows.
    \begin{equation*}
        \set{1,2}, \set{3,4}, \dots,
        \set{2n-1, 2n}\,.
    \end{equation*}
    Assigning rule: the pigeons are assigned to the holes that contain the corresponding label.
    
    There are $n+1$ pigeons assigning to $n$ holes so by pigeonhole principle, 
    there must be one hole with  2 pigeons.
    By definition of the holes, the two numbers in that hole needs to have the difference of $1$.
    
    Step 1.
    From now on $j\geq 2$.
    By definition of divisor, there exists $q$ such that 
    \begin{equation*}
        n = qj\,.
    \end{equation*}
    Let the pigeons be the numbers that got picked ($n+1$ of them).

    Let the pigeonholes be the set of numbers that have the same remainder when
    it gets divided by $j$.
    So, for example, all the numbers that looks like $kj + r$ with the same $r$ but different
    $k$ would belong to the same pigeonhole.
    Listing them all out
    \begin{gather*}
        S_0 = \set{ kj + 0 \st k\in \N }\,,\\
        S_1 = \set{ kj + 1 \st k \in \N }\,,\\
        \dots\\
        S_{j-1} = \set{ kj + j-1\st k \in \N }\,,
    \end{gather*}
    There are $j$ pigeonholes.
    
    Assigning rule: put the number to the set that contain the corresponding label.

    Because $n = qj$, in each of the sets above, $k \leq 2q \leq n$ (since $ j \geq 2$ and $n = jq$).
    So, as we picked $n+1$ numbers, using the generalized pigeonhole principle, 
    one of the pigeonholes, say $S_r$, above must have $q+1$ numbers (from the selected numbers).

    Step 2.
    Now, in $S_r$, there are $q+1$ members, which we will make to be pigeons.

    Recall each number in $S_r$ looks like
    \begin{equation*}
        kj + r\,.
    \end{equation*}
    
    In the set $S_1$ the possible $k$'s are
    \begin{equation*}
        k = 1,\dots, 2q \,.
    \end{equation*}
    
    For $S_r$ where $r=2,\dots, j-1$
    the possible $k$'s are
    \begin{equation*}
        k = 0,1, \dots, 2q-1 \,.
    \end{equation*}

    Now, we make the possible $k$'s to be the pigeons (remember, we only have $q+1$ of them but don't know which ones).

    The pigeonholes will be
    \begin{equation*}
        \set{1,2}, \set{3,4}, \dots, \set{2q - 1, 2q} 
    \end{equation*}
    if $r =1$,
    or 
    \begin{equation*}
        \set{0,1}, \set{2,3}, \dots, \set{2q - 2, 2q-1} 
    \end{equation*}
    if $r \geq 2$.
    
    In each case, there are $q$ of the pigeonholes and $q+1$ pigeons.
    By the pigeonhole principle once again, there exists a hole with two numbers that are consecutive to each other. 
    But these are the posibilities for $k$'s.
    We want to translate back to the original numbers in the set $S_r$, which will look like
    $mj + r$ and $(m+1)j + r$.
    These two numbers have the difference of $j$.
\end{proof}

\item
 Show that if you select $6$ numbers from the set $\{1,2,\dots,10\}$, one must be a multiple of another.

\begin{proof}[Solution]
    Let the numbers be the pigeons.

    The pigeonholes are designed as follows.
    \begin{equation*}
        \set{1,2,6}, \set{3,9}, \set{4,8}, \set{5,10}, \set{7}\,.
    \end{equation*}

    Assigning rule: put the number to the set that contain the corresponding label.

   We have then picked $6$ pigeons for $5$ pigeonholes.


   By the pigeonhole principle, one hole must have 2 members. That hole cannot be
   $\set{7}$ as it only has 1 member.
   The other holes contain the property that given two numbers belonging to the set,
    one must be a multiple of the other.
\end{proof}

 \item 
 The Earth has more than 7 billion people and almost no one lives 100 years. Suppose this longevity fact remains true. How do you know that some year soon, more than 50 million people will die? 

\begin{proof}[Solution]
   Let people be pigeons. 

   Let each of the number of years from 1 to 100 be the pigeonholes. 
    So each hole is labeled according to the number of years lived.

    Currently, there are more than 7 billion people on the earth. If most people
    die within 100 years, in 2121, most of the 7 billion people who are currently alive
    will die.

    By the pigeonhole principle, there must be one year in the next 100 years when
    more than 50 million people will die in the same year.
\end{proof}


 \item  You have 10 pairs of socks, five black and five rainbow, 
but they are all mixed up in a drawer. 
It's 5 a.m. in the morning, and you don't want to turn on the lights in your dark room. 
How many socks must you pull out to guarantee that you have a pair of one color? How many must you pull out to have two good pairs (one black pair and one rainbow pair is okay)?

\begin{proof}[Solution]
    I will only give the answer, you have to give the justification.

    1. 3 socks must be pulled out to have a pair of one color.

    2. 5 socks must be pulled  out to have two good pairs.
\end{proof}



\end{enumerate}




%\printbibliography 
%\bibliography{refs}
%\bibliographystyle{halpha-abbrv}


\end{document}
