\documentclass[12pt]{amsart}
\usepackage{../marktext} 
%% Remove draft for real article, put twocolumn for two columns
\usepackage{../svmacro}
\usepackage[utf8]{inputenc}
\usepackage{lineno}
%\usepackage{authblk}
\usepackage[style=alphabetic, backend=biber]{biblatex}
\usepackage{enumitem}

%% commentary bubble
\newcommand{\SV}[2][]{\sidenote[colback=green!10]{\textbf{SV\xspace #1:} #2}}

%% Title 
\title{ MATH 170: Homework 10 }
%\author[1]{Co-author}
\author{Due: Dec 10, 2021}
%\affil[1]{Institute}
\date{}

\begin{document}

\maketitle


\noindent{\bf Graded for accuracy:}
1, 2.
\\
\noindent{\bf Graded for completion:} 3, 4.

\noindent{\bf Instructions:}
Problems that are graded for accuracy must be correct to get points.
Problems that are graded for completion must show some trying effort.

\centerline

\hrule

\centerline

\begin{enumerate}[label=\arabic*.,itemsep=10pt, leftmargin=*]
    \item Two students A and B are both registered for a certain course. Assume that student A attends class 80 percent of the time, student B attends class 60 percent of the time, and the absences of the two students are independent.
        \begin{enumerate}
            \item  What is the probability that at least one of the two students will be in class on a given day?
                \item  If at least one of the two students is in class on a given day, 
                    what is the probability that A is in class that day?
        \end{enumerate}
      
        
\item Suppose that a box contains five coins, labeled 1, 2, 3, 4, 5, and that for each coin $i$ there is a different probability $p_i$ that a head will be obtained when the coin is tossed. The probabilities are $p_1 = 0$, 
$p_2 = 1/4$, $p_3=1/2$, $p_4=3/4$,
$p_5 = 1$.
Suppose that one coin is selected at random from the box and when it is tossed once, a head is obtained. What is the posterior probability that the $i$th coin was selected (for each $i = 1, \dots , 5$)?

    \item You will calculate your own grades.
       The formula for expected value for a probability space with $n$ outcomes
       $(\set{x_1, \dots, x_n}, \P)$ is
        \begin{equation*}
            E = x_1 \P(\set{x_1}) + \dots x_n \P(\set{x_n}).
        \end{equation*}
        Your average is an expected value with the probability space
        $$\paren[\Big]{\set[\big]{\frac{\text{ your homework score}}{\text{possible total homework score}}, 
                \frac{\text{Your total exam score}}{\text{possible total exam score }}}, \P} \,.$$
                The function $\P$ is defined as 
                \begin{equation*}
                    \P\paren[\Big]{ \set[\Big]{\frac{\text{ your homework score}}{\text{possible total homework score}} } }= 0.7 \,,
                \end{equation*}
                and 
                \begin{equation*}
                   \P \paren[\Big]{\set[\Big]{  \frac{\text{Your total exam score}}{\text{possible total exam score }} }}  = 0.3 \,.
                \end{equation*}
                Assuming you get zero in this homework.
                Take into account all the dropped homeworks and exams (according to the syllabus). 
                What is the least score for you in the final, up to this point of the class, to pass this course? To get an A? (Don't think about $+-$ for this problem.)
                
                
                
                
\item 
In this problem, you will curve your grade. Remember that it is NOT the grade that you will get for the course! Feel free to use calculators for this problem.

In all three sections of Math 170, there are 133 students in total. The average current score is $a = 87.26$. Standard deviation is $s = 14.75$. In a normal distribution, $68\%$ of the population lies within one standard deviation of the mean (average), and $95\%$ lies within two standard deviations. The distribution of the scores is not normal, but we will still apply this model for curving.

\begin{enumerate}
    \item Calculate how many standard deviations from the mean your score is. That is, find the absolute value of $x$, where \textit{your score} $= a + xs$.
    \item The curved grade is decided by the following rule:
    \begin{itemize}
        \item top $3\%$ get A$+$;
        \item next $13\%$ get A or A$-$;
        \item next $34\%$ get B$\pm$;
        \item next $34\%$ get C$\pm$;
        \item next $13\%$ get D$\pm$;
        \item lowest $3\%$ fail.
    \end{itemize}
    Determine what your grade on the curve is. The distribution of the grades is given in  Table 1: left column says how many students are in the ``bin'', and second column says which scores these students have.
\begin{table}[]
\begin{tabular}{|l|l|}
\hline
4  & 99.51-100   \\ \hline
18 & 97.66-99.50 \\ \hline
15 & 96.02-97.65 \\ \hline
15 & 94.77-96.01 \\ \hline
15 & 92.71-94.76 \\ \hline
15 & 90.11-91.70 \\ \hline
15 & 87.64-90.10 \\ \hline
15 & 81.63-87.63 \\ \hline
18 & 49.81-81.62 \\ \hline
3  & 0-49.80     \\ \hline
\end{tabular}
\caption{Distribution of grades.}
\end{table}   
\item Compare your curved grade to the grade with the course's cutoffs. What do you think are advantages and drawbacks of curving?

\end{enumerate}


\end{enumerate}



\end{document}
