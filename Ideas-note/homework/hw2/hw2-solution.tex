\documentclass[12pt]{amsart}
\usepackage{../marktext} 
%% Remove draft for real article, put twocolumn for two columns
\usepackage{../svmacro}
\usepackage[utf8]{inputenc}
\usepackage{lineno}
\usepackage[style=alphabetic, backend=biber]{biblatex}
\addbibresource{bibliography.bib}
\usepackage{enumitem}

%% commentary bubble
\newcommand{\SV}[2][]{\sidenote[colback=green!10]{\textbf{SV\xspace #1:} #2}}

%% Title 
\title{ MATH 170: Homework 1 Solutions }
%\author[1]{Co-author}
\author{Teaching team}
%\affil[1]{Institute}
\date{\today}

\begin{document}

\maketitle

\begin{enumerate}[label=\arabic*.,itemsep=10pt, leftmargin=*]
\item  Are these numbers divisible by 2, 3, 5? To get points, use the criteria of divisibility that we covered in class. Optional: check your answers with a calculator.
    \begin{enumerate}[label=\alph*.,itemsep=5pt, leftmargin=*]
    \item 12345;
    \item 10101010.
    \end{enumerate}
    
    \begin{proof}[Solution]
       a. Divisible by 3 and 5 as the sums of the digits is 15 which is divisible by 3 and the last digit is 5.

       b. divisible by 2 and 5 as the last digit is 0.
    \end{proof}

\item  
    \begin{enumerate}[label=\alph*.,itemsep=5pt, leftmargin=*]
    \item Let $a, b \in \mathbb Z$. Prove that $\gcd(a,b) = \gcd(a,-b)$.
    \item Compute $\gcd(12,-30)$ using Euclid's algorithm.
    \item Compute $\gcd(12\,345,11\,100)$ using Euclid's algorithm.
    \item Optional: check your answers on \url{https://www.wolframalpha.com/}.
    \\ \includegraphics[width=0.9\textwidth]{wolframalpha.png}
    \end{enumerate}

    \begin{proof}[Solution]
        a. We have that $\gcd ( a,b ) \st b$. So there exists $q \in \Z$ such that
        \begin{equation*}
            b = q \gcd(a,b) \,.
        \end{equation*}
        This means,
        \begin{equation*}
            -b = (-q) \gcd(a,b) \,.
        \end{equation*}
        So $\gcd(a,b) \st -b$ and, therefore, $\gcd(a,b) \leq \gcd(a,-b)$.
        Argue similarly, $\gcd(a,-b) \leq \gcd(a,b)$.
        Thus, the equality holds.

        b. By part (a), $\gcd(12, -30) = \gcd(12, 30)$. Therefore, we can apply 
        Euclid's algorithm
        \begin{gather*}
            30 = 2 \cdot 12 + 6 \\
            12 = 2\cdot 6 + 0 \,.
        \end{gather*}
        So $\gcd(12,-30) = 6$.

        c. 
        \begin{gather*}
            12345 = 1\cdot 11100 + 1245 \\
            11100 = 8\cdot 1245 + 1140 \\
            1245 = 1\cdot 1140 + 105 \\
            1140 = 10 \cdot 105 + 90 \\
            105 = 1\cdot 90 + 15 \\
            90 = 60\cdot 15 + 0 \,.
        \end{gather*}
        Thus, $\gcd(12345, 11100) = 15 $.
    \end{proof}

\item
    \begin{enumerate}[label=\alph*.,itemsep=5pt, leftmargin=*]
    \item Divide the following numbers by 10 with remainder: 1234, 2021. That is, write them in the form $10\cdot q + r$, where $0 \leq r < 10$. Compute $1234 + 2021$ and divide the result by 10 with remainder. Observe that the remainder you get is the sum of the first two remainders.
    \item Compute the remainder of $1236$ and $2027$ when dividing by 10, then compute the remainder of their sum. Why is it no longer just the sum of remainders?
    \item Rephrase the following rule in your own words: If you divide $a_1, a_2 \in \mathbb Z$ by $b>0$ with remainder, write the remainder of $a_1$ as $r_1$ and the remainder of $a_2$ as $r_2$. Then the remainder of $a_1 + a_2$ is $r_1 + r_2$ or $r_1 + r_2 - b$.
    
    \item[*]
    Bonus: prove c.
    \end{enumerate}
    \begin{proof}[Solution]
        a.
        \begin{gather*}
           1234 = 10\cdot 123 + 4  \,, \\
           2021 = 10 \cdot 202 + 1  \,, \\
           1234 + 2021 = 3255 = 10 \cdot 325 + 5 \,.
        \end{gather*}

        b. 
        \begin{gather*}
           1236 = 10\cdot 123 + 6  \,, \\
           2027 = 10 \cdot 202 + 7  \,, \\
           1236 + 2027 = 3263 = 10 \cdot 326 + 3 \,.
        \end{gather*}

        c. I will give a proof of this problem, not rephras it.

        Let's rewriting $a_1$ and $a_2$ so we have a visual effect of what is 
        going on.
        \begin{gather*}
            a_1 =  q_1 b + r_1 \\
            a_2 = q_2 b + r_2 \,,
        \end{gather*}
        for some $q_1, q_2 \in \Z$.
        We know that because $b>0$ and by the definition of remainders,  
        \begin{equation*}
            0\leq r_1, r_2 < b \,.
        \end{equation*}
        Note that there is no restriction for $r_1 + r_2$ as they are just another
        natural number. So, there are 2 cases
        \begin{enumerate}
            \item When $r_1 + r_2 <b$. Then,
                \begin{equation*}
                    a_1 + a_2 = q_1 b + q_2 b + r_1 + r_2
                    = (q_1 + q_2) b + (r_1 + r_2) \,.
                \end{equation*}
                This is the unique representation of $a_1 + a_2$ with its remainder 
                (by the division theorem (look it up again)) as $0\leq r_1 + r_2 < b$.
                In other words, we have found the unique $q$ 
                and $r$ in the division theorem
                so that $a_1 + a_2$ can be written in terms of $b$.
                So, the remainder of $a_1 + a_2$ divided by $b$ in this case
                is $r_1 + r_2$.

            \item When $r_1 + r_2 >b$. Note that $r_1 + r_2 < 2b$ as each of them
                is less than $b$. Therefore, $r_1 + r_2 - b < b$.
                We want to find the unique $q$ and $r$ in the division theorem in 
                this case as well. To do that, we write
                \begin{align*}
                    a_1 + a_2 &= q_1 b + q_2 b + r_1 + r_2 \\
                              &= (q_1 + q_2) b + (r_1 + r_2 - b) + b  \\
                              &= (q_1 + q_2 + 1) b + (r_1 + r_2 - b)  
                    \,.
                \end{align*}
                As $r_1 + r_2 - b <b$, we just show that the remainder of 
                $a_1+ a_2$ divided by $b$ in this case is $r_1 + r_2 - b$.
        \end{enumerate}
    \end{proof}




\end{enumerate}




%\printbibliography 
%\bibliography{refs}
%\bibliographystyle{halpha-abbrv}


\end{document}
