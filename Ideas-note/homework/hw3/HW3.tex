\documentclass[12pt]{amsart}
\usepackage{../marktext} 
%% Remove draft for real article, put twocolumn for two columns
\usepackage{../svmacro}
\usepackage[utf8]{inputenc}
\usepackage{lineno}
%\usepackage{authblk}
\usepackage[style=alphabetic, backend=biber]{biblatex}
\usepackage{enumitem}

%% commentary bubble
\newcommand{\SV}[2][]{\sidenote[colback=green!10]{\textbf{SV\xspace #1:} #2}}

%% Title 
\title{ MATH 170: Homework 3 }
%\author[1]{Co-author}
\author{Due: September 27, 2021}
%\affil[1]{Institute}
\date{}

\begin{document}

\maketitle


\noindent{\bf Graded for accuracy:}
2, 3.
\\
\noindent{\bf Graded for completion:}
1.
\\
\noindent{\bf Instructions:}
Problems that are graded for accuracy must be correct to get points.
Problems that are graded for completion must show some trying effort.

\centerline

\hrule

\centerline

\begin{enumerate}[label=\arabic*.,itemsep=10pt, leftmargin=*]
\item  A practice in inductive logic.
    \begin{enumerate}[label=\alph*.,itemsep=5pt, leftmargin=*]
    \item Give an argument that supports the statement ``Having a pet is good''.
    \item Give an argument that supports the statement
    ``Having a pet is bad''.
    \item Which of your arguments do you find stronger than another?
    \item Is the statement from part (a) a \emph{mathematical statement}? Why / why not?
    \end{enumerate}

\item  For each of the following plain English statements, translate
    it into a symbolic logic propositional formula.
    Propositional variables in your formulae should represent
    the simplest propositions that they can.

    (Note that each statement doesn't need to be true. Your job is 
    just to translate to symbolic logic.)

    \begin{enumerate}[label=\alph*.,itemsep=5pt, leftmargin=*]
    \item Guinea pigs are quiet, but they are loud when they are hungry.
    \item $\sqrt{2}$ can't be an integer because it is a rational number.
    \end{enumerate}


\item  
    For fixed $n\in \N$, let $P$ represent the proposition 
    `$n$ is even', $Q$ represent the proposition `$n$ is prime' and
    $R$ represent the proposition `$n = 2$'. 
    For each of the following propositional formulae, translate
    it into plain English and determine whether it is true for all 
    $n\in \N$, true for some values of $n$ and false for some values 
    of $n$, or false for all $n\in \N$.
    \begin{enumerate}[label=\alph*.,itemsep=5pt, leftmargin=*]
        \item $(P \wedge Q) \implies R$
        \item $ Q \wedge (\neg R) \implies (\neg P)$
        \item $ ((\neg P) \vee (\neg Q)) \vee (\neg R)$
        \item $ (Q \wedge P) \vee (\neg R)$
    \end{enumerate}




\end{enumerate}



\end{document}
