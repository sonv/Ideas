\documentclass[12pt]{amsart}
\usepackage{../marktext} 
%% Remove draft for real article, put twocolumn for two columns
\usepackage{../svmacro}
\usepackage[utf8]{inputenc}
\usepackage{lineno}
%\usepackage{authblk}
\usepackage[style=alphabetic, backend=biber]{biblatex}
\usepackage{enumitem}

%% commentary bubble
\newcommand{\SV}[2][]{\sidenote[colback=green!10]{\textbf{SV\xspace #1:} #2}}

%% Title 
\title{ MATH 170: Homework 3 }
%\author[1]{Co-author}
\author{Due: September 27, 2021}
%\affil[1]{Institute}
\date{}

\begin{document}

\maketitle


\noindent{\bf Graded for accuracy:}
2, 3.
\\
\noindent{\bf Graded for completion:}
1.
\\
\noindent{\bf Instructions:}
Problems that are graded for accuracy must be correct to get points.
Problems that are graded for completion must show some trying effort.

\centerline

\hrule

\centerline

\begin{enumerate}[label=\arabic*.,itemsep=10pt, leftmargin=*]
\item  A practice in inductive logic.
    \begin{enumerate}[label=\alph*.,itemsep=5pt, leftmargin=*]
    \item Give an argument that supports the statement ``Having a pet is good''.
    \item Give an argument that supports the statement
    ``Having a pet is bad''.
    \item Which of your arguments do you find stronger than another?
    \item Is the statement from part (a) a \emph{mathematical statement}? Why / why not?
    \end{enumerate}

    \begin{proof}[One possible answer]
    \begin{enumerate}
        \item My mother has a pet (me) and she is happy. 
            My grandfather has a pet (my father) and he is happy. 
            My sister has a cat and she is happy. 
            Therefore, having a pet is good.

        \item My father is always stressed about his pet (me).
            My niece was bitten by a dog.
            My friend Jake has to clean up for his cat everyweek and it seems torturing.
            Therefore, having a pet is bad.

        \item (b) seems stronger to me.
        \item The statement in part (a) is not a mathematical statement because
            it depends on the background theory of each person.
            There is no abstract principle that one can go to deduce the truth value
            of this statement.
    \end{enumerate}
        
    \end{proof}

\item  For each of the following plain English statements, translate
    it into a symbolic logic propositional formula.
    Propositional variables in your formulae should represent
    the simplest propositions that they can.

    (Note that each statement doesn't need to be true. Your job is 
    just to translate to symbolic logic.)

    \begin{enumerate}[label=\alph*.,itemsep=5pt, leftmargin=*]
    \item Guinea pigs are quiet, but they are loud when they are hungry.
    \item $\sqrt{2}$ can't be an integer because it is a rational number.
    \end{enumerate}
    \begin{proof}[Answer]
       \begin{enumerate}
           \item Let $P = $ ``Guinea pigs are quiet'', 
               and $R =$ ``guinea pigs are hungry''.
               Then the propositional formula is
               \begin{equation*}
                   P \vee (R \implies \neg P) \,.
               \end{equation*}
            
            \item Let $P = $ ``$\sqrt{2}$ is an integer'', 
                $Q = $ ``$\sqrt{2}$ is a rational number''.
                Then the propositional formular is 
                \begin{equation*}
                    Q \implies \neg P \,.
                \end{equation*}
       \end{enumerate} 
    \end{proof}


\item  
    For fixed $n\in \N$, let $P$ represent the proposition 
    `$n$ is even', $Q$ represent the proposition `$n$ is prime' and
    $R$ represent the proposition `$n = 2$'. 
    For each of the following propositional formulae, translate
    it into plain English and determine whether it is true for all 
    $n\in \N$, true for some values of $n$ and false for some values 
    of $n$, or false for all $n\in \N$.
    \begin{enumerate}[label=\alph*.,itemsep=5pt, leftmargin=*]
        \item $(P \wedge Q) \implies R$
        \item $ Q \wedge (\neg R) \implies (\neg P)$
        \item $ ((\neg P) \vee (\neg Q)) \vee (\neg R)$
        \item $ (Q \wedge P) \vee (\neg R)$
    \end{enumerate}
    \begin{proof}[Answer]
        \begin{enumerate}
            \item This is true for all $n$.
                To see this, we need to plug in all $n$ into the following statement
                to see if it is true.
                
                ``If $n$ is even and $n$ is prime then $n = 2$.''

                There are two cases to consider:

                Case 1: $n=2$. In this case, $P$ is true and $Q$ is true. Therefore
                $P \wedge Q$ is true. Of course, $2=2$, so $R$ is true.
                As the premise and conclusion are both true, this statement is true.

                Case 2: $n\not= 2$. There are two sub-cases here. 
                If $n$ is even, then $n$ is not a prime, so $Q$ is false, 
                which makes
                $P \wedge Q$ false. Thus, the premise is false, making the whole statement
                true.

                If $n$ is odd, then $P$ is false, 
                which makes
                $P \wedge Q$ false. Thus, the premise is false, making the whole statement
                true.

            \item English translation:

                ``If $n$ is prime and $n$ is not $2$ then $n$ is odd.''

                This statement is true for all $n$.
                Again, there are two cases.

                Case 1: $n=2$. This makes $\neg R$ false and therefore
                $Q \wedge (\neg R)$ is false, making the premise false.
                Therefore, the whole statement is true.

                Case 2: $n\not= 2$. There are two sub-cases as well. 
                If $n$ is prime (thus $Q$ is true), 
                then $n$ is odd (thus $\neg P$ is true).
                Both premise and conclusion are true, which means the statement is true.

                If $n$ is not prime then $Q$ is false, making the premise false.
                Therefore, the whole statement is true.

            \item English translation:

                ``$n$ is odd or $n$ is not a prime or $n$ is not $2$.''

                This statement is false for $n=2$ but true for $n=25$.
                Therefore, it is true for some $n$ and true for some $n$.

            \item English translation:

                ``$n$ is an even prime or $n$ is not $2$.''

                This statement is true for all $n$.

                Again, there are two cases.

                Case 1: $n = 2$. This makes $P$ and  $Q$ both  true; therefore,
                $P \wedge Q$ is true by the rule of conjunction.
                Therefore, by the rule of disjunction, the statement is true.

                Case 2: $n \not = 2$. This makes $\neg R$ true. By the rule
                of disjunction, the whole statement is true.
                
        \end{enumerate}
        
    \end{proof}




\end{enumerate}



\end{document}
