\documentclass[12pt]{amsart}
\usepackage{../marktext} 
%% Remove draft for real article, put twocolumn for two columns
\usepackage{../svmacro}
\usepackage[utf8]{inputenc}
\usepackage{lineno}
%\usepackage{authblk}
\usepackage[style=alphabetic, backend=biber]{biblatex}
\usepackage{enumitem}

%% commentary bubble
\newcommand{\SV}[2][]{\sidenote[colback=green!10]{\textbf{SV\xspace #1:} #2}}

%% Title 
\title{ MATH 170: Homework 4 solution }
%\author[1]{Co-author}
\author{Teaching team}
%\affil[1]{Institute}
\date{}

\begin{document}

\maketitle


\noindent{\bf Graded for accuracy:}
1, 2.
\\
\noindent{\bf Graded for completion:}
3, 4, 5.
\\
\noindent{\bf Not graded (practice problems):}
6, 7, 8.
\\
\noindent{\bf Instructions:}
Problems that are graded for accuracy must be correct to get points.
Problems that are graded for completion must show some trying effort.

\centerline

\hrule

\centerline

\begin{enumerate}[label=\arabic*.,itemsep=10pt, leftmargin=*]

\item Recall that the operator ``IF AND ONLY IF'' (``biconditional operator''), denoted $\Leftrightarrow$, is defined as follows:
$$ P \Leftrightarrow Q
\mbox{ by definition means the formula }
(P\Rightarrow Q) \wedge (Q \Rightarrow P).
$$
Using this definition, write the truth table for this operator.

\begin{proof}[Answer]
    Therefore we will have the following truth table.

   \begin{center}
    \begin{tabular}{ |c|c|c|c|c|c|c|c| } 
    \hline
    $P$ & $Q$  &    $ P \implies Q$ & $Q \implies P$ & $ (P \implies Q) \wedge ( Q \implies P) $ \\
    \hline
    T & T & T & T &T \\
    F & T & T &F & F \\
    T & F & F &T & F \\
    F & F & T &T & T \\
    \hline
    \end{tabular}
\end{center} 
\end{proof}

\item  Using logical rules or truth tables, prove logical equivalences:
    \begin{enumerate}[label=\alph*.,itemsep=5pt, leftmargin=*]
    \item $(P \vee Q) \Rightarrow R \equiv (P \Rightarrow R) \wedge (Q \Rightarrow R)$.
    \item $P \Leftrightarrow Q \equiv
    (P \Rightarrow Q) \wedge ((\neg P) \Rightarrow (\neg Q))$ 
    (use Problem 1 to recover the truth table for $\Leftrightarrow$). 
\end{enumerate}

\begin{proof}[Answer]
    a. 
   \begin{center}
        \begin{tabular}{ |c|c|c|c|c|c|c|c| } 
        \hline
        $P$ & $Q$  &  $R$ & $P \vee Q$  & $(P \vee Q) \Rightarrow R$   & $P \Rightarrow R$ & $Q \Rightarrow R$ & $(P \Rightarrow R) \wedge (Q \Rightarrow R)$\\
        \hline
        T & T & T & T & T & T & T & T \\
        T & T & F & T & F & F & F & F \\
        T & F & T & T & T & T & T & T \\
        T & F & F & T & F & F & T & F \\
        F & T & T & T & T & T & T & T \\
        F & T & F & F & F & T & F & F \\
        F & F & T & T & T & T & T & T \\
        F & F & F & F & T & T & T & T \\
        \hline
        \end{tabular}
    \end{center} 

    b. We don't need the truth table for this but you can use it if you insist. 
    Note that $Q \Rightarrow P \equiv (\neg P) \Rightarrow (\neg Q)$.
    From problem 1, we get 
     $P \Leftrightarrow Q \equiv
    (P \Rightarrow Q) \wedge ((\neg P) \Rightarrow (\neg Q))$ 
\end{proof}

\item  In plain English, explain the equivalence in 2a.
    \begin{proof}[Answer]
        If $P$ or $Q$ implies $R$ then $P$ implies $R$ and $Q$ implies $R$ and vice versa.
    \end{proof}

\item Let $P$ be the set of all people and
     $E$ be the set of all even numbers.
    Translate the following sentences into logical formulae.
    \begin{enumerate}
        \item Everyone loves someone.
        \item Everyone is short or tall.
        \item Everyone is short or everyone is tall.
        \item Every even number is a multiple of 2 with some integer.
    \end{enumerate}

    \begin{proof}[Answer]
        a. Let $P(x,y)$ be ``$x$ loves $y$'' and $S$ be teh set of all people. 
        Then, $\forall x \in S, \exists y \in S, P(x,y)$.

        b. Let $P(x)$ be ``$x$ is tall and $Q(x)$ be $x$ is short. 
        Then, $\forall x\in S, P(x)\vee Q(x)$.

        c. $(\forall x \in S, P(x) ) \vee (\forall x\in S, Q(x))$.

        d. Let $S$ be the set of all even numbers.
        Let $P(x,n)$ be   ``$x =  2n$''.
        $\forall x\in S, \exists n\in \Z, P(x,n)$.
    \end{proof}



\item  
    \begin{enumerate}[label=\alph*.,itemsep=5pt, leftmargin=*]
    \item
    Translate into purely logical formula the following (quite convoluted) plain English sentence:
    \\
    ``It is not true that for every cat you cannot find a dog such that under no circumstances them sharing a toy means that they go to walks together.''
    \\
    Suggested use of variables: $c$ for a cat, $d$ for a dog, $T(c,d)$ for ``$c$ and $d$ share a toy'', $W(c,d)$ for ``$c$ and $d$ go to walks together''.
    \item 
    Simplify the formula from 5a by moving the negations past the quantifiers so that each negation in the formula only appears right in front of $T(c,d)$ or $W(c,d)$. I.e. write a maximally negated logical formula that is logically equivalent to the formula from 5a.
    \item
    Now translate the formula from 5b into plain English without using any variables. Is it easier to read? If some time later you see a complicated sentence as in 5a, would you rather analyze it as a logical formula or try to come up with a simplified version without rewriting it with quantifiers and logical operators?
    \end{enumerate}
    
   


\end{enumerate}

\begin{proof}[Answer]
    a. 
    Let $C$ be the set of all cats and $D$ be the set of all dogs.
    $$\neg (\forall c\in C, \neg( \exists d\in D, \neg (T(c,d) \Rightarrow W(c,d))))\,.$$ 

    b.
    \begin{equation*}
        \exists c \in C, \exists d\in D, T(c,d)\wedge (\neg W(c,d))\,.
    \end{equation*}

    c. There are a cat and a dog who share a toy and cannot walk together.
\end{proof}



\end{document}
