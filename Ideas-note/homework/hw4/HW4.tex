\documentclass[12pt]{amsart}
\usepackage{../marktext} 
%% Remove draft for real article, put twocolumn for two columns
\usepackage{../svmacro}
\usepackage[utf8]{inputenc}
\usepackage{lineno}
%\usepackage{authblk}
\usepackage[style=alphabetic, backend=biber]{biblatex}
\usepackage{enumitem}

%% commentary bubble
\newcommand{\SV}[2][]{\sidenote[colback=green!10]{\textbf{SV\xspace #1:} #2}}

%% Title 
\title{ MATH 170: Homework 4 }
%\author[1]{Co-author}
\author{Due: October 8, 2021}
%\affil[1]{Institute}
\date{}

\begin{document}

\maketitle


\noindent{\bf Graded for accuracy:}
1, 2.
\\
\noindent{\bf Graded for completion:}
3, 4, 5.
\\
\noindent{\bf Not graded (practice problems):}
6, 7, 8.
\\
\noindent{\bf Instructions:}
Problems that are graded for accuracy must be correct to get points.
Problems that are graded for completion must show some trying effort.

\centerline

\hrule

\centerline

\begin{enumerate}[label=\arabic*.,itemsep=10pt, leftmargin=*]

\item Recall that the operator ``IF AND ONLY IF'' (``biconditional operator''), denoted $\Leftrightarrow$, is defined as follows:
$$ P \Leftrightarrow Q
\mbox{ by definition means the formula }
(P\Rightarrow Q) \wedge (Q \Rightarrow P).
$$
Using this definition, write the truth table for this operator.

\item  Using logical rules or truth tables, prove logical equivalences:
    \begin{enumerate}[label=\alph*.,itemsep=5pt, leftmargin=*]
    \item $(P \vee Q) \Rightarrow R \equiv (P \Rightarrow R) \wedge (Q \Rightarrow R)$.
    \item $P \Leftrightarrow Q \equiv
    (P \Rightarrow Q) \wedge ((\neg P) \Rightarrow (\neg Q))$ (use Problem 1 to recover the truth table for $\Leftrightarrow$). 
\end{enumerate}

\item  In plain English, explain the equivalence in 2a.

\item Let $P$ be the set of all people and
     $E$ be the set of all even numbers.
    Translate the following sentences into logical formulae.
    \begin{enumerate}
        \item Everyone loves someone.
        \item Everyone is short or tall.
        \item Everyone is short or everyone is tall.
        \item Every even number is a multiple of 2 with some integer.
    \end{enumerate}



\item  
    \begin{enumerate}[label=\alph*.,itemsep=5pt, leftmargin=*]
    \item
    Translate into purely logical formula the following (quite convoluted) plain English sentence:
    \\
    ``It is not true that for every cat you cannot find a dog such that under no circumstances them sharing a toy means that they go to walks together.''
    \\
    Suggested use of variables: $c$ for a cat, $d$ for a dog, $T(c,d)$ for ``$c$ and $d$ share a toy'', $W(c,d)$ for ``$c$ and $d$ go to walks together''.
    \item 
    Simplify the formula from 5a by moving the negations past the quantifiers so that each negation in the formula only appears right in front of $T(c,d)$ or $W(c,d)$. I.e. write a maximally negated logical formula that is logically equivalent to the formula from 5a.
    \item
    Now translate the formula from 5b into plain English without using any variables. Is it easier to read? If some time later you see a complicated sentence as in 5a, would you rather analyze it as a logical formula or try to come up with a simplified version without rewriting it with quantifiers and logical operators?
    \end{enumerate}
    
    
\item
    \textit{Review: pigeonhole principle.}
    \begin{enumerate}
        \item In a country in a far far away land, there is a run for President with 10 parties and only the candidates can vote for each other.
        Assuming each party has one candidate
        and that each candidate can vote $n$ times, each for a different person (and he/she is allowed to vote for his/herself) -- total count will determine the winner.
        How many votes should each candidate have in order to make sure that at least one of them has at least 4 votes?
        
        \item I have 20 pairs of socks that have 4 different colors.
        How many socks do I have to pull out without looking in order to make sure that I have 2 pairs of the same color?
    \end{enumerate} 
    



\item
    \textit{Review: divisibility and fundamental theorem of arithmetic.}
    \begin{enumerate}
        \item Determine whether the following numbers are divisible by $2,3,5$?
        \begin{equation*}
            181, 21505, 1990 \,.
        \end{equation*}
        
        \item What are the prime factorizations of the following numbers?
        \begin{equation*}
            1990, 256, 134400 \,.
        \end{equation*}
        
%        \item Without calculator, is the following number a prime?
%        \begin{equation*}
%            2*3*5*7+ 1 \,.
%        \end{equation*}
    \end{enumerate}
    
    
    

\item
    \textit{Review: gcd and Euclid's algorithm.}
    Find 
    \begin{gather*}
        \gcd(2501, 81)\\
        \gcd(-15, 60)
    \end{gather*}
    
    


\end{enumerate}



\end{document}
