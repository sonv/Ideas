\documentclass[12pt]{amsart}
\usepackage{../marktext} 
%% Remove draft for real article, put twocolumn for two columns
\usepackage{../svmacro}
\usepackage[utf8]{inputenc}
\usepackage{lineno}
%\usepackage{authblk}
\usepackage[style=alphabetic, backend=biber]{biblatex}
\usepackage{enumitem}

%% commentary bubble
\newcommand{\SV}[2][]{\sidenote[colback=green!10]{\textbf{SV\xspace #1:} #2}}

%% Title 
\title{ MATH 170: Homework 6 }
%\author[1]{Co-author}
\author{Due: October 29, 2021}
%\affil[1]{Institute}
\date{}

\begin{document}

\maketitle


\noindent{\bf Graded for accuracy:}
1,2, 4.
\\
\noindent{\bf Graded for completion:}
3.
\\
\noindent{\bf Instructions:}
Problems that are graded for accuracy must be correct to get points.
Problems that are graded for completion must show some trying effort.

\centerline

\hrule

\centerline

\begin{enumerate}[label=\arabic*.,itemsep=10pt, leftmargin=*]


    \item 
        Let $f:\R \to \R$, $f(x) = x^2$, $g: [0,\infty) \to \R$, $g(x) = \sqrt{x}$
        Determine the formulas for the following compositions of functions.
        \begin{enumerate}
            \item $f\circ g$
            \item $g\circ f$
            \item $g\circ g$
            \item $f \circ g\circ g$
        \end{enumerate}
        
    \item 
        Let $f: X \to Y$ be a function between two sets.
        \begin{enumerate}
            \item Assume that $f$ is a bijection. Show that then you can define a function $g: Y \to X$ such that $f\circ g$ is the identity function of $Y$ and $g \circ f$ is the identity function of $X$.
            \item Assume that $g: Y \to X$ is a function as in the previous part. Prove that then $f$ is necessarily a bijection. 
        \end{enumerate}

    \item Watch Vi Hart's videos about Fibonacci spirals in nature:
    
    Part 1: \url{https://www.youtube.com/watch?v=ahXIMUkSXX0}
    
    Part 2:
    \url{https://www.youtube.com/watch?v=lOIP_Z_-0Hs}
    
    Part 3: \url{https://www.youtube.com/watch?v=14-NdQwKz9w}
    
        \begin{enumerate}
            \item Create an angle $137.5^\circ$ and draw approximately $30$ petals, each $137.5^\circ$ from the previous one. 
            \item When you are done with part (a), mark the spirals and count the number of them.
            \item After watching the third part, in your own words explain why we almost always get a Fibonacci number of spirals in plants.
        \end{enumerate}

\item Prove by induction that for every $n\in \N$
        \begin{equation*}
            1 + 2 + 2^2 + \dots + 2^n = 2^{n+1} - 1 \,.
        \end{equation*}

    
\end{enumerate}



\end{document}
