\documentclass[12pt]{amsart}
\usepackage{../marktext} 
%% Remove draft for real article, put twocolumn for two columns
\usepackage{../svmacro}
\usepackage[utf8]{inputenc}
\usepackage{lineno}
%\usepackage{authblk}
\usepackage[style=alphabetic, backend=biber]{biblatex}
\usepackage{enumitem}

%% commentary bubble
\newcommand{\SV}[2][]{\sidenote[colback=green!10]{\textbf{SV\xspace #1:} #2}}

%% Title 
\title{ MATH 170: Homework 9 }
%\author[1]{Co-author}
\author{Due: Dec 1, 2021}
%\affil[1]{Institute}
\date{}

\begin{document}

\maketitle


\noindent{\bf Graded for accuracy:}
1, 2.
\\
\noindent{\bf Graded for completion:}

\noindent{\bf Instructions:}
Problems that are graded for accuracy must be correct to get points.
Problems that are graded for completion must show some trying effort.

\centerline

\hrule

\centerline

\begin{enumerate}[label=\arabic*.,itemsep=10pt, leftmargin=*]


    \item 
        \begin{enumerate}
            \item If two balanced dice are rolled, what is the probability that the sum of the two numbers that appear will be odd?
            \item If two balanced dice are rolled, what is the probability that the difference between the two numbers that appear will be less than 3?
        \end{enumerate}

    \item 
        In a school, there are 1000 students. 700 of them are in Rock Climbing club, 400
        of them are in Swimming club.
        It is possible that not everyone belongs to any club.
        \begin{enumerate}
            \item What is the probability that a randomly picked student belongs to the Climbing club? What about Swimming club?
            \item Draw the Venn diagram to represent the students and the clubs in the school.
            \item 
                Let $A$ be the event that a  student belongs to Climbing club.
                Let $B$ be the event that a student belongs to Swimming club.
        Determine the maximum and minimum possible values of 
        $\P(A\cap B)$ and the conditions under which each of these values is attained.
        \end{enumerate}
\end{enumerate}



\end{document}
