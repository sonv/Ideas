%%%
%%%
\chapter{Mathematical induction}
%%%
%%%

Mathematical induction is a way to prove a sequence of statements by scaffolding. In a way, it can be compared to inductive logic, because in both cases we start by considering ``small examples'' and from those, we deduce that all of the examples have some property. However, there is an important distinction: mathematical induction is a part of \emph{deductive reasoning}, because it provides a \emph{formal proof}   that yields \emph{correct statements}, and does not just show that these statements are plausible. Here is the formal statement:

\begin{theorem}[The Induction Principle]
Suppose that we have a sequence of statements $P(n)$ labeled by the natural numbers $0,1,2,\dots$ such that we know that 
\begin{enumerate}
    \item $P(0)$ is true, and
    \item $\big( P(0) \wedge P(1) \wedge \cdots \wedge P(n) \big) \Rightarrow P(n+1)$.
\end{enumerate}
  Then all the statements $P(0), P(1), P(2), \dots$ are true.
\end{theorem}


%%%
\section{All natural numbers are interesting}
%%%

As a warm-up, let us prove that all natural numbers are interesting.

\begin{definition}
A natural number $n$ is called \emph{interesting} if it has some special property that no other natural number has.
\end{definition}

\begin{theorem}
All natural numbers are interesting.
\end{theorem}

Let us first do a survey of the first few natural numbers:
\begin{itemize}
\item $0$ is interesting because it is the only number that yields itself when you multiply it by another number.
\item $1$ is interesting because it is the only number that doesn't change the other number when multiplied.
\item $2$ is interesting because it is the first prime number, i.e. the number that has exactly two positive divisors ($1$ and itself).
\item $3$ is interesting because it is the first odd prime number.
\item $4$ is interesting because it is the first nontrivial square: $4 = 2^2$.
\item $5$...
\end{itemize}

We could say that $5$ is a prime number, but it is not the first one of those. However, it is recognizable as a member of a certain sequence that is well-known to mathematicians and even in popular culture -- Fibonacci sequence, which appears in Indian mathematics in connection with Sanskrit prosody (study of poetic metres and verse in Sanskrit).

\begin{exercise}
    Watch this really nice introduction to Fibonacci numbers in nature by Vi Hart (total duration of all combined is 18 minutes):
    
    Part 1: \url{https://www.youtube.com/watch?v=ahXIMUkSXX0}
    
    Part 2:
    \url{https://www.youtube.com/watch?v=lOIP_Z_-0Hs}
    
    Now create an angle $137.5^\circ$, draw approximately $30$ petals, each $137.5^\circ$ from the previous one, and when you are done, mark the spirals and count the number of them. 
    
    Part 3: \url{https://www.youtube.com/watch?v=14-NdQwKz9w}
    
    After watching the third video, can you explain why we almost always get a Fibonacci number of spirals in plants? 
\end{exercise} 
    
And so we can continue finding interesting properties for numbers:
\begin{itemize}
\item $5$ is the first Fibonacci number for which we didn't find another property.
\item $6$ is the first \emph{perfect number}, which by definition means that it is the sum of its positive divisors excluding itself: $6 = 1+2+3$. Incidentally, $6$ is also a \emph{triangular number}, i.e. a sum of consecutive numbers starting from $1$. The next perfect number is $28 = 1 + 2 + 4 + 7 + 14$. 
\item $7$ is the first \emph{Mersenne prime number} for which we didn't find another property. Mersenne prime numbers are prime numbers that can be written as $2^k - 1$, e.g. $7 = 2^3 - 1$, and the next one is $31 = 2^5 - 1$.
\end{itemize}

In fact, even perfect numbers and Mersenne prime numbers are connected by this beautiful theorem:
\begin{theorem}[Euclid-Euler theorem]
If $2^k-1$ is a Mersenne prime number, then $2^{k-1} \cdot (2^k - 1)$ is a perfect number, and all even perfect numbers are of this form.
\end{theorem}

We omit the proof, but the interested reader will be able to read and understand the proof here: \url{https://primes.utm.edu/notes/proofs/EvenPerfect.html}. Before you read, note that $\sigma(n)$ is defined as the \emph{sum of divisors function}, e.g. $\sigma(6) = 1 + 2 + 3 + 6 = 2\cdot 6$, so a number $n$ is perfect exactly when $\sigma(n) = 2n$.

It is still an open question whether there are infinitely many Mersenne prime numbers or perfect numbers. Additionally, it is not even known if there are odd perfect numbers! It has been proved however that there are no odd perfect numbers that have $1500$ digits or less, or in math terms, a lower bound for the odd perfect numbers is $10^{1500}$.

So far, we have seen three sequences of natural numbers: Fibonacci, Mersenne primes, perfect number. For even more, visit this page: \url{https://oeis.org}.

\begin{exercise}
    Imagine the sequence that starts with $1,1,1,1$. What would be the next term? Go to OEIS and see what the encyclopedia shows.
\end{exercise}


%%%
\section{Proof by induction}
%%%

We can talk about numbers, their properties and sequences of numbers all day long, but since there are infinitely many of them, we will never stop the case by case analysis. So let us do all cases at once by induction!

\begin{proof}
Recall the Induction Principle: we first need a list of statements labeled by natural numbers. We have a natural candidate for this:
$$ P(n) = \mbox{``$n$ is interesting''}
.$$
Then we need to prove that $P(0)$ is true, but we have already observed a unique property of $0$. 

Now, let us assume that $P(0) \wedge P(1) \wedge \cdots \wedge P(n)$ is true, or in plain English, that all natural numbers up to $n$ are interesting. We prove that $n+1$ is interesting by contradiction: if it wasn't, then it will have the special property that it is the smallest natural number that is not interesting. Isn't that interesting?! So $P(n+1)$ is true.

Finally, we can apply the Principle of Induction and get that all $P(n)$ are true, i.e. that all natural numbers are interesting.
\end{proof}



%%%
\section{Structure of proofs by induction}
%%%

By analyzing the proof above, we can divide proofs by induction into several steps:
\begin{enumerate}
\item Identify the statements $P(n)$.
\item Step 2 is also called \emph{base of induction}: prove $P(0)$ or $P(1)$ (sometimes it doesn't make sense to talk about $P(0)$).
\item Assume that all $P(k)$ with $k\leq n$ are true -- this is called the \emph{induction hypothesis}. Now perform the \emph{step of induction}: prove that $P(n+1)$ is true.
\item Finally, conclude by the Principle of Induction that all $P(n)$ are true.
\end{enumerate}



%%%
\section{Exercises}
%%%

\begin{example}[Triangular numbers]
Prove that the sum of the first $n$ positive integers is $\frac{n(n+1)}{2}$.
\begin{enumerate}
\item 
Here $P(n)$ means ``$1+2+\cdots+n = \frac{n(n+1)}{2}$'', and we will avoid talking about $P(0)$. Although, strictly speaking, it makes sense, because $P(0)$ simply states that $0=0$.
\item 
$P(1)$ states $1 = \frac{1\cdot2}{2}$, which can be seen is true.
\item 
Now assume that $P(n)$ is true, i.e. $1 + \cdots + n = \frac{n(n+1)}{2}$. Therefore, by induction hypothesis,
$$ 1 + \cdots + n + (n+1) = \frac{n(n+1)}{2} + (n+1)
.$$
We can calculate the sum and get:
$$ 1 + \cdots + n + (n+1) = \frac{n(n+1) + 2(n+1)}{2} =
\frac{(n+2)(n+1)}{2}
.$$
But now the equality that we get is exactly the statement $P(n+1)$.
\item
So by induction, the sum $1+ \cdots + n$ is equal to $\frac{n(n+1)}{2}$ for all natural numbers $n$.
\end{enumerate}
\end{example}

\begin{example}
Prove that the sum of the first $n$ odd positive integers is $n^2$.

Let's first recall that odd integers look like $2k+1$ where $k \in \N$.
Then, the sum of the first $n$ odd positive integers should be
\begin{equation*}
    1 + 3+ 5+ \dots + 2n+1 \,.
\end{equation*}

So, our statement $P(n)$ is 

\begin{enumerate}
    \item $P(n)$ says 
        \begin{equation*}
            1+ 3+ 5+ \dots + 2n+1 = (n+1)^2 \,.
        \end{equation*}
    \item $P(0)$ says ``$0 = 0^2$'' and
     $P(1)$ says ``$1 = 1^2$''.
    \item Assume that for $k > 1$, $P(k)$ is true.
        We want to show that $P(k+1)$ is true.
        By the induction hypothesis,
        \begin{align*}
            &1+ 3+ \dots + (2k+1) + (2(k+1)+1) \\
            &= (k+1)^2 + 2(k+1) + 1 = (k+2)^2 \,.
        \end{align*}
        The last equality is the quadratic formula $(a+b)^2 = a^2 + 2ab + b^2$.
    \item By the induction theorem, $P(k)$ is true for all $k\in \N$.
\end{enumerate}
\end{example}
We also have a picture proof.

\begin{example}
    Prove that the natural number $n^3 -n$ is divisible by  $3$. 

    \begin{enumerate}
        \item $P(n)$ says
            \begin{equation*}
                3 \st (n^3 - n)\,.
            \end{equation*}
        \item $P(0)$ says ``$3 \st 0$'', which is true.
        \item Assume that for $k>0$, $P(k)$ is true.
            This means that there exists a number $l$ such that
            \begin{equation*}
                k^3 - k = 3l\,.
            \end{equation*}
            We want to show that $P(k+1)$ is true.
            We write 
            \begin{align*}
                (k+1)^3 - (k+1) &= k^3 + 3k^2 + 3k + 1 - k -1 \\
                                &= k^3 - k + 3k^2 + 3k = 3(l + k^2 + k) \,.
            \end{align*}
            By definition of divisibility, $P(k+1)$ is true.
        \item By the induction theorem, $P(k)$ is true for all $k\in \N$.
    \end{enumerate}
\end{example}

\begin{example}
   Show that for every $n \in \N$, 
   \begin{equation*}
       5 \st F_{5n} \,.
   \end{equation*}
\end{example}
\begin{example}
   Recall that the Fibonacci sequence is a sequence of number with the following pattern
   \begin{equation*}
       F_0 = 0\,, F_1 = 1\,, \dots, F_n = F_{n-1} + F_{n-2}\,.
   \end{equation*}
   The golden ratio is the following number
   \begin{equation*}
       \varphi = \frac{1 + \sqrt{5}}{2}\,.
   \end{equation*}
   The conjugate of the golden ratio is the following number
   \begin{equation*}
       \phi = \frac{1 - \sqrt{5}}{2}\,.
   \end{equation*}
   Show by induction that
   \begin{equation*}
       F_n = \frac{\varphi^n - \phi^n}{\sqrt5}\,.
   \end{equation*}
\end{example}




