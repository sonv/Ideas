\chapter*{Introduction}

You may still be thinking about your major and trying this class as a part of exploration. Or this may be the only math class that you take in your four years of undergraduate studies, and you may be wondering why the university imposed such a requirement on you. 

Instead of trying to find my own words for a motivational speech, let me cite Abraham Lincoln when he answered in 1864 how he had aquired his persuasive rhetorical skill:

\begin{displayquote}
``In the course of my law-reading I constantly came upon the word
\textit{demonstrate}.
I thought, at first, that I understood its meaning, but soon became satisfied that I did not. \dots
I consulted Webster's dictionary. That told of ``certain proof'', ``proof beyond the possibility of doubt''; but I could form no idea what sort of proof that was. I thought a great many things were proved beyond a possibility of doubt, without recourse to any such extraordinary process of reasoning as I understood ``demonstration'' to be. I consulted all the dictionaries and books of reference I could find, but with no better results. You might as well have defined \textit{blue} to a blind man.
At last I said, ``Lincoln, you can never make a lawyer if you do not understand what \textit{demonstrate} means''; and I left my situation in Springfield, went home to my father's house, and staid there till I could give any propositions in the six books of Euclid at sight. I then found out what ``demostrate'' means, and went back to my law studies.''
\end{displayquote}

We see that the 16th president of the US highly regarded Euclid's ``Elements'' for its teaching of rigor and reasoning, and not as much for its content. The textbook itself is 2300 years old and you may think that it may have outdated material (arguably, this is a correct guess), but it has survived more than a thousand of editions, and mathematicians only came up with other logically consistent geometries in the nineteenth century, thus rendering Euclid's work as one of the many possibilities. For more than two thousand years, this book was considered something all educated people had read, and it only came down this pedestal in the 20th century, by which time its content was universally taught through other school textbooks.

But as much as this book was central to the western European civilization (second to Bible), not all people are fond of planar geometry. In this course, we will offer an alternative invitation to the land of reason by means of showing many possible facets of mathematics. If you wish, you may consider it a collection of trailers for higher-level math courses.