\documentclass[12pt]{amsart}
\usepackage{../marktext} 
%% Remove draft for real article, put twocolumn for two columns
\usepackage{../svmacro}
\usepackage[utf8]{inputenc}
\usepackage{lineno}
%\usepackage{authblk}
\usepackage[style=alphabetic, backend=biber]{biblatex}
\usepackage{enumitem}

%% commentary bubble
\newcommand{\SV}[2][]{\sidenote[colback=green!10]{\textbf{SV\xspace #1:} #2}}

%% Title 
\title{ MATH 170: Practice problems for Midterm 1 }
%\author[1]{Co-author}
%\affil[1]{Institute}
\date{}

\begin{document}

\maketitle


\noindent{\bf Not graded.}

\centerline

\hrule

\centerline

\begin{enumerate}[label=\arabic*.,itemsep=10pt, leftmargin=*]
\item  
Consider the following sets:
\begin{itemize}
    \item $A$ is the set of all animals;
    \item $B$ is the set of animals that eat bamboo;
    \item $P$ is the set of animals that eat plants;
    \item $M$ is the set of animals that eat meat.
\end{itemize}
    \begin{enumerate}[label=\alph*.,itemsep=5pt, leftmargin=*]
    \item Draw the Venn diagram that correctly represents the inclusions of the above four sets. Note that fish and insects are not meat.
    \item Is $P \cup M$ a proper subset of $A$?
    \item Is $B \cup M$ a proper subset of $B$?
    \item Based on the previous parts, determine if the following proposition is true:
    $$ \forall a \in A :
    ``a \mbox{ eats plants''} \vee
    ``a \mbox{ eats meat''}
    .$$
    \end{enumerate}

\item  Consider a set $U$ and four of its subsets $X$, $Y$, $Z$ such that we know:
$$ X \cap Y \subset Z
.$$
    \begin{enumerate}[label=\alph*.,itemsep=5pt, leftmargin=*]
    \item Draw the Venn diagram that  represents the inclusions of the above four sets. Make sure that there is no region corresponding to $(X\cap Y) \setminus Z$, since it is the empty set by assumption.
    \item Determine if this formula is necessarily true: $Z \cap Y \subset X$.
    \item Determine if this formula is necessarily true: $Z \setminus (X \cap Y) = \varnothing$.
    \item Determine if this formula is necessarily true: 
    $$(X \Delta Y) \cap Z = (X\cap Z) \Delta (Y \cap Z)
    .$$
    \item For each formula above that was not correct, draw a new Venn diagram, for which the formula would be correct.
\end{enumerate}



\item  Consider the following function 
$f : \mathbb{N} \to \mathbb{N}$ that takes $n>0$ to the number of its positive divisors, and let $f(0) = 0$. For example:
$f(1) = 1$,
$f(2) = 2$,
$f(4) = 3$,
$f(6) = 2$.
    \begin{enumerate}[label=\alph*.,itemsep=5pt, leftmargin=*]
    \item Calculate $f(3)$, $f(5)$, $f(7)$, $f(8)$, $f(9)$, $f(16)$.
    \item Show that $p$ is a prime number if and only if $f(p)=2$.
    \item Show that for every prime number $p$ and any $k\in \mathbb{N}$, we have $f(p^k) = k+1$.
    \item Is $f$ injective, surjective, bijective? Why / why not?
    \item Draw the graph of this function (enough to draw up to the value at 9).
    \end{enumerate}

\item Show that for every $n\in \N$
\begin{equation*}
    F_{n+2} F_n - F_{n+1}^2 = (-1)^{n+1}    
\end{equation*}
where $F_n$ is the Fibonacci sequence.


\end{enumerate}



\end{document}
