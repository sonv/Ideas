\documentclass[12pt]{amsart}
\usepackage{../marktext} 
%% Remove draft for real article, put twocolumn for two columns
\usepackage{../svmacro}
\usepackage[utf8]{inputenc}
\usepackage{lineno}
%\usepackage{authblk}
\usepackage[style=alphabetic, backend=biber]{biblatex}
\usepackage{enumitem}

%% commentary bubble
\newcommand{\SV}[2][]{\sidenote[colback=green!10]{\textbf{SV\xspace #1:} #2}}

%% Title 
\title{ MATH 170: Topics for midterm 2 }
%\author[1]{Co-author}
\author{Good luck!}
%\affil[1]{Institute}
\date{}

\begin{document}


\maketitle


\section*{Rules}

\begin{itemize}
    \item The exam is on Dec 8 (Wednesday) for Section 601, on Dec 9 (Thursday) for Section 1, and on Dec 10 (Friday) for Section 2.
    \item Open notes (you can use your notes freely).
    \item No use of internet, textbooks, computer algebra systems, calculators. 
    \item No collaboration.
    \item The exam is during lecture and 50 minutes long.
    \item At the end of the exam, you will have 10 min to scan and upload your work to gradescope.
\end{itemize}


\section*{Scanning}

The easiest way to scan is to use the free version of the app ``Adobe Scan'' that is available for iPhone and Android.

\textbf{Before the midterm.} Download the app and try to use it at least once to create a pdf document out of 5 pages. E.g. scan your lecture notes.

\textbf{Using the app.} 
\begin{enumerate}
    \item Open the app.
    \item Allow it to access camera.
    \item Now the app should show you the camera and attempt to find a document.
    \item Move your camera so that the document you chose (lecture notes, shopping list, etc.) fits completely into the screen.
    \item The app will highlight the corners of the document with blue dots. Wait till the app makes the photo of the document (no need to press buttons). 
    \item Now the app will suggest you to drag the blue dots to adjust the corners of the document.
    \item Click ``Continue'' to return to the camera mode.
    \item Scan another page. Or if you are done, click the bottom right where you see the miniatures of the scans. You will then be able to review the pdf as it will appear in the file, reorder, crop or apply filters.
    \item When you are done reviewing, click ``Save PDF'' in the top right corner.
    \item Now you should see your document. Click ``Share'', choose ``Share a copy'', then choose any appropriate way of uploading the file to Gradescope. E.g. you can save the file to your file system and upload it from there, email it to yourself and use your laptop to upload, or come up with any other means of uploading the file to Gradescope.
    
\end{enumerate}



\section*{Advice}

\begin{itemize}
    \item {\bf Zeroth principle.}
On the day of the midterm, make sure that you have enough sleep (at least 7 hours), and don't forget to eat lunch and drink water beforehand -- your brain will thank you for taking care of it.   
    \item {\bf First principle.}
Make sure you learn how to present your work properly by looking
    at the solution / lecture notes for sample writings.
    Learning to have a style of your own is important.
    Going over solutions of homework is a good idea.
    \item Even though you are allowed to use your notes, prepare a cheat sheet with formulas for faster search during the midterm.

\end{itemize}




\section*{List of topics}

\begin{enumerate}

    \item Graphs:
        \begin{itemize}
            \item Eulerian paths and cycles, and when they exist
            \item Graph dual to a map
            \item Chromatic number
            \item Chromatic polynomial, deletion-contraction formula
            \item Planar graphs, Euler characteristic
        \end{itemize}

    \item Probability theory
        \begin{itemize}
            \item Probability spaces, outcomes, events, counting probability
            \item Factorials, binomial coefficients
            \item Independent events
            \item Conditional probability, Bayes' theorem
            \item  NOT INCLUDED: statistics (random variables, expectation, standard deviation)
        \end{itemize}

\end{enumerate}



\end{document}
