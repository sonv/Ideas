\chapter{Sets and functions}
\section{Sets}

The notion of a \emph{set} is extremely fundamental (as we already used it in 
previous discussions) yet, if not defined carefully, could lead to paradoxes and
break mathematics from its core.
Perhaps the most infamous paradox in naive set theory  is the so-call Russell's paradox.
However, for the sake of sanity in this course, let's stick with the naive notion 
and trust that mathematicians already fixed this issue and unbroke mathematics...
\begin{definition}
   A set is a collection of objects. 
   The objects in the sets are called \emph{elements} of the set.
   If $x$ is an element in the set $X$ then we write $x\in X$.
   We write $x\not\in X$ to mean $\neg (x\in X)$.
\end{definition}

The way that Bertrand Russell broke naive set theory is via the following chain of reasoning, taken directly from Wikipedia:

\begin{displayquote}
   Let $R$ be the set of all sets that are not members of themselves.
   If $R$ is not a member of itself, then its definition entails that 
   it is a member of itself; if it is a member of itself, 
   then it is not a member of itself, since it is the set of 
   all sets that are not members of themselves. 
\end{displayquote}

We will need to use the so-called \emph{set-builder notation} to
describe sets in general.
Given a set $X$, set set of elements of $X$ satisfying some property $P(x)$ is denoted by
\begin{equation*}
\set[\big]{ x \in X \st P(x)} \,.
\end{equation*}


From now on, we define the set of all rationals to be 
\begin{equation*}
    \Q = \set[\big]{ p / q \st p \in \Z, q\in \Z \text{ and } q\not= 0} \,,
\end{equation*}
and the set of real numbers to be
\begin{equation*}
\R = \set[\big]{ \text{rationals and irrational numbers}}\,.
\end{equation*}
We are cheating in the definition of real numbers above but that is too technical for the moment. Let's just go with what you imagine it to be from high school.
We will also use the usual open and closed set notations.
\begin{gather*}
[a,b] = \set[\big]{x \in \R \st a \leq x \leq b } \,,\\
[a,b) = \set[\big]{x \in \R \st a \leq x <  b } \,, \\
(a,b] = \set[\big]{x \in \R \st a < x \leq  b } \,, \\
(a,b) = \set[\big]{x \in \R \st a < x <  b } \,.
\end{gather*}

See Newstead~\cite{Newstead} (Chapter 2) for representations of intervals on the number line.

\begin{definition}
    Let $X$ be a set. A \emph{subset} of $X$ is a set $U$ such that 
    \begin{equation*}
        \forall a, ( a \in U \implies a \in X) \,.
    \end{equation*}
    We write $U \subseteq X$ for the assertion that $U$ is a subset of $X$.
    The notation $\subsetneqq$ means that $U$ is a \emph{proper subset} of $X$, 
    that is a subset of $X$ that is not equal to $X$.
\end{definition}

In order to prove that $U$ is a subset of $X$, it is sufficient to take an 
arbitrary element $a\in U$ and prove that $a \in X$.

\begin{example}
    Prove that $\Z \subset \Q$.
\end{example}

We want to be able to say when two sets are the same (equal) with each other.
There are different ways to go about this.
One can say that two sets are equal if they have the exact same definition or
there definition are somehow logically equivalent to each other.
In real practice, these are not very useful.
However, there are those who believe that ``we are what we are made of.''
This sounds like a reasonable description and is a criteria to distinguish two different
people. Surely, even though our bodies might be made of atoms
but my atoms are different than your atoms.
This inspires the following axiom about set equality.


\begin{axiom}[Axiom of extentionality]
   Let $X$ and $Y$ be sets. Then $X = Y$ if and only if $X \subseteq Y$ and $Y\subseteq X$. 
\end{axiom}

\begin{example}
   Prove that, 
   \begin{equation*}
   \set[\big]{ x \in \R \st x^2 \leq 1 } = [-1,1] \,.
   \end{equation*}
\end{example}

One question may arise when we define intevals.
Consider, $(a,b)$, for example. Some people might object this definition because
if $a >b$ we have a contradiction 
\begin{equation*}
    b< a < x < b \,.
\end{equation*}
How can this be?
A careful look at this definition, we realized that it says, 
if $x \in \R$ and $ a< x <b$, then we admit $x$ into the set.
If the description itself does not make sense, we can't even start 
to consider anything, let alone the admission.
When this situation, the set $(a,b)$ simply does not contain any element, and we 
call that an empty set.

\begin{definition}
    A set is~\emph{non-empty} if it contains at least one element.
    Otherwise, it is~\emph{empty}.
\end{definition}

\begin{question}
   How many empty sets are there? 
\end{question}

\begin{proof}[Answer]
   There is only one empty set. That is if $E'$ and $E$ are both empty set, then
   \begin{equation*}
       E' = E \,.
   \end{equation*}
   To see this, we want to show $E\subseteq E'$ and $E' \subseteq E$ (axiom of extentionality).
    By definition,
    Let $a$ be an element in the universe that $E$ belongs to.
    $a \in E $ is always false because $E$ is empty.
    Therefore, the statement
    \begin{equation*}
        a\in E \implies a\in E'
    \end{equation*}
    is always true.
    By definition of subsets, $E \subseteq E'$.
    Likewise, $E' \subseteq E$, showing our claim.
\end{proof}


\subsection{Set operations}

We will introduce some basic operations on sets. There are many more but
the interested reader could find them in fuller details in Newstead's book.

\begin{definition}[Pairwise intersection]
    Let $X$ and $Y$ be sets. 
    The \emph{pairwise intersection} of $X$ and $Y$, denoted $X \cap Y$
    is defined by
    \begin{equation*}
        X \cap Y \defeq \set{ a \st a \in X \wedge a \in Y}\,.
    \end{equation*}
\end{definition}

\begin{example}
    Let $X = \set{1,2,3,4}$ and $Y = \set{2,4,7}$, then
    \begin{equation*}
        X \cap Y = \set{ 2,4} \,.
    \end{equation*}
\end{example}


\begin{definition}[Pairwise union]
    Let $X$ and $Y$ be sets. 
    The \emph{pairwise union} of $X$ and $Y$, denoted $X \cup Y$
    is defined by
    \begin{equation*}
        X \cup Y \defeq \set{ a \st a \in X \vee a \in Y}\,.
    \end{equation*}
\end{definition}

\begin{example}
    Let $X = \set{1,2,3,4}$ and $Y = \set{2,4,7}$, then
    \begin{equation*}
        X \cup Y = \set{ 1,2,3,4, 7} \,.
    \end{equation*}
\end{example}


\begin{definition}[Relative complement]
    Let $X$ and $Y$ be sets. 
    The \emph{relative complement} of $Y$ and $X$, denoted $Y \setminus X$
    is defined by
    \begin{equation*}
        Y \setminus X \defeq \set{ a \st a \in Y \wedge a \not\in X}\,.
    \end{equation*}
\end{definition}

\begin{example}
    Let $X = \set{1,2,3,4}$ and $Y = \set{2,4,7}$, then
    \begin{gather*}
        Y \setminus X  = \set{ 7} \,,\\
        X \setminus Y  = \set{ 1, 3} \,.
    \end{gather*}
\end{example}


\subsection{Venn diagram}
For this part, please refer to in-class lecture. 


You can prove the following using Venn diagram.
\begin{theorem}[de Morgan's laws for sets]
    Let $X,Y, Z$ be sets. We have
    \begin{enumerate}
        \item $X \setminus (Y\cup Z) = (X\setminus Y) \cap (X \setminus Z)$,
        \item $X \setminus (Y\cap Z) = (X\setminus Y) \cup (X \setminus Z)$.
    \end{enumerate}
\end{theorem}

\begin{definition}[Ordered pair]
    For any two objects, $x$ and $y$, an ordered pair $(x,y)$ is the notation for the 
    two objects being arranged in that particular order. 

    Thus, $(x,y)\not = (y,x)$ unless $x=y$.
\end{definition}
\begin{definition}[Cartesian product]
    Let $X$ and $Y$ be sets. The Cartesian product of $X$ and $Y$, denoted by $X\times Y$
    is the set of all ordered pairs $(x,y)$ such that $x\in X$ and $y\in Y$.
    In set-builder notation
    \begin{equation*}
        X\times Y \defeq \set{(x,y)\st x\in X, y\in Y}\,.
    \end{equation*}
    
\end{definition}


%%%
\subsection{Application of set theory thinking to sudoku}

\begin{theorem}[Phistomefel ring]
In a completed sudoku grid, the set of numbers (with multiplicities) in the blue region is exactly the same as in the red region.

\begin{center}
\begin{tabular}{|c|c|c||c|c|c||c|c|c|}
\hline
\cellcolor{red!50}&\cellcolor{red!50}&&&&&&\cellcolor{red!50}&\cellcolor{red!50} \\
\hline
\cellcolor{red!50}&\cellcolor{red!50}&&&&&&\cellcolor{red!50}&\cellcolor{red!50} \\
\hline
&& \cellcolor{blue!50} & \cellcolor{blue!50} & \cellcolor{blue!50} & \cellcolor{blue!50} & \cellcolor{blue!50} && \\
\hline
\hline
&& \cellcolor{blue!50} &&&& \cellcolor{blue!50} && \\
\hline
&& \cellcolor{blue!50} &&&& \cellcolor{blue!50} && \\
\hline
&& \cellcolor{blue!50} &&&& \cellcolor{blue!50} && \\
\hline
\hline
&& \cellcolor{blue!50} & \cellcolor{blue!50} & \cellcolor{blue!50} & \cellcolor{blue!50} & \cellcolor{blue!50} && \\
\cellcolor{red!50}&\cellcolor{red!50}&&&&&&\cellcolor{red!50}&\cellcolor{red!50} \\
\hline
\cellcolor{red!50}&\cellcolor{red!50}&&&&&&\cellcolor{red!50}&\cellcolor{red!50} \\
\hline
\end{tabular}
\end{center}

\end{theorem}


%%%
\section{Functions}
%%%

We all talk about functions.
So much in math classes that we  almost think they are synonyms.
Believe it or not, they are not synonyms and a function has its own definition.

\begin{definition}[Function]
    A \emph{function} $f$ from a set $X$ to a set $Y$ is a specification of elements
    $f(x)\in Y$ for $x\in X$, such that
    \begin{equation*}
        \forall x \in X, \exists! y \in Y, y = f(x)\,.
    \end{equation*}
    The symbol $\exists !$ represents the phrase``there exists a \emph{unique}''. 
    The unique element $f(x)\in Y$ is called the \emph{value} of $f$ at $x\in X$.

    $X$ is called the \emph{domain} of $f$ and $Y$ is called the \emph{codomain} of $f$.
\end{definition}

We next discuss how to specify a function so that it satisfies the above definition.

\begin{enumerate}
    \item \emph{Totality.} A value $f(x)$ should be specified for each $x\in X$ -- 
        this corresponds to the quantifier $\forall x$.
    \item \emph{Existence.} For each $X$, the specified value $f(x)$ should exist, and should be an element in $Y$.
    \item \emph{Uniqueness.} For each $x$ the specified value $f(x)$ should refer to only one 
        element in $Y$.
\end{enumerate}
(2) and (3) correspond to the quantifier $\exists ! y$.

\begin{example}
    The following are functions:
    \begin{enumerate}
        \item $f: X\to X$, where $f(x) = x$ for any set $X$.
            This function is called the \emph{identity function}.
        \item $f:\emptyset \to X$ is called the \emph{empty function}.
            It has no values since there is no element in its domain.
        \item $f:\set{1,2,3} \to \set{ \text{red, yellow, green, blue}}$ 
            where
            $f(1)= red$, $f(2)= blue$, $f(3) = blue$.
        \item $g: \R \to \R$, where  $g(x) = 2x$.
    \end{enumerate}

    The following are NOT functions:
    \begin{enumerate}
        \item $f:\set{1,2,3} \to \set{ \text{red, yellow, green, blue}}$ 
            where
            $f(1)= red$,  $f(3) = blue$.
        \item $g: \R \to \R$, where 
            \begin{equation*}
                g(x) = \frac{1}{x} \,.
            \end{equation*}
    \end{enumerate}
\end{example}

\begin{definition}[Graph of a function]
   Let $f:X \to Y$ be a functin. 
   The \emph{graph} of $f$ is the subset $\mathrm{Gr}(f) \subseteq X\times Y$ defined by
   \begin{equation*}
       \mathrm{Gr}(f) \defeq \set{(x,f(x))\st x\in X} = \set{(x,y)\in X\times Y \st y = f(x)}\,.
   \end{equation*}
\end{definition}

The graph of a function is perhaps the most important idea in modern mathematics as one can graphically draw functions
on paper or turn graphs on papers into mathematical equations so that computations can be done.
It is this idea that bridges geometry and calculus together.
It is not an understatement to say science (and pseudo-science!) would not reach the its height today without this simple idea of Decartes.

\begin{example}
    Graph of the function $f: \R \to \R$, $f(x) = x/2$ is
    \begin{equation*}
\mathrm{Gr}(f) = \set[\Big]{ \paren[\Big]{x, \frac{x}{2}}\st x\in \R}\,.
    \end{equation*}
\end{example}
    
\begin{definition}[Composition of functions]
    Given two functions $f: X\to Y$ and $g: Y \to Z$. Their \emph{composite} $g\circ f$ 
    (read $g$ composed with $f$) is the function $g\circ f: X\to Z$, defined by
    \begin{equation*}
        (g\circ f)(x) = g(f(x)) \quad \text{ for all } x\in X\,.
    \end{equation*}
\end{definition}

\begin{example}
    Let $f:[0,\infty) \to [0,\infty)$ be that $f(x) = x^3$ and $g: [0,\infty) \to [0,\infty)$ be that
    $g(x) = \frac{1}{1+ x}$.
    Then, $f\circ g: [0,\infty) \to [0,1]$ is given by
    \begin{equation*}
        (f\circ g)(x) = f(g(x)) = (g(x))^3 = \frac{1}{(1+ x)^3}\,.
    \end{equation*}

    What happens if I keep the above formula and change
    $f:\R \to \R$ and $g: \R\setminus\set{-1} \to [0,\infty)$?
\end{example}

\begin{example}
    We can write the function $M: \Q \to \Q$, $M(x) = \frac{(2x+ 5)^2}{6}$ 
    as a composition as follows.
    \begin{equation*}
        M = ((k\circ h) \circ g)\circ f \,,
    \end{equation*}
    where 
    \begin{itemize}
        \item $f:\Q \to \Q$ is defined by $f(x) = 2x$,
        \item $g: \Q \to Q$ is defined by $g(x) = x + 5$,
        \item $h: \Q \to \Q$ is defined by $h(x) = x^2$,
        \item $k:\Q \to \Q$ is defined by $k(x) =  \frac{x}{6}$.
    \end{itemize}
\end{example}


\subsection{Injections and surjections}

The concepts of injections and surjections play very crucial roles in mathematics.
Among other use, they are the tools for mathematicians to compare sizes of sets.
It is with these concepts that one could talk about different sizes of infinities!
We will just list definitions here, for a full reading, please read~\cite{Newstead}.
He does a fantastic job there discussing the concepts so there's no need to copy
his text to here.
    
\begin{definition}[Injection]
    A function $f:X \to Y$ is \emph{injective} (or \emph{one-to-one}) if 
    \begin{equation*}
        \forall a \in X, \forall b \in X, f(a) = f(b) \Rightarrow a = b\,.
    \end{equation*}
    An injective function is said to be an \emph{injection}.
\end{definition}

\begin{example}
    Let $f:\Z \to \Z$ be a function that $f(n) = 2n +1$.
    We will show that $f$ is injective. 
    Fix some $m,n \in \Z$ and suppose that $f(m) = f(n)$.
    By definition, we have
    \begin{equation*}
        2m + 1= 2n + 1 \iff m = n \,.
    \end{equation*}
    Therefore,  $f$ is injective.
\end{example}


\begin{example}
    Let $f:\R \to [0,\infty)$ be a function that $f(x) = x^2$.
    $f$ is not injective since $f(-1)= f(1) = 1$, for example.

    However, if we change the domain of $f$ so that
    $f:[0,\infty) \to [0,\infty)$, it would be injective (why?).
\end{example}

\begin{proposition}
    Let $f: X \to Y$ and $g:Y\to Z$ be functions. 
    If $f$ and $g$ are injective, then $f\circ g$ is injective.
\end{proposition}

\begin{proof}
    Let $a,b\in X$ and suppose that $(f\circ g)(a) = (f\circ g)(b)$. 
    By definition of composition,
    \begin{equation*}
        f(g(a)) = f(g(b))\,.
    \end{equation*}
    Because $f$ is injective, $g(a) = g(b)$; and because $g$ is injective, $a=b$.
    Because $a,b$ are arbitrary in $X$, $(f\circ g)$ is injective.
\end{proof}

\begin{definition}[Surjection]
    A function $f:X \to Y$ is \emph{surjective} (or \emph{onto}) if 
    \begin{equation*}
        \forall y \in Y, \exists x \in X, f(x) = y \,.
    \end{equation*}
    An surjective function is said to be an \emph{surjection}.
\end{definition}

\begin{example}
    Fix $n \in \N$ with $n >0$ and define a function 
    $r: \Z \to \set{0, 1, \dots, n -1}$ by letting
    $r(a)$ be the remainder of $a$ when divided by $n$.
    This function is surjective since for each $k \in \set{0, 1, \dots, n-1}$,
    we have $r(k) = k$.

    This function is not injective, however (why?).
\end{example}

\begin{definition}[Bijection]
    A function $f:X\to Y$ is bejective if it is both injective and surjective.
    A bijective function is called a bijection.
\end{definition}

\begin{example}
    Let $f: \Z \to \Z$ be a function that $f(n) = n +10$.
    Then $f$ is a bijective.

    When we change the domain of $f$ so that $f: \N \to \Z$, it is no longer
    a bijective. Which fails -- injectivity or surjectivity?
\end{example}







