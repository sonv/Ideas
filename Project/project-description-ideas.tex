\documentclass[12pt]{amsart}
\usepackage{amsaddr}
\usepackage{marktext} 
%% Remove draft for real article, put twocolumn for two columns
\usepackage{svmacro}
\usepackage[utf8]{inputenc}
\usepackage{lineno}
\usepackage[style=alphabetic, backend=biber]{biblatex}
\addbibresource{bibliography.bib}

%% commentary bubble
\newcommand{\SV}[2][]{\sidenote[colback=green!10]{\textbf{SV\xspace #1:} #2}}

%% Title 
\title{ Project description }
\author{Ideas of Math, Fall 2024}

%\author{Co-author}
%\address{  }
%\email {  }
%
\date{\today}

\begin{document}
\maketitle


\section{Description}
The goal of this class project is to provide the students a chance to work with
the concepts they learn in MATH 102 on certain topics of their chosen.
The project is about a mathematical concept that has not been discussed in class.

The students must produce a mathematical report and a deliverable (video or presentation).

There are two options for the deliverable:
\begin{enumerate}
	\item Create a video
	\item Create a poster / Powerpoint presentation
\end{enumerate}

\section{Timeline}
\begin{enumerate}
	\item Each group is given a topic on August 27, 2024.
	\item Team/topic change requests must be submitted via email by September 6, 2024.
	\item Meeting with Prof. to talk about initial progress.
	      (Schedule by Google Sheet from Oct 21-25).
	\item Your written report is due Friday, Dec 1, 11:59 PM. Note
	      \begin{itemize}
		      \item The report will be rejected if it is not written in LaTeX.
		      \item Late work will be rejected.
	      \end{itemize}
	\item Video showoff / presentation in class on December 3.
\end{enumerate}

\section{Topics and Groups}
There are 6 topics and 11 groups in total. (What does it say about the number of groups with the topics?)
Please follow the link to the Google Sheet on Canvas to know your assigned group and topic.

\section{Requirement and evaluation}
There are two objectives:
\begin{enumerate}
	\item You must write a full mathematical report. The report will be from 3-5 pages.
	\item You present a mathematical concept to other people. The presentation will be power point and you will have from 10-15 minutes to discuss the topic.
	      The topics and contents are of your choice.
	      Basically, based on your reading (of the assigned text or even more), what you find most interesting and the basic foundation leading to it. Maybe discuss applications if you find any.
\end{enumerate}
Both will be evaluated using the following criteria (adapted from a grading Rubric
by Kathryn Mann of Cornell):

\section*{I. Oral Presentation}

\subsection*{a) Content (3/5)}
\begin{itemize}
	\item Appropriate choice of material given the time constraint
	\item Mathematical concepts (examples, theorem statements, etc.) explained correctly
	\item Sufficient definitions, illustrations, and/or motivation are given
	\item The mathematical material is at a level appropriate for math 130 students to understand
\end{itemize}

\subsection*{b) Clarity and Delivery (2/5)}
\begin{itemize}
	\item Board writing is clear and large enough. (Same for diagrams/pictures if you draw any.)
	\item Speaking is clear and well-paced; the presenter faces the audience rather than the board when possible
	\item Concepts are clearly and concisely explained
	\item The presentation appears rehearsed and is within the time limit. (Note: it is okay – in fact, I encourage you – to have some written notes with you while you present. But ideally, you should not have to look at them very much!)
\end{itemize}

\section*{II. Written Report}

\subsection*{a) Mathematical Content (4/10)}
\begin{itemize}
	\item Is the mathematics consistent and correct?
	\item Is it at a level of sophistication appropriate for this class?
	\item Are topics and ideas introduced with sufficient explanation?
	\item If there are pictures, figures, or examples, are they accurate, appropriately used, and do they support the text?
\end{itemize}

\subsection*{b) Clarity of Mathematical Exposition (3/10)}
\begin{itemize}
	\item Are topics presented in a logical order?
	\item Does the paper achieve an appropriate balance of conciseness and explanation?
	\item Are complicated parts/proofs (if any) broken into steps?
\end{itemize}

\subsection*{c) Style (3/10)}
\begin{itemize}
	\item Is the paper clearly written, in paragraph form?
	\item Is the grammar, spelling, and sentence construction correct?
	\item Does the introduction serve its purpose?
	\item Is the paper readable and does it flow?
\end{itemize}


\section{Advice}
Start early and meet with the instructor regularly when you have questions.



Here's a good guideline:
\url{https://e.math.cornell.edu/people/mann/classes/berkeley/writingadvice.pdf}


%\bibliography{refs}
%\bibliographystyle{halpha-abbrv}


\end{document}
