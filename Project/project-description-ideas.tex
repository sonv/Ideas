\documentclass[12pt]{amsart}
\usepackage{amsaddr}
\usepackage{marktext} 
%% Remove draft for real article, put twocolumn for two columns
\usepackage{svmacro}
\usepackage[utf8]{inputenc}
\usepackage{lineno}
\usepackage[style=alphabetic, backend=biber]{biblatex}
\addbibresource{bibliography.bib}

%% commentary bubble
\newcommand{\SV}[2][]{\sidenote[colback=green!10]{\textbf{SV\xspace #1:} #2}}

%% Title 
\title{ Project description }
\author{Ideas of Math, Fall 2023}

%\author{Co-author}
%\address{  }
%\email {  }
%
\date{Nov 1, 2023}

\begin{document}
\maketitle


\section{Description}
The goal of this class project is to provide the students a chance to work with
the concepts they learn in MATH 102 on certain topics of their chosen.
The project must be about a mathematical concept that has not been discussed in class
        or a specific use of the idea discussed in class to solve certain problem.
        
        The students must produce a mathematical report and a deliverable (video or presentation).

There are two options for the deliverable:
\begin{enumerate}
    \item Create a video  
    \item Create a poster / Powerpoint presentation
\end{enumerate}

\section{Timeline}
In order to sign up for doing a project, you must follow the following steps:
\begin{enumerate}
    \item Meet with the instructor to discuss your proposal of a project before Friday, October 10. 
        In this meeting, you are expected to outline an idea for your project and 
        how you would approach the project.
        The instructor will give you feedback and guide you on what to do.
    Failure to make this deadline will resolve in you not eligible for the project.
    \item Sketch out detailed plan and necessary steps in a follow up meeting after
    doing some research/reading about the topic by Nov 17.
    \item Compete the first draft (written in LaTeX) of your written report by Friday, Dec 1.
        The first draft will be rejected if it is not written in LaTeX. 
        Failure to make this deadline will disqualify you.
    \item Video showoff / presentation in class on December 5.
    \item Final draft due December 15.
\end{enumerate}

\section{Requirement and evaluation}
There are two objectives:
\begin{enumerate}
    \item You must write a full mathematical report.
    \item You present a mathematical concept to other people.
\end{enumerate}
Both will be evaluated using the following criteria (adapted from a grading Rubric
by Kathryn Mann of Cornell):

\section*{I. Oral Presentation}

\subsection*{a) Content (3/5)}
\begin{itemize}
    \item Appropriate choice of material given the time constraint
    \item Mathematical concepts (examples, theorem statements, etc.) explained correctly
    \item Sufficient definitions, illustrations, and/or motivation are given
    \item The mathematical material is at a level appropriate for math 130 students to understand
\end{itemize}

\subsection*{b) Clarity and Delivery (2/5)}
\begin{itemize}
    \item Board writing is clear and large enough. (Same for diagrams/pictures if you draw any.)
    \item Speaking is clear and well-paced; the presenter faces the audience rather than the board when possible
    \item Concepts are clearly and concisely explained
    \item The presentation appears rehearsed and is within the time limit. (Note: it is okay – in fact, I encourage you – to have some written notes with you while you present. But ideally, you should not have to look at them very much!)
\end{itemize}

\section*{II. Written Report}

\subsection*{a) Mathematical Content (4/10)}
\begin{itemize}
    \item Is the mathematics consistent and correct?
    \item Is it at a level of sophistication appropriate for this class?
    \item Are topics and ideas introduced with sufficient explanation?
    \item If there are pictures, figures, or examples, are they accurate, appropriately used, and do they support the text?
\end{itemize}

\subsection*{b) Clarity of Mathematical Exposition (3/10)}
\begin{itemize}
    \item Are topics presented in a logical order?
    \item Does the paper achieve an appropriate balance of conciseness and explanation?
    \item Are complicated parts/proofs (if any) broken into steps?
\end{itemize}

\subsection*{c) Style (3/10)}
\begin{itemize}
    \item Is the paper clearly written, in paragraph form?
    \item Is the grammar, spelling, and sentence construction correct?
    \item Does the introduction serve its purpose?
    \item Is the paper readable and does it flow?
\end{itemize}

\section{Possible topics}
You can choose one of the following topics, but you are not restricted to them.
\begin{enumerate}
    \item Social: applications of mathematics in arts and music, visualizing math 
    \item Concepts: topology, knots, sums of squares theorems
\end{enumerate}


\section{Advice}
Start early and meet with the instructor regularly when you have questions.

A guide for math writing:
\url{https://math.berkeley.edu/~kpmann/writingadvice.pdf}

\printbibliography 
%\bibliography{refs}
%\bibliographystyle{halpha-abbrv}


\end{document}
