\documentclass[12pt]{amsart}
\usepackage{amsaddr}
\usepackage{marktext} 
%% Remove draft for real article, put twocolumn for two columns
\usepackage{svmacro}
\usepackage[utf8]{inputenc}
\usepackage{lineno}
\usepackage[style=alphabetic, backend=biber]{biblatex}
\addbibresource{bibliography.bib}

%% commentary bubble
\newcommand{\SV}[2][]{\sidenote[colback=green!10]{\textbf{SV\xspace #1:} #2}}

%% Title 
\title{ MATH 102: Ideas  of Math }
\author{ Worksheet 10 }

\date{Nov 8, 2023}

\begin{document}

\maketitle

\section{Concepts}

    \begin{definition}[Injective functions]
        Let $f:A\to B$ be a function.
        $f$ is said to be \emph{injective} (or \emph{one-to-one})
        if 
        \begin{equation*}
            \forall a_1, a_2 \in A, f(a_1) = f(a_2) \implies a_1 = a_2 \,.
        \end{equation*}
        An injective function is called an \emph{injection}.
    \end{definition}


    \begin{definition}[Surjective functions]
        Let $f:A\to B$ be a function.
        $f$ is said to be \emph{surjective} (or \emph{onto}) if
        \begin{equation*}
            \forall b\in B \exists a \in A, f(a) = b \,.
        \end{equation*}
        A surjective function is called a \emph{surjection}.
    \end{definition}

    \begin{definition}[Bijective function]
        A function $f$ is \emph{bijective} if 
        it is both injective and surjective.

        A bijective function is called a \emph{bijection}.
    \end{definition}

    \begin{definition}
        Let $A$ be a set. A function $f:A \to A$ is called an identity function if $f(a) =a$.

        An identity function on a set $A$ is denoted by $\text{id}_A$.
    \end{definition}


    \begin{definition}
        Let $f:A\to B$ and $g:B\to C$ be functions.
       The composition of $g$ and $f$, denoted by $g\circ f: A\to C$, is defined
       to be
       $$(g\circ f)(a) = g(f(a))\,,$$
       for all $a\in A$.
    \end{definition}
\section{Problems}

\begin{problem}
    Let $f:\R\to \R$ and $g:\R \to [0,\infty)$ be such that
    $f(x) = 2+ 5x$, $g(x) = x^2 +3$.
    Find
    \begin{enumerate}
        \item Domain and codomain of $g\circ f$,
        \item A formula for $g\circ f$.
    \end{enumerate}
\end{problem}

\begin{problem}
    Let $A = \mathbb{R} \setminus \{1\}$, and let $f: A \rightarrow A$ be defined as follows: $f(x) = \frac{x+1}{x-1}$.

    \begin{enumerate}[a.]
        \item  Show that $f$ is one-to-one and onto.
        \item Show that $f \circ f = \text{id}_A$.
    \end{enumerate}

\end{problem}

\begin{problem}
    Let $A = \mathcal{P}(\mathbb{R})$. Define $f: \mathbb{R} \rightarrow A$ by the formula $f(x) = \{y \in \mathbb{R} \,|\, y^2 < x\}$.

    \begin{enumerate}[a.]
        \item Find $f(2)$.
        \item Is $f$ one-to-one? Is it onto?
    \end{enumerate}

\end{problem}

\begin{problem}
    Suppose $f: A \rightarrow B$ and $g: B \rightarrow C$. 
    \begin{enumerate}
        \item If $f$ and $g$ are both one-to-one, then so is $g \circ f$.
        \item If $f$ and $g$ are both onto, then so is $g \circ f$.
    \end{enumerate}

\end{problem}
\end{document}
