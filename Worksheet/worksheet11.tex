\documentclass[12pt]{amsart}
\usepackage{amsaddr}
\usepackage{marktext} 
%% Remove draft for real article, put twocolumn for two columns
\usepackage{svmacro}
\usepackage[utf8]{inputenc}
\usepackage{lineno}
\usepackage[style=alphabetic, backend=biber]{biblatex}
\addbibresource{bibliography.bib}

%% commentary bubble
\newcommand{\SV}[2][]{\sidenote[colback=green!10]{\textbf{SV\xspace #1:} #2}}

%% Title 
\title{ MATH 102: Ideas  of Math }
\author{ Worksheet 11 }

\date{Nov 14, 2023}

\begin{document}

\maketitle

\section{Concepts}

\begin{definition}
    Let \( f: X \rightarrow Y \) be a function. A left inverse (or post-inverse) for \( f \) is a function \( g: Y \rightarrow X \) such that \( g \circ f = \text{id}_X \).
\end{definition}

\begin{definition}
    Let \( f: X \rightarrow Y \) be a function. A right inverse (or pre-inverse) for \( f \) is a function \( g: Y \rightarrow X \) such that \( f \circ g = \text{id}_Y \).
\end{definition}


\section{Problems}

\begin{problem}
    Let $D \subseteq \Q$ be the set of \emph{dyadic rational numbers}, that is
    \begin{equation*}
        D =\left\{x \in \Q | \exists a \in \Z, \exists n\in \N, x = \frac{a}{2^n} \right\} \,.
    \end{equation*}
    Let $k \in \N$ and define $f:D\to D$ by
    \begin{equation*}
        f(x) = \frac{x}{2^k} \,.
    \end{equation*}
    Show that $f$ is bijective.
\end{problem}

\begin{problem}
    \begin{enumerate}
        \item Are left and right inverses the same?
        \item When can a function have right inverse?
        \item When can a function have left inverse?
    \end{enumerate}



\end{problem}

\end{document}
