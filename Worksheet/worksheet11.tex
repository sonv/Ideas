\documentclass[12pt]{amsart}
\usepackage{amsaddr}
\usepackage{marktext} 
%% Remove draft for real article, put twocolumn for two columns
\usepackage{svmacro}
\usepackage[utf8]{inputenc}
\usepackage{lineno}
\usepackage[style=alphabetic, backend=biber]{biblatex}
\addbibresource{bibliography.bib}

%% commentary bubble
\newcommand{\SV}[2][]{\sidenote[colback=green!10]{\textbf{SV\xspace #1:} #2}}

%% Title 
\title{ MATH 102: Ideas  of Math }
\author{ Worksheet 11 \\ adapted from \url{https://www.math.toronto.edu/lshorser/Summer2017/PUMPII-Quantifiers-Complete.pdf} }

\date{Oct 31, 2024}

\begin{document}

\maketitle

\section{Review}
\begin{problem}
Consider the example in the book:

\begin{enumerate}
	\item Good food is not cheap.
	\item Cheap food is not good.
\end{enumerate}
Do the two sentences mean the same?
\end{problem}

\begin{problem}
Watch
\url{https://www.youtube.com/watch?v=fQ3Md3CZxks}

What's going on?
Sketch the truth table to describe the situation in the video.
\end{problem}


\begin{problem}[name of the problem]
Watch
\url{https://www.youtube.com/watch?v=zqSlij2Idgg}

What's going on?
Sketch the truth table to describe the situation in the video.
\end{problem}


\begin{problem}[name of the problem]
Watch
\url{https://www.youtube.com/watch?v=juFsA25b9EY}

What's going on?
Sketch the truth table to describe the situation in the video.
\end{problem}

\begin{problem}[name of the problem]
Watch
\url{https://www.youtube.com/watch?v=BUqMNVnELzE&list=PLMpofmkxKHBJfta_JzekLbWGHUSLUJoLt}

What's going on?
Sketch the truth table to describe the situation in the video.
\end{problem}

\section{Quantifiers}

Open sentences are sometimes technically called predicates.
Because open sentences doesn't have a truth value as they may have unknowns, we denote them as something that look
like a function $S(x)$.

For example, we may use $S(n)$ to denote the sentence ``$n$ is even.''

Then, $S(1)$ is false and $S(10)$ is true.

\begin{definition}
	In logic, quantifiers are very important. The most important among them are:
	\begin{enumerate}
		\item Universal quantifier: $\forall$, means ``for all'', ``for every'', ``for each''
		\item Existential quantifier: $\exists$, means ``there exists'', ``there is at least one''
	\end{enumerate}
\end{definition}


\begin{problem}
Translate the logical expression into English and determine their truth values.

\begin{enumerate}
	\item $\forall x \in \Z, 2 \mid  x \implies x \mid 4$
	      \vspace{3cm}
	\item $\forall x\in \Z, 4 \mid x \implies 2 \mid x$
	      \vspace{3cm}
	\item $\exists x \in \N: x< -2$
	      \vspace{3cm}
	\item $\forall x \in \Z, \forall y \Z, x-y = y -x $
	      \vspace{3cm}
	\item $\forall x \in \R, \forall y \in \R, xy = yx$
	      \vspace{3cm}
	\item $\forall x\in \R, \forall y \in R, x + y = 1$
	      \vspace{3cm}
	\item $\exists x\in \R, \forall y \in R, x + y = 1$
	      \vspace{3cm}
	\item $\forall x\in \R, \exists y \in R, x + y = 1$
	      \vspace{3cm}
\end{enumerate}
\end{problem}

\begin{definition}[Negation with quantifiers]
	What do you think the following would be equivalent with?
	\begin{enumerate}
		\item $\sim \left( \forall x \in A, S(x)  \right)$
		      \vspace{3cm}
		\item $\sim \left( \exists x \in A, S(x)  \right)$
		      \vspace{3cm}
	\end{enumerate}
\end{definition}
\newpage

\begin{problem}
Negate all the statements in the previous problem.
\end{problem}

\newpage

\begin{problem}
True or False?

If everyone loves my baby and my baby loves only me, then I am my own baby.
\end{problem}

\end{document}
