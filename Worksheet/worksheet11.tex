\documentclass[12pt]{amsart}
\usepackage{amsaddr}
\usepackage{marktext} 
%% Remove draft for real article, put twocolumn for two columns
\usepackage{svmacro}
\usepackage[utf8]{inputenc}
\usepackage{lineno}
\usepackage[style=alphabetic, backend=biber]{biblatex}
\addbibresource{bibliography.bib}

%% commentary bubble
\newcommand{\SV}[2][]{\sidenote[colback=green!10]{\textbf{SV\xspace #1:} #2}}

%% Title 
\title{ MATH 102: Ideas  of Math }
\author{ Worksheet 11 }

\date{Nov 14, 2023}

\begin{document}

\maketitle

\section{Concepts}

\begin{definition}
    Let \( f: X \rightarrow Y \) be a function. A \emph{left inverse} for \( f \) is a function \( g: Y \rightarrow X \) such that \( g \circ f = \text{id}_X \).
\end{definition}

\begin{definition}
    Let \( f: X \rightarrow Y \) be a function. A \emph{right inverse}  for \( f \) is a function \( g: Y \rightarrow X \) such that \( f \circ g = \text{id}_Y \).
\end{definition}

\begin{definition}
    Let $f: x\rightarrow Y$ be a function.
    An \emph{inverse} (or two-sided inverse) of $f$ is a function $g: Y\to X$ that is both left and right inverse.
\end{definition}

\begin{theorem}[Fundatmental theorem of arithmetic]
    Let $a \in \mathbb{N}$ be a nonzero number that is bigger than 1. There exist primes $p_1, \ldots, p_k \in \mathbb{N}$ such that
\[ a = p_1 \cdot \ldots \cdot p_k \]
Moreover, this expression is essentially unique: if $a = q_1 \cdot \ldots \cdot q_l$ is another expression of $a$ as a product of primes, then $k = l$ and, re-ordering the $q_i$ if necessary, for each $i$,  $q_i = p_i$.
\end{theorem}


\section{Problems}


\begin{problem}
    \begin{enumerate}
        \item Suppose $X$ is non-empty. Show that function $f:X\to Y$ has a left inverse if and only if it is injective.
        \item Show that a function $f:X \to Y$ has a right inverse if and only if it is surjective.
    \end{enumerate}
\end{problem}

\begin{problem}
    Let $e:\N \times \N \to \N$ be a function defined by
    \begin{equation*}
        e(m,n) = 2^m 3^n \,.
    \end{equation*}
    \begin{enumerate}
        \item Show that $e$ is injective.
        \item Find a left inverse for $e$.
    \end{enumerate}
\end{problem}

\begin{problem}
    Let $f:\R \to [0,\infty)$ be a function defined by 
    \begin{equation*}
        f(x) = x^2 \,.
    \end{equation*}
    \begin{enumerate}
        \item Show that $f$ is surjective.
        \item Find a right inverse for $f$.
    \end{enumerate}
\end{problem}

\begin{problem}
    Let $D \subseteq \Q$ be the set of \emph{dyadic rational numbers}, that is
    \begin{equation*}
        D =\left\{x \in \Q \vert \exists a \in \Z, \exists n\in \N \cup \{ 0 \}, x = \frac{a}{2^n} \right\} \,.
    \end{equation*}
    Let $k \in \N$ and define $f:D\to D$ by
    \begin{equation*}
        f(x) = \frac{x}{2^k} \,.
    \end{equation*}
    Show that $f$ is bijective.
\end{problem}





\end{document}
