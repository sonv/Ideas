\documentclass[12pt]{amsart}
\usepackage{amsaddr}
\usepackage{marktext} 
%% Remove draft for real article, put twocolumn for two columns
\usepackage{svmacro}
\usepackage[utf8]{inputenc}
\usepackage{lineno}
\usepackage[style=alphabetic, backend=biber]{biblatex}
\addbibresource{bibliography.bib}

%% commentary bubble
\newcommand{\SV}[2][]{\sidenote[colback=green!10]{\textbf{SV\xspace #1:} #2}}

%% Title 
\title{ MATH 102: Ideas  of Math }
\author{ Worksheet 12 }

\date{Nov 24, 2023}

\begin{document}

\maketitle

\section{Concepts}


\begin{theorem}[The Induction Principle]
Suppose that we have a sequence of statements $P(n)$ labeled by the natural numbers $1,2,\dots$ such that we know that 
\begin{enumerate}
    \item $P(1)$ is true, and
    \item $\big( P(1) \wedge P(2) \wedge \cdots \wedge P(n) \big) \Rightarrow P(n+1)$.
\end{enumerate}
  Then all the statements $P(1), P(2), \dots$ are true.
\end{theorem}
We will not discuss the proof of this theorem. However, curious minds can find the proof
in Section 4.2 of Newstead.

\begin{proof}[Structure of induction proof]
    Induction proofs typically have the following structure:
\begin{enumerate}
\item Identify the statements $P(n)$.
\item Step 2 is also called \emph{base of induction}: prove $P(0)$ or $P(1)$ (sometimes it doesn't make sense to talk about $P(0)$).
\item Assume that all $P(k)$ with $k\leq n$ are true -- this is called the \emph{induction hypothesis}. Now perform the \emph{step of induction}: prove that $P(n+1)$ is true.
\item Finally, conclude by the Principle of Induction that all $P(n)$ are true.
\end{enumerate}
\end{proof}


\section{Problems}

\begin{problem}
    Prove that 
    \begin{equation*}
        1+2+\dots + n = \frac{n(n+1)}{2} \,.
    \end{equation*}
\end{problem}

\begin{proof}
    Done.
\end{proof}

\begin{problem}
    Prove that
    for all natural number $n$,
    $n^3 - n$ is divisible by 3.
\end{problem}

\begin{proof}
    Step 1. $P(n) =$ ``$n^3 - n$ is divisible by 3.''
   
    Step 2. Base case: For $n=1$,
    \begin{equation*}
        1^3 - 1 = 0, 
    \end{equation*}
    which is divisible by 3.

    Step 3. (Induction hypothesis) Assume now that $P(k)$ is true for some $k\in \N$, i.e.,
    \begin{equation*}
        k^3 - k = 3 m
    \end{equation*}
    for some $m\in \N$.

    We want to show that $P(k+1)$ is true. To see this, consider
    \begin{align*}
        (k+1)^3 - (k+1) &= k^3 + 3k^2 + 3k + 1 - k - 1 \\
                        &= k^3 - k + 3k^2 + 3k\\
                        &= 3m + 3k^2 + 3k = 3( m + k^2 + k) = 3 l \,,
    \end{align*}
    where $l = m + k^2 + k \in \N$. By definition of divisibility,
    \begin{equation*}
        3 | (k+1)^3 - (k+1) \,.
    \end{equation*}
    So, $P(k+1)$ is true.

    Step 4. By induction, $P(n)$ is true for all $n\in \N$.
\end{proof}

\begin{problem}
    Prove by induction that
    for all $n \in \N$,
    \begin{equation*}
        2^0 + 2^1 + 2^2 + \dots+ 2^n = 2^{n+1}-1 \,.
    \end{equation*}
\end{problem}

\begin{proof}
    Step 1. $P(n) =$ ``$2^0 + 2^1 + 2^2 + \dots+ 2^n = 2^{n+1}-1$'' 

    Step 2. Base case. For $n =1$, we have
    \begin{equation*}
        2^0 + 2^1 = 3 = 2^{1+1} - 1 \,.
    \end{equation*}
    So $P(1)$ is true.


    Step 3. Suppose that $P(k)$ is true for some $k\in \N$, i.e.,
    \begin{equation*}
        2^0 + 2^1 + 2^2 + \dots+ 2^k = 2^{k+1}-1 \,.
    \end{equation*}

    We want to show that $P(k+1)$ is true. To do this, consider
    \begin{equation*}
        2^0 + 2^1 + 2^2 + \dots+ 2^k + 2^{k+1} = 2^{k+1} - 1 + 2^{k+1} = 2\cdot  2^{k+1} - 1 = 2^{k+2} - 1\,.
    \end{equation*}
    So, $P(k+1)$ is true.

    Step 4. By induction, $P(n)$ is true for all $n\in \N$.
\end{proof}


\begin{problem}
    For all $n \geq 4$, we have $3n < 2^n$.
\end{problem}

\begin{proof}
    Step 1. $P(n) =$ ``$3n < 2^n$''. 

    Step 2. Base case. For $n =4$, we have
        \begin{equation*}
            3\cdot 4 = 12 < 16 = 2^4 \,.
        \end{equation*}
    So $P(4)$ is true.


    Step 3. Suppose that $P(k)$ is true for some $k\in \N$ and $k\geq 4$, i.e.,
    \begin{equation*}
        3 k < 2^k \,.
    \end{equation*}

    We want to show that $P(k+1)$ is true. To do this, consider
        \begin{align*}
            3(k+1) = 3k + 3 < 2^k + 3 < 2^k + 2^k  = 2^{k+1}
        \end{align*}
        because $k\geq 4$.
    So, $P(k+1)$ is true.

    Step 4. By induction, $P(n)$ is true for all $n\in \N$ and $n\geq 4$.
\end{proof}




\end{document}
