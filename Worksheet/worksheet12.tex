\documentclass[12pt]{amsart}
\usepackage{amsaddr}
\usepackage{marktext} 
%% Remove draft for real article, put twocolumn for two columns
\usepackage{svmacro}
\usepackage[utf8]{inputenc}
\usepackage{lineno}
\usepackage[style=alphabetic, backend=biber]{biblatex}
\addbibresource{bibliography.bib}

%% commentary bubble
\newcommand{\SV}[2][]{\sidenote[colback=green!10]{\textbf{SV\xspace #1:} #2}}

%% Title 
\title{ MATH 102: Ideas  of Math }
\author{ Worksheet 12 }

\date{Nov 204, 2023}

\begin{document}

\maketitle

\section{Concepts}

\begin{definition}

\begin{theorem}[The Induction Principle]
Suppose that we have a sequence of statements $P(n)$ labeled by the natural numbers $1,2,\dots$ such that we know that 
\begin{enumerate}
    \item $P(1)$ is true, and
    \item $\big( P(1) \wedge P(2) \wedge \cdots \wedge P(n) \big) \Rightarrow P(n+1)$.
\end{enumerate}
  Then all the statements $P(1), P(2), \dots$ are true.
\end{theorem}
We will not discuss the proof of this theorem. However, curious minds can find the proof
in Section 4.2 of Newstead.

\begin{proof}[Structure of induction proof]
    Induction proofs typically have the following structure:
\begin{enumerate}
\item Identify the statements $P(n)$.
\item Step 2 is also called \emph{base of induction}: prove $P(0)$ or $P(1)$ (sometimes it doesn't make sense to talk about $P(0)$).
\item Assume that all $P(k)$ with $k\leq n$ are true -- this is called the \emph{induction hypothesis}. Now perform the \emph{step of induction}: prove that $P(n+1)$ is true.
\item Finally, conclude by the Principle of Induction that all $P(n)$ are true.
\end{enumerate}
\end{proof}


\section{Problems}

\begin{problem}
    Prove that 
    \begin{equation*}
        1+2+\dots + n = \frac{n(n+1)}{2} \,.
    \end{equation*}
\end{problem}

\begin{problem}
    Prove that
    for all natural number $n$,
    $n^3 - n$ is divisible by 3.
\end{problem}






\end{document}
