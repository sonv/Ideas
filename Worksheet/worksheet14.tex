\documentclass[12pt]{amsart}
\usepackage{amsaddr}
\usepackage{marktext} 
%% Remove draft for real article, put twocolumn for two columns
\usepackage{svmacro}
\usepackage[utf8]{inputenc}
\usepackage{lineno}
\usepackage[style=alphabetic, backend=biber]{biblatex}
\addbibresource{bibliography.bib}

%% commentary bubble
\newcommand{\SV}[2][]{\sidenote[colback=green!10]{\textbf{SV\xspace #1:} #2}}

%% Title 
\title{ MATH 102: Ideas  of Math }
\author{ Worksheet 14 }

\date{Nov 14, 2024}

\begin{document}

\maketitle

\begin{definition}
	Given a pair of sets $A$ and $B$, suppose that each element $x\in A$ is
	associated to a unique element $f(x)$ of $B$.
	Then $f$ is said to be a function from $A$ to $B$.
	This is denoted by
	\begin{equation*}
		f: A\to B \,.
	\end{equation*}
	$A$ is called the \emph{domain} of $f$, $B$ is called the \emph{codomain}
	of $f$.
	The set $\set{ f(x): x\in A  }$ is called the \emph{range} of $f$.
\end{definition}

\begin{question}
	What is the vertical line test?
\end{question}

\vspace{5cm}

\begin{definition}
	A function $f:A \to B$ is \emph{injective} (of \emph{one-to-one}) if
	$f(a_1) = f(a_2)$ implies $a_1 = a_2$.
\end{definition}

\begin{question}
	Give a few examples of  injective function.
\end{question}
\vspace{5cm}

\begin{question}
	Give an equivalent definition of injectivity using contrapositivity.
\end{question}

\begin{definition}
	A function $f:A \to B$ is \emph{surjective} (of \emph{onto}) if
	for every $b\in B$, there exists some $a\in A$ such that $f(a) = b$.
\end{definition}

\begin{question}
	Give a few examples of  injective function.
\end{question}
\vspace{5cm}

\begin{definition}
	A function is \emph{bijective} if it is both injective and surjective.
\end{definition}

\begin{question}
	\begin{enumerate}
		\item Write a strategy to prove injectivity.
		      \vspace{7cm}
		\item Write a strategy to prove surjectivity.
		      \vspace{7cm}
	\end{enumerate}
\end{question}

\begin{question}
	Let $\mathbb{R}^+$ denote the nonnegative real numbers. Prove the following.

	\begin{itemize}
		\item[(a)] $f : \mathbb{R} \to \mathbb{R}$ where $f(x) = x^2$ is not injective, surjective or bijective.
		      \vspace{5cm}
		\item[(b)] $g : \mathbb{R}^+ \to \mathbb{R}$ where $g(x) = x^2$ is injective, but not surjective or bijective.
		      \vspace{5cm}
		\item[(c)] $h : \mathbb{R} \to \mathbb{R}^+$ where $h(x) = x^2$ is surjective, but not injective or bijective.
		      \vspace{5cm}
		\item[(d)] $k : \mathbb{R}^+ \to \mathbb{R}^+$ where $k(x) = x^2$ is injective, surjective and bijective.
		      \vspace{5cm}
	\end{itemize}
\end{question}


\begin{definition}
	Let $A, B$ be sets. A Cartesian product of $A$ and $B$, denoted by $A\times B$
	is the set of all ordered pair $(x,y)$, where $x\in A$ and $y \in B$.
	\begin{equation*}
		A\times B = \set{ (x,y) : x\in A, y\in B}\,.
	\end{equation*}
\end{definition}

\begin{question}
	Prove the function $f:\Z\times \Z \to \Z\times \Z$
	where
	$f(x,y) = (x+2y, 2x+3y)$
	is a bijection.
\end{question}



\end{document}
