\documentclass[12pt]{amsart}
\usepackage{amsaddr}
\usepackage{marktext} 
%% Remove draft for real article, put twocolumn for two columns
\usepackage{svmacro}
\usepackage[utf8]{inputenc}
\usepackage{lineno}
\usepackage[style=alphabetic, backend=biber]{biblatex}
\addbibresource{bibliography.bib}

%% commentary bubble
\newcommand{\SV}[2][]{\sidenote[colback=green!10]{\textbf{SV\xspace #1:} #2}}

%% Title 
\title{ MATH 102: Ideas  of Math }
\author{ Worksheet 15 }

\date{Nov 18, 2024}

\begin{document}

\maketitle

\begin{problem}
Suppose $A$ and $B$ are sets with finite number of elements
and $f:A\to B$ is a function.
Can you compare the sizes of $A$ and $B$ based on
injectivity and surjectivity?
\end{problem}

\vspace{7cm}


\begin{definition}
	Let $A,B$, and $C$ bet sets, $g: A\to B$ and $f:B\to C$.
	Then the \emph{composition} function $f\circ g:A\to C$ is defined
	by:
	\begin{equation*}
		(f \circ g) (a) = f(g(a)) \,.
	\end{equation*}
\end{definition}


\begin{theorem}
	Suppose $A,B$ and $C$ are sets.
	If $g:A\to B$ and $f:B\to C$ are injective.
	Then,
	$f\circ g: A\to C$ is injective.
\end{theorem}

\vspace{7cm}

\begin{theorem}
	Suppose $A,B$ and $C$ are sets.
	If $g:A\to B$ and $f:B\to C$ are surjective.
	Then,
	$f\circ g: A\to C$ is surjective.
\end{theorem}

\vspace{7cm}

\begin{theorem}
	Suppose $A,B$ and $C$ are sets.
	If $g:A\to B$ and $f:B\to C$ are bijective.
	Then,
	$f\circ g: A\to C$ is bijective.
\end{theorem}

\vspace{7cm}

\end{document}
