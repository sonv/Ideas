\documentclass[12pt]{amsart}
\usepackage{amsaddr}
\usepackage{marktext} 
%% Remove draft for real article, put twocolumn for two columns
\usepackage{svmacro}
\usepackage[utf8]{inputenc}
\usepackage{lineno}
\usepackage[style=alphabetic, backend=biber]{biblatex}
\addbibresource{bibliography.bib}

%% commentary bubble
\newcommand{\SV}[2][]{\sidenote[colback=green!10]{\textbf{SV\xspace #1:} #2}}

%% Title 
\title{ MATH 102: Ideas  of Math }
\author{ Worksheet 2 }

\date{\today}

\begin{document}

\maketitle

\section{ Logical connectives}

There are four logical connectives (some call them logical operators)

$$\neg, \wedge, \vee, \implies$$

\begin{definition}
    A {\bf propositional variable} is a symbol that represents a proposition.
\end{definition}
\begin{definition}
    A {\bf propositional formula} is an expression that is either a propositional 
    variable, or is built up from simpler propositional formulae using logical
    connectives.
\end{definition}

\begin{problem}
Analyze the form of the following statements.

\begin{enumerate}
    \item Either John went to the store, or we’re out of eggs.
    \item  Joe is going to leave home and not come back.
    \item Either Bill is at work and Jane isn’t, or Jane is at work and Bill isn’t.
    \item If today is Sunday, then I don't have to go to work today.
\end{enumerate}
\end{problem}

\begin{problem}
    \begin{enumerate}
        \item Write a propositional formula that is built from at least two propositional formulae that are not propositional variables.
        \item Cook up an English sentence that has the structure made by the above propositional formula.
    \end{enumerate}
\end{problem}

\begin{problem}
    True or false?
    $(P\wedge \neg P) \implies Q$.
    Here $P$ and $Q$ stand for any sentence/statement.
\end{problem}

\printbibliography 
%\bibliography{refs}
%\bibliographystyle{halpha-abbrv}


\end{document}
