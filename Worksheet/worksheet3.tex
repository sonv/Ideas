\documentclass[12pt]{amsart}
\usepackage{amsaddr}
\usepackage{marktext} 
%% Remove draft for real article, put twocolumn for two columns
\usepackage{svmacro}
\usepackage[utf8]{inputenc}
\usepackage{lineno}
\usepackage[style=alphabetic, backend=biber]{biblatex}
\addbibresource{bibliography.bib}

%% commentary bubble
\newcommand{\SV}[2][]{\sidenote[colback=green!10]{\textbf{SV\xspace #1:} #2}}

%% Title 
\title{ MATH 102: Ideas  of Math }
\author{ Worksheet 3 }

\date{\today}

\begin{document}

\maketitle

\section{Statement}
\begin{problem}
    Classify the following as statements or non-statemetns:
    \begin{enumerate}
        \item I had a hearty laugh at the joke!
        \item He had a fever of $102^o$ but still completed his task.
        \item Why would anyone want to do this?
        \item If you go to the city, you will find what you are seeking.
        \item I am so happy today!
        \item $x > x-1.$
        \item For all integers $x$, $x+2$ is even.
    \end{enumerate}
\end{problem}

\begin{problem}
    Classify whether the following statements are propositions.
    \begin{enumerate}
        \item The Sun is a star.
        \item Every prime number is odd.
        \item The Moon is a star.
        \item Every even number
        \item If 4 is a prime, then FUV is the best university in the world.
        \item  ``$P \implies Q$'' has the same meaning as ``$\neg P \vee Q$''.
    \end{enumerate}
\end{problem}

\section{Logical equivalence}

\begin{definition}
    Let $P$ and $Q$ be logical formulae.
    We say that $P$ and $Q$ are logically equivalent and write $P\equiv Q$
    if $Q$ can be derived from $P$ and $P$ can be derived from $Q$.
\end{definition}

\begin{problem}
    Show that
        $$ P \wedge (Q \vee R) \equiv (P \wedge Q)\vee (P \wedge R) \,,$$
        where $P, Q, R$ are propositional variables.
\end{problem}

It turns out that the following is true.
\begin{theorem}
    Two propositional formulae are logically equivalent if and only if their truth values
are the same under any assignment of truth values to their constituent propositional
variables.
\end{theorem}
Even though the proof of this theorem is beyond the scope of this class, we can use
this to our advantage to check whether $P$ and $Q$ are logically equivalent by
just looking at their truth tables.


\begin{theorem}
    Two propositional formulae are logically equivalent if and only if they have the 
    same truth table.
\end{theorem}

\begin{problem}
    Check the theorem above by redo Problem 1.2.6 and Problem 2.2
\end{problem}


\printbibliography 
%\bibliography{refs}
%\bibliographystyle{halpha-abbrv}


\end{document}
