\documentclass[12pt]{amsart}
\usepackage{amsaddr}
\usepackage{marktext} 
%% Remove draft for real article, put twocolumn for two columns
\usepackage{svmacro}
\usepackage[utf8]{inputenc}
\usepackage{lineno}
\usepackage[style=alphabetic, backend=biber]{biblatex}
\addbibresource{bibliography.bib}

%% commentary bubble
\newcommand{\SV}[2][]{\sidenote[colback=green!10]{\textbf{SV\xspace #1:} #2}}

%% Title 
\title{ MATH 102: Ideas  of Math }
\author{ Worksheet 4 }

\date{\today}

\begin{document}

\maketitle


\section{Logical equivalence}

\begin{definition}
    Let $P$ and $Q$ be logical formulae.
    We say that $P$ and $Q$ are logically equivalent and write $P\equiv Q$
    if $Q$ can be derived from $P$ and $P$ can be derived from $Q$.
\end{definition}

\begin{problem}
    Show that
        $$ P \wedge (Q \vee R) \equiv (P \wedge Q)\vee (P \wedge R) \,,$$
        where $P, Q, R$ are propositional variables.
\end{problem}

It turns out that the following is true.
\begin{theorem}
    Two propositional formulae are logically equivalent if and only if their truth values
are the same under any assignment of truth values to their constituent propositional
variables.
\end{theorem}

\section{Truth table}


\begin{theorem}[De Morgan's laws for logical operators]
   Let $P, Q$ be propositional variables. Then,
   \begin{enumerate}
       \item $\neg (P \vee Q) \equiv (\neg P) \wedge (\neg Q) $,
       \item $\neg (P \wedge Q) \equiv (\neg P) \vee (\neg Q) $ .
   \end{enumerate}
\end{theorem}

\begin{theorem}[Distributive laws]
    Let $P, Q, R$ be propositional variables. Then,
    \begin{enumerate}
        \item $P\wedge (Q\vee R) \equiv (P \wedge Q) \vee (P \wedge R)$,
        \item $P \vee (Q\wedge R) \equiv (P \vee Q) \wedge (P \vee R)$.
    \end{enumerate}
\end{theorem}

\begin{theorem}[Absorption laws]
   Let $P, Q$ be propositional variables. Then,
    \begin{enumerate}
        \item $P \vee (P \wedge Q) \equiv P$,
        \item $P \wedge (P \vee Q) \equiv P$.
    \end{enumerate}
\end{theorem}


\begin{problem}
    Prove the theorems above by truth tables.
\end{problem}

There are a few other laws that are useful to know.
\begin{theorem}[Tautology laws]
    \begin{enumerate}
        \item $P \wedge$ a tautology $\equiv P$
        \item $P \vee$ a tautology $\equiv$ tautology
        \item $\neg$ (a tautology) is a contradiction.
    \end{enumerate}
\end{theorem}

\begin{theorem}[Contradiction laws]
    \begin{enumerate}
        \item $P \wedge$ contradiction  $\equiv$ contradiction.
        \item $P \vee$ contradiction  $\equiv P$ 
        \item $\neg$ (a contradiction) is a tautology.
    \end{enumerate}
\end{theorem}
\begin{problem}[Problem 9 in section 1.2]
    Use truth table to determine which of these statements are tautologies,
which are contradictions, and which are neither:
\begin{enumerate}
    \item $(P \vee Q)\wedge (\neg P \vee \neg Q)$
    \item $(P \vee Q) \wedge (\neg P \wedge \neg Q)$
    \item $(P \vee Q) \vee (\neg P \vee \neg Q)$
    \item $[P \wedge (Q \vee \neg R)] \vee (\neg P \vee R)$.
\end{enumerate}
\end{problem}



\end{document}
