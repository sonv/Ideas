\documentclass[12pt]{amsart}
\usepackage{amsaddr}
\usepackage{marktext} 
%% Remove draft for real article, put twocolumn for two columns
\usepackage{svmacro}
\usepackage[utf8]{inputenc}
\usepackage{lineno}
\usepackage[style=alphabetic, backend=biber]{biblatex}
\addbibresource{bibliography.bib}

%% commentary bubble
\newcommand{\SV}[2][]{\sidenote[colback=green!10]{\textbf{SV\xspace #1:} #2}}

%% Title 
\title{ MATH 102: Ideas  of Math }
\author{ Worksheet 5 }

\date{\today}

\begin{document}

\maketitle

Important notations:
\begin{enumerate}
	\item $\wedge$ stands for ``and''
	\item $\vee$ stands for ``or''
\end{enumerate}

\begin{question}
	What is a set? What is the set-builder notation?
\end{question}

\vspace{5cm}

\begin{question}
	Write the descriptions of
	\begin{enumerate}
		\item Natural numbers
		      \vspace{3cm}
		\item Rational numbers
		      \vspace{3cm}
		\item Real numbers
		      \vspace{3cm}
		\item $3\times 3$ real matrices
		      \vspace{3cm}
		\item Points on the circle of radius $r$
	\end{enumerate}
\end{question}

\begin{question}
	What does it mean for $A\subseteq B$?
\end{question}
\vspace{5cm}

\begin{question}
	Prove that
	\begin{equation*}
		\left\{ x\in \N: 10 \mid n  \right\}
		\subseteq
		\left\{ x\in \N: 2 \mid n  \right\}
	\end{equation*}
\end{question}
\vspace{5cm}

\begin{question}
	Prove that the empty set is a subset of every set.
\end{question}
\vspace{5cm}

\begin{question}
	Prove that
	\begin{equation*}
		\left\{ n\in \Z : 12\mid n   \right\}
		=
		\left\{ n\in \Z : 3\mid n \; \wedge \; 4 \mid n   \right\}
	\end{equation*}
\end{question}




\end{document}
