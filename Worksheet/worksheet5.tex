\documentclass[12pt]{amsart}
\usepackage{amsaddr}
\usepackage{marktext} 
%% Remove draft for real article, put twocolumn for two columns
\usepackage{svmacro}
\usepackage[utf8]{inputenc}
\usepackage{lineno}
\usepackage[style=alphabetic, backend=biber]{biblatex}
\addbibresource{bibliography.bib}

%% commentary bubble
\newcommand{\SV}[2][]{\sidenote[colback=green!10]{\textbf{SV\xspace #1:} #2}}

%% Title 
\title{ MATH 102: Ideas  of Math }
\author{ Worksheet 5 }

\date{\today}

\begin{document}

\maketitle

\section{Set}

\begin{definition}
    A \emph{set} is a collection of objects.
\end{definition}
Set of objects $x$ satisfying some property $P(x)$ is denoted by
\begin{equation}
    \label{e:set}
\set[\big]{ x  \st P(x)} \,.
\end{equation}

Denote
\begin{enumerate}
    \item The set of all integers to be $\Z$
    \item The set of all natural numbers to be $\N$
    \item the set of all rational numbers to be $\Q$
\end{enumerate}


\begin{problem}
    From high school knowledge, try to describe $\Q$ in terms of $\Z$ using set notation~\eqref{e:set}.
\end{problem}


\begin{problem}
    What's the difference between a logical formula and a propositional formula?
\end{problem}

\begin{definition}[Notations]
    The following are standard notations
    \begin{enumerate}
        \item   $x\in X$ represents ``$x$ belongs to set $X$''.
        \item $x\not\in X$ represents ``$x$ does not belong to set $X$''.
    \end{enumerate}
\end{definition}

Typically, a set is written by a capitalized letter, unless it is 
a special set such as $\Z, \N, \Q$.


\section{Logical formula}
 
Last time, we talked briefly about universal quantifier and existential quantifier.

\begin{definition}[\bf The universal quantifier $\forall$]
    If $p(x)$ is a logical formula with free variable with free variable $x$ with domain $X$,
    then $\forall x\in X, p(x)$ is the logical formula defined according to the following rules:
    \begin{itemize}
        \item If $p(x)$ can be derrived from the assumption that $x$ is an arbitrary element of $X$,
            then $\forall x\in X, p(x)$ is true;
        \item If $a\in X$ and  $\forall x \in X, p(x)$ is true, then $p(a)$ is true.
    \end{itemize}
\end{definition}

\begin{definition}[\bf The existential quantifier $\exists$]
    If $p(x)$ is a logical formula with free variable $x$ with domain $X$, then 
    $\exists x\in X, p(x)$ is the logical formula defined according to the following rules:
    \begin{itemize}
        \item If $a\in X$ and $p(a)$ is true, then $\exists x \in X, p(x)$ is true;
        \item If $\exists x\in X, p(x)$ is true, and $q$ can be derived from the assumption
            that $p(a)$ is true for some fixed $a \in X$, then $q$ is true.
    \end{itemize}
\end{definition}


\begin{problem}
    In your own words, re-interpret the definitions of the quantifiers.
\end{problem}





\begin{problem}
    Represent the following sentences in logical formula form. 
    Some sentences need to be rephrased so that things are clear to identify variables and predicates.
    \begin{enumerate}
        \item $x-y$ is rational.
        \item Every even natural number $n\geq 2$ is divisible by $k$.
        \item There is an integer that is divisible by every integer.
        \item There is no greatest odd integer.
        \item Between any two distinct rational numbers is a third distinct rational number.
        \item If any integer has a rational square root, then that root is an integer.
    \end{enumerate}
\end{problem}

\begin{problem}
    Translate the following into English.
    \begin{equation*}
        \forall a \in \R, (a \geq 0 \implies \exists b \in \R, a=b^2) \,.
    \end{equation*}
\end{problem}

\begin{problem}
    Let $P$ be the set of all prime numbers.
    \begin{enumerate}
        \item Translate the following logical formulas into English.
   \begin{enumerate}
       \item $\forall n \in P, (n> 2 \implies (\exists k \in \Z, n = 2k+ 1))$.
       \item $\neg \exists n \in P, (n> 2 \wedge ( \exists k \in \Z, n = 2k))$.
   \end{enumerate}
\item Are they true? How would you go on to prove them?
\item Are both statements talk about the same thing?
    \end{enumerate}
\end{problem}

\begin{problem}
    From the previous problem,
    find equivalent statements to the following
    \begin{equation*}
       \neg (\exists x \in X, P(x)) \,,
    \end{equation*}
and
    \begin{equation*}
       \neg ( \forall x \in X, P(x)) \,.
    \end{equation*}
\end{problem}

Hint: re-call the following from Worksheet 4
\begin{theorem}[De Morgan's laws for logical operators]
   Let $P, Q$ be propositional variables. Then,
   \begin{enumerate}
       \item $\neg (P \vee Q) \equiv (\neg P) \wedge (\neg Q) $,
       \item $\neg (P \wedge Q) \equiv (\neg P) \vee (\neg Q) $ .
   \end{enumerate}
\end{theorem}


\end{document}
