\documentclass[12pt]{amsart}
\usepackage{amsaddr}
\usepackage{marktext} 
%% Remove draft for real article, put twocolumn for two columns
\usepackage{svmacro}
\usepackage[utf8]{inputenc}
\usepackage{lineno}
\usepackage[style=alphabetic, backend=biber]{biblatex}
\addbibresource{bibliography.bib}

%% commentary bubble
\newcommand{\SV}[2][]{\sidenote[colback=green!10]{\textbf{SV\xspace #1:} #2}}

%% Title 
\title{ MATH 102: Ideas  of Math }
\author{ Worksheet 5 }

\date{\today}

\begin{document}

\maketitle

\begin{definition}
    A \emph{set} is a collection of objects.
\end{definition}
Set of objects $x$ satisfying some property $P(x)$ is denoted by
\begin{equation}
    \label{e:set}
\set[\big]{ x  \st P(x)} \,.
\end{equation}

Denote
\begin{enumerate}
    \item The set of all integers to be $\Z$
    \item The set of all natural numbers to be $\N$
    \item the set of all rational numbers to be $\Q$
\end{enumerate}

\begin{problem}
    From high school knowledge, try to describe $\Q$ in terms of $\Z$ using set notation~\eqref{e:set}.
\end{problem}

\begin{problem}
    What's the difference between a logical formula and a propositional formula?
\end{problem}

\begin{problem}
    We short hand the phrase ``$x$ belongs to set $X$'' by $x\in X$.
\end{problem}

\begin{problem}
    Represent the following sentences in logical formula form. 
    Some sentences need to be rephrased so that things are clear to identify variables and predicates.
    \begin{enumerate}
        \item $x-y$ is rational.
        \item Every even natural number $n\geq 2$ is divisible by $k$.
        \item There is an integer that is divisible by every integer.
        \item There is no greatest odd integer.
        \item Between any two distinct rational numbers is a third distinct rational number.
        \item If any integer has a rational square root, then that root is an integer.
    \end{enumerate}
\end{problem}

\begin{problem}
    Translate the following into English.
    \begin{equation*}
        \forall a \in \R, (a \geq 0 \implies \exists b \in \R, a=b^2) \,.
    \end{equation*}
\end{problem}




\end{document}
