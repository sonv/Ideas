\documentclass[12pt]{amsart}
\usepackage{amsaddr}
\usepackage{marktext} 
%% Remove draft for real article, put twocolumn for two columns
\usepackage{svmacro}
\usepackage[utf8]{inputenc}
\usepackage{lineno}
\usepackage[style=alphabetic, backend=biber]{biblatex}
\addbibresource{bibliography.bib}

%% commentary bubble
\newcommand{\SV}[2][]{\sidenote[colback=green!10]{\textbf{SV\xspace #1:} #2}}

%% Title 
\title{ MATH 102: Ideas  of Math }
\author{ Worksheet 6 }

\date{\today}

\begin{document}

\maketitle

Notations:

\begin{enumerate}
	\item $\R$ is the set of all real numbers
	\item $\N$ is the set of all natural numbers
	\item $\Z$ is the set of integers
	\item $\emptyset = \set{}$ is the empty set
	\item $(a,b) = \set{ x\in \R : a<x<b}$
	\item $[a,b) = \set{ x\in \R : a\leq x<b}$
	\item $(a,b] = \set{ x\in \R : a< x\leq b}$
	\item $[a,b] = \set{ x\in \R : a\leq x\leq b}$
\end{enumerate}

\begin{problem}
What are the following?
\begin{enumerate}
	\item Union:
	      \vspace{3cm}
	\item Intersection:
	      \vspace{3cm}
	\item Subtraction:
	      \vspace{3cm}
	\item Complement:
	      \vspace{3cm}
\end{enumerate}
\end{problem}

\begin{problem}
\begin{enumerate}
	\item Prove that $\Q\subseteq \R$.
	      \vspace{5cm}
	\item Prove that $\R \not\subseteq \Q$.
	      \vspace{5cm}
\end{enumerate}
\end{problem}

\begin{problem}
Prove that,
\begin{equation*}
	\set[\big]{ x \in \R \st x^2 \leq 1 } = [-1,1] \,.
\end{equation*}
\vspace{5cm}
\end{problem}

\begin{problem}
Prove that for any two sets $X$ and $Y$,
\begin{equation*}
	X\cap Y \subseteq X \cup Y \,.
\end{equation*}
\vspace{5cm}
\end{problem}

\begin{problem}
Prove that
\begin{equation*}
	\bigcap_{n\geq 1} \left[ 0, 1+ \frac{1}{n} \right) = [0,1] \,.
\end{equation*}
\vspace{5cm}
\end{problem}

\begin{problem}
Prove that $A \setminus (B \setminus C) \subseteq (A \setminus B) \cup C$.
\vspace{5cm}
\end{problem}


\begin{problem}
Prove that for every integer, the remainder when $x^2$ is divided by 4 is either 0 or 1.
\end{problem}
\vspace{5cm}

\begin{definition}
	Let $X$ be a set. The \emph{power set} of $X$, written $\mathcal{P}(X)$, is the set of all subsets of $X$.

	$$\mathcal{P}(X) = \set{ A: A \subseteq  X }.$$

	For example, $$\mathcal{P}(\set{1,2}) = \set{ \emptyset, \set{1}, \set{2}, \set{1,2}}\,.$$
\end{definition}

\begin{problem}
Write out elements of
\begin{enumerate}
	\item  $\mathcal{P}(\set{1,2,3})$,
	\item $\mathcal{P}(\emptyset)$,
	\item $\mathcal{P}(\mathcal{P}(\emptyset))$,
	\item $\mathcal{P}(\mathcal{P}(\mathcal{P}(\emptyset)))$.
\end{enumerate}
\end{problem}
\vspace{5cm}

The following problem is to help you clear some of the common confusion between
the use of $\in$ and $\subseteq$ in power set.

\begin{problem}
True or false and prove your claim.

\begin{enumerate}
	\item $\mathcal{P}(\emptyset) \in \mathcal{P}(\mathcal{P}(\emptyset))$
	      \vspace{5cm}
	\item $\emptyset \in \set{\set{\emptyset}}$
	      \vspace{5cm}
	\item $\set{\emptyset} \in \set{\set{\emptyset}}$
	      \vspace{5cm}
	\item $\mathcal{P}(\mathcal{P}(\emptyset)) \in \set{\emptyset, \set{\emptyset, \set{\emptyset}}}$
	      \vspace{5cm}
\end{enumerate}
\end{problem}

\begin{problem}
True or False and prove your claim.
\begin{enumerate}
	\item $\mathcal{P}(X\cup Y) = \mathcal{P}(X) \cup \mathcal{P}(Y)$,
	      \vspace{5cm}
	\item $\mathcal{P}(X\cap Y) = \mathcal{P}(X) \cap \mathcal{P}(Y)$.
	      \vspace{5cm}
\end{enumerate}
\end{problem}






\end{document}
