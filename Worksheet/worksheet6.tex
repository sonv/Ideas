\documentclass[12pt]{amsart}
\usepackage{amsaddr}
\usepackage{marktext} 
%% Remove draft for real article, put twocolumn for two columns
\usepackage{svmacro}
\usepackage[utf8]{inputenc}
\usepackage{lineno}
\usepackage[style=alphabetic, backend=biber]{biblatex}
\addbibresource{bibliography.bib}

%% commentary bubble
\newcommand{\SV}[2][]{\sidenote[colback=green!10]{\textbf{SV\xspace #1:} #2}}

%% Title 
\title{ MATH 102: Ideas  of Math }
\author{ Worksheet 6 }

\date{\today}

\begin{document}

\maketitle


\begin{problem}
    \begin{enumerate}
        \item Prove that $\Q\subseteq \R$.
        \item Prove that $\R \not\subseteq \Q$.
    \end{enumerate}
\end{problem}

\begin{problem}
    Prove that, 
   \begin{equation*}
   \set[\big]{ x \in \R \st x^2 \leq 1 } = [-1,1] \,.
   \end{equation*}
\end{problem}

\begin{problem}
    Prove that for any two sets $X$ and $Y$, 
    \begin{equation*}
        X\cap Y \subseteq X \cup Y \,.
    \end{equation*}
\end{problem}

\begin{problem}
    Prove that 
    \begin{equation*}
        \bigcap_{n\geq 1} \left[ 0, 1+ \frac{1}{n} \right) = [0,1] \,.
    \end{equation*}
\end{problem}

\begin{problem}
    Prove that $A \setminus (B \setminus C) \subseteq (A \setminus B) \cup C$.
\end{problem}

Here's a theorem that we will discuss more in depth later.

\begin{theorem}
    For all natural numbers $n$ and $m$, if $m > 0$, then there are natural numbers $q$ and $r$ such that $n = mq + r$ and $r < m$. 
(The numbers $q$ and $r$ are called the quotient and remainder when $n$ is
divided by $m$.)
\end{theorem}

\begin{problem}
    Prove that for every integer, the remainder when $x^2$ is divided by 4 is either 0 or 1.
\end{problem}

\begin{definition}
    Let $X$ be a set. The \emph{power set} of $X$, written $\mathcal{P}(X)$, is the set of all subsets of $X$.

    For example, $$\mathcal{P}(\set{1,2}) = \set{ \emptyset, \set{1}, \set{2}, \set{1,2}}\,.$$ 
\end{definition}

\begin{problem}
    Write out elements of
    \begin{enumerate}
        \item  $\mathcal{P}(\set{1,2,3})$,
        \item $\mathcal{P}(\emptyset)$,
        \item $\mathcal{P}(\mathcal{P}(\emptyset))$,
        \item $\mathcal{P}(\mathcal{P}(\mathcal{P}(\emptyset)))$.
    \end{enumerate}
\end{problem}

\begin{problem}
    True or False and prove your claim.
    \begin{enumerate}
        \item $\mathcal{P}(X\cup Y) = \mathcal{P}(X) \cup \mathcal{P}(Y)$,
        \item $\mathcal{P}(X\cap Y) = \mathcal{P}(X) \cap \mathcal{P}(Y)$.
    \end{enumerate}
\end{problem}





\end{document}
