\documentclass[aspectratio=169]{beamer}
%% Remove draft for real article, put twocolumn for two columns
\usepackage[draft]{svmacro}
\usepackage[utf8]{inputenc}
%Information to be included in the title page:
\title{MATH 102: Ideas of Math}
\author{Day 10}
\date{Sep 28, 2023}

\begin{document}

\frame{\titlepage}

\begin{frame}
\frametitle{More about sets}
\begin{gather*}
[a,b] = \set[\big]{x \in \R \st a \leq x \leq b } \,,\\
[a,b) = \set[\big]{x \in \R \st a \leq x <  b } \,, \\
(a,b] = \set[\big]{x \in \R \st a < x \leq  b } \,, \\
(a,b) = \set[\big]{x \in \R \st a < x <  b } \,.
\end{gather*}
\end{frame}

\begin{frame}
    \frametitle{Set comparisons}
\begin{definition}
    Let $X$ be a set. A \emph{subset} of $X$ is a set $U$ such that 
    \begin{equation*}
        \forall a, ( a \in U \implies a \in X) \,.
    \end{equation*}
    We write $U \subseteq X$ for the assertion that $U$ is a subset of $X$.

    The notation $U \subsetneqq X$ means that $U$ is a \emph{proper subset} of $X$, 
    that is a subset of $X$ that is not equal to $X$.

    The notation $U \not\subseteq X$ means that $U$ is NOT a subset of $X$.
\end{definition}

In order to prove that $U$ is a subset of $X$, it is sufficient to take an 
arbitrary element $a\in U$ and prove that $a \in X$.
\end{frame}

\begin{frame}
    \frametitle{Axiom of extentionality}
   Let $X$ and $Y$ be sets. Then $X = Y$ if and only if $X \subseteq Y$ and $Y\subseteq X$. 
\end{frame}

\begin{frame}
    \frametitle{Empty sets}
    \begin{definition}
    A set is~\emph{non-empty} if it contains at least one element.
    Otherwise, it is~\emph{empty}.
\end{definition}
\end{frame}

\begin{frame}
    Question: How many empty sets are there?
\end{frame}

\begin{frame}
    \frametitle{Set operations}

    \begin{definition}[Pairwise intersection]
    Let $X$ and $Y$ be sets. 
    The \emph{pairwise intersection} of $X$ and $Y$, denoted $X \cap Y$
    is defined by
    \begin{equation*}
        X \cap Y \defeq \set{ a \st a \in X \wedge a \in Y}\,.
    \end{equation*}
\end{definition}
\end{frame}

\begin{frame}
    \frametitle{Set operations}
    \begin{definition}[Pairwise union]
    Let $X$ and $Y$ be sets. 
    The \emph{pairwise union} of $X$ and $Y$, denoted $X \cup Y$
    is defined by
    \begin{equation*}
        X \cup Y \defeq \set{ a \st a \in X \vee a \in Y}\,.
    \end{equation*}
\end{definition}
\end{frame}

\begin{frame}
    \frametitle{Set operations}
    \begin{definition}[Relative complement]
    Let $X$ and $Y$ be sets. 
    The \emph{relative complement} of $Y$ and $X$, denoted $Y \setminus X$
    is defined by
    \begin{equation*}
        Y \setminus X \defeq \set{ a \st a \in Y \wedge a \not\in X}\,.
    \end{equation*}
\end{definition}
\end{frame}


\begin{frame}
    \frametitle{Indexed families of sets}
    \begin{definition}
        An \emph{indexed family of sets} is a specification of a set $X_i$ for each element $i$ of some \emph{indexing set} $I$. We write $\set{X_i \st i\in I}$ for the indexed family of sets.
    \end{definition}

    \begin{example}
        \begin{equation*}
        \set[\Big]{ \left[ 0, 1+ \frac{1}{n} \right) \st n\in \N}
        \end{equation*}
    \end{example}
\end{frame}



\end{document}
