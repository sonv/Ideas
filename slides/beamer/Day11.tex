\documentclass[aspectratio=169]{beamer}
%% Remove draft for real article, put twocolumn for two columns
\usepackage[draft]{svmacro}
\usepackage[utf8]{inputenc}
%Information to be included in the title page:
\title{MATH 102: Ideas of Math}
\author{Day 11}
\date{Oct 2, 2023}

\begin{document}

\frame{\titlepage}

\begin{frame}
    \frametitle{ Some proof strategies }
    \begin{enumerate}
        \item Direct proof
        \item Contradiction
        \item Contrapositive (new)
        \item Proofs with quantifiers
    \end{enumerate}
\end{frame}

\begin{frame}
    \frametitle{ Direct proof }
    Based on definitions only.
    \begin{example}
        Prove that for $a,b>0$,  $a<b \implies a^2 < b^2$.
    \end{example}
\end{frame}

\begin{frame}
    \frametitle{ Proof by contradiction }
    To prove that $P$ is true, suppose $P$ is false then derive a contradiction.
    \begin{example}
        Prove that if $a$ is irrational and $r$ is rational, then $a+r$ is irrational.
    \end{example}
\end{frame}

\begin{frame}
    \frametitle{ Proof by contrapositive}
    In order to prove $P\implies Q$, it is equivalent to prove $\neg Q \implies \neg P$.
    \begin{example}
        Let $m,n \in \N$. Prove that if $mn > 64$, then either $m>8$ or $n>8$.
    \end{example}
\end{frame}

\begin{frame}
    \frametitle{Proof with quantifiers}
    To prove $\forall x \in X, P(x)$, pick an arbitrary $x \in X$ and try to 
    deduce that $P(x)$ is true.
    That is, just use the properties of $x$ that are described in the description of $X$ to deduce
    that $P(x)$ is true.
    \begin{example}
        Prove that the square of every odd integer is odd.
    \end{example}
\end{frame}

\begin{frame}
    \frametitle{Proof with quantifiers (cont.)}
    To prove  $\exists x\in X, P(x)$, it is enough to find a particular $x_0\in X$
    so that $P(x_0)$ is true.
    \begin{example}
        Prove that there is a natural number that is a perfect square and is one more than a
perfect cube.
    \end{example}

\end{frame}


\end{document}
