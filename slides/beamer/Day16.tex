\documentclass[aspectratio=169]{beamer}
%% Remove draft for real article, put twocolumn for two columns
\usepackage[draft]{svmacro}
\usepackage[utf8]{inputenc}
\usepackage{verbatim}

%Information to be included in the title page:
\title{MATH 102: Ideas of Math}
\author{Day 16}
\date{Oct 19, 2023}

\begin{document}

\frame{\titlepage}


\begin{frame}
    \frametitle{Uniqueness}
    Uniqueness is typically coupled with existential quantifier.
    \begin{definition}
        The \emph{unique existential quantifier} is the quantifier $\exists!$, defined such that
        $\exists! x\in X, p(x)$ is short hand for
        \begin{equation*}
            (\exists x \in X, p(x)) \wedge (\forall a\in X, \forall b\in X, [(p(a)\wedge p(b)) \implies a = b] \,.
        \end{equation*}
    \end{definition}
\end{frame}
\begin{frame}
    \frametitle{ Function }

    \begin{definition}
    A function $f$ from a set $X$ to a set $Y$ is a specification of elements 
    $f(x) \in Y$ for $x\in X$ such that
    \begin{equation*}
        \forall x \in X, \exists! y \in Y, y = f(x) \,.
    \end{equation*}
    Given $x\in X$, the unique element $f(x)\in Y$ is called the value of $f$ at $x$.

    $X$ is called the \emph{domain} of $f$, and $Y$ is called the \emph{codomain}.

    We write $f:X\to Y$ to denote the assertion that $f$ is a function with domain $X$ and codomain $Y$.
    \end{definition}
\end{frame}

\end{document}
