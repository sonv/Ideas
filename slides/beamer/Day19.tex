\documentclass[aspectratio=169]{beamer}
%% Remove draft for real article, put twocolumn for two columns
\usepackage[draft]{svmacro}
\usepackage[utf8]{inputenc}
\usepackage{verbatim}
%Information to be included in the title page:
\title{MATH 102: Ideas of Math}
\author{Day 19}
\date{Nov 8, 2023}

\begin{document}

\frame{\titlepage}
\begin{frame}
    \frametitle{ Functions again }
    \begin{definition}[Injective functions]
        Let $f:A\to B$ be a function.
        $f$ is said to be \emph{injective} (or \emph{one-to-one})
        if 
        \begin{equation*}
            \forall a_1, a_2 \in A, f(a_1) = f(a_2) \implies a_1 = a_2 \,.
        \end{equation*}
        An injective function is called an \emph{injection}.
    \end{definition}

\end{frame}

\begin{frame}
    \begin{definition}[Surjective functions]
        Let $f:A\to B$ be a function.
        $f$ is said to be \emph{surjective} (or \emph{onto}) if
        \begin{equation*}
            \forall b\in B \exists a \in A, f(a) = b \,.
        \end{equation*}
        A surjective function is called a \emph{surjection}.
    \end{definition}
\end{frame}

\begin{frame}
    \begin{definition}[Bijective function]
        A function $f$ is \emph{bijective} if 
        it is both injective and surjective.

        A bijective function is called a \emph{bijection}.
    \end{definition}
\end{frame}




\end{document}
